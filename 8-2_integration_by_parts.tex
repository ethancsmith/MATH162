%!TEX root =  main.tex

\lectureheader{162}{Calculus II}{Integration by parts}{\textit{Thomas' Calculus} \textsection 8.2}

\begin{mdframed}
\begin{minipage}{0.5\linewidth}
\begin{enumerate}
%\item $\DS\int \big(f(x) + g(x)\big)\dee x = \int f(x)\dee x + \int g(x)\dee x$
\item $\DS\int x^n\dee x = \frac{x^{n+1}}{n+1}+C\quad (n\ne -1)$
\item $\DS\int\frac{\dee x}{x} = \ln|x| + C$
\item $\DS\int\E^x\dee x = \E^x + C$
\item $\DS\int\sinh x\dee x = \cosh x+C$
\item $\DS\int\cosh x\dee x=\sinh x + C$
\item $\DS\int\frac{\dee x}{\sqrt{1-x^2}} = \arcsin x + C$
\item $\DS\int\frac{\dee x}{1+x^2} = \arctan x + C$
\item $\DS\int\frac{\dee x}{x\sqrt{x^2-1}} = \arcsec|x|+C$
\item $\DS\int\frac{\dee x}{\sqrt{1+x^2}} = \sinh^{-1} x + C$
\item $\DS\int\frac{\dee x}{\sqrt{x^2-1}} = \cosh^{-1} x + C$
\end{enumerate}
\end{minipage}
\begin{minipage}{0.5\linewidth}
\begin{enumerate}
\setcounter{enumi}{10}
\item $\DS\int\sin x\dee x=-\cos x + C$
\item $\DS\int\cos x\dee x=\sin x +C$
\item $\DS\int\tan x\dee x=\ln|\sec x| + C$
\item $\DS\int\csc x\dee x=-\ln|\csc x+\cot x| + C$
\item $\DS\int\sec x\dee x=\ln|\sec x+\tan x|+C$
\item $\DS\int\cot x\dee x=\ln|\sin x| +C$
\item $\DS\int\sec^2 x\dee x = \tan x + C$
\item $\DS\int\csc x\cot x\dee x=-\csc x+C$
\item $\DS\int\sec x\tan x\dee x = \sec x+C$
\item $\DS\int\csc^2 x\dee x = -\cot x+C$
\end{enumerate}
\end{minipage}
\end{mdframed}

%\begin{example}
%Evaluate $\DS\int_3^5\frac{2x-3}{\sqrt{x^2-3x+1}}\dee x$.
%\end{example}
%
%\newpage
%
%\begin{example}
%Compute the following.
%\begin{enumerate}
%\item $\DS\int\frac{\dee x}{\sqrt{8x-x^2}}$
%\vfill
%\item $\DS\int\big(\cos x\sin 2x + \sin x\cos 2x\big)\dee x$
%\vfill
%\end{enumerate}
%\end{example}
%
%\newpage
%
%\begin{example}
%Evaluate the following.
%\begin{enumerate}
%\item $\DS\int_0^{\pi/4}\frac{\dee x}{1-\sin x}$
%\vfill
%\item $\DS\int\frac{3x^2-7x}{3x+2}\dee x$
%\vfill
%\end{enumerate}
%\end{example}
%
%\newpage
%
%\begin{example}
%Determine the following.
%\begin{enumerate}
%\item $\DS\int\frac{3x+2}{\sqrt{1-x^2}}\dee x$
%\vfill
%\item $\DS\int\frac{\dee x}{(1+\sqrt x)^3}$
%\vfill
%\end{enumerate}
%\end{example}
%
%\newpage

\begin{remark}\,
\begin{itemize}
\item Viewing the chain rule (for differentiation) as a fact about integration led to the technique of $u$-substitution.
\item Similarly, viewing the product rule as a fact about integration leads to a technique known as \textbf{integration by parts}.
\item In particular, the formula
\begin{equation*}
\frac{\dee }{\dee x}f(x)g(x) = g(x)\frac{\dee}{\dee x}f(x) + f(x)\frac{\dee}{\dee x}g(x)
\end{equation*}
is rearranged as
\begin{equation*}
f(x)\frac{\dee}{\dee x}g(x) = \frac{\dee }{\dee x}f(x)g(x) - g(x)\frac{\dee}{\dee x}f(x) 
\end{equation*}
and then integrated to reveal
\begin{equation*}
\int f(x)\frac{\dee}{\dee x}g(x)\dee x = f(x)g(x) - \int g(x)\frac{\dee}{\dee x}f(x) \dee x.
\end{equation*}
\item Putting $u=f(x)$ and $v=g(x)$, we usually remember this in the convenient form
\begin{equation}
\int u\dee v = uv - \int v\dee u.
\end{equation}
\end{itemize}
\end{remark}

\newpage

\begin{example}
Evaluate the following.
\begin{enumerate}
\item $\DS\int x\cos x\dee x$
\vfill
\item $\DS\int\ln x\dee x$
\vfill
\end{enumerate}
\end{example}

\newpage

\begin{example}
Evaluate the following.
\begin{enumerate}
\item $\DS\int x^2\E^x\dee x$
\vfill
\item $\DS\int\E^x\cos x\dee x$
\vfill
\end{enumerate}
\end{example}

\newpage

\begin{example}
Find the area under the curve $y=x\E^x$ over the interval $[0,4]$.
\end{example}
