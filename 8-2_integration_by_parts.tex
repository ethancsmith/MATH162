%!TEX root =  main.tex
\setcounter{chapter}{8}
\setcounter{section}{2}
\setcounter{theorem}{0}
\setcounter{equation}{0}

\lectureheader{162}{Calculus II}{Integration by parts}{\textit{Thomas' Calculus}  \thesection}

%\begin{mdframed}
%\begin{minipage}{0.5\linewidth}
%\begin{enumerate}
%%\item $\DS\int \big(f(x) + g(x)\big)\dee x = \int f(x)\dee x + \int g(x)\dee x$
%\item $\DS\int x^n\dee x = \frac{x^{n+1}}{n+1}+C\quad (n\ne -1)$
%\item $\DS\int\frac{\dee x}{x} = \ln|x| + C$
%\item $\DS\int\E^x\dee x = \E^x + C$
%\item $\DS\int\sinh x\dee x = \cosh x+C$
%\item $\DS\int\cosh x\dee x=\sinh x + C$
%\item $\DS\int\frac{\dee x}{\sqrt{1-x^2}} = \arcsin x + C$
%\item $\DS\int\frac{\dee x}{1+x^2} = \arctan x + C$
%\item $\DS\int\frac{\dee x}{x\sqrt{x^2-1}} = \arcsec|x|+C$
%\item $\DS\int\frac{\dee x}{\sqrt{1+x^2}} = \sinh^{-1} x + C$
%\item $\DS\int\frac{\dee x}{\sqrt{x^2-1}} = \cosh^{-1} x + C$
%\end{enumerate}
%\end{minipage}
%\begin{minipage}{0.5\linewidth}
%\begin{enumerate}
%\setcounter{enumi}{10}
%\item $\DS\int\sin x\dee x=-\cos x + C$
%\item $\DS\int\cos x\dee x=\sin x +C$
%\item $\DS\int\tan x\dee x=\ln|\sec x| + C$
%\item $\DS\int\csc x\dee x=-\ln|\csc x+\cot x| + C$
%\item $\DS\int\sec x\dee x=\ln|\sec x+\tan x|+C$
%\item $\DS\int\cot x\dee x=\ln|\sin x| +C$
%\item $\DS\int\sec^2 x\dee x = \tan x + C$
%\item $\DS\int\csc x\cot x\dee x=-\csc x+C$
%\item $\DS\int\sec x\tan x\dee x = \sec x+C$
%\item $\DS\int\csc^2 x\dee x = -\cot x+C$
%\end{enumerate}
%\end{minipage}
%\end{mdframed}

%\begin{example}
%Evaluate $\DS\int_3^5\frac{2x-3}{\sqrt{x^2-3x+1}}\dee x$.
%\end{example}
%
%\newpage
%
%\begin{example}
%Compute the following.
%\begin{enumerate}
%\item $\DS\int\frac{\dee x}{\sqrt{8x-x^2}}$
%\vfill
%\item $\DS\int\big(\cos x\sin 2x + \sin x\cos 2x\big)\dee x$
%\vfill
%\end{enumerate}
%\end{example}
%
%\newpage
%
%\begin{example}
%Evaluate the following.
%\begin{enumerate}
%\item $\DS\int_0^{\pi/4}\frac{\dee x}{1-\sin x}$
%\vfill
%\item $\DS\int\frac{3x^2-7x}{3x+2}\dee x$
%\vfill
%\end{enumerate}
%\end{example}
%
%\newpage
%
%\begin{example}
%Determine the following.
%\begin{enumerate}
%\item $\DS\int\frac{3x+2}{\sqrt{1-x^2}}\dee x$
%\vfill
%\item $\DS\int\frac{\dee x}{(1+\sqrt x)^3}$
%\vfill
%\end{enumerate}
%\end{example}
%
%\newpage

\begin{remark}\,
\begin{itemize}
\item Viewing the chain rule (for differentiation) as a fact about integration led to the technique of $u$-substitution.
\item Similarly, viewing the product rule as a fact about integration leads to a technique known as \textbf{integration by parts}.
\item In particular, the formula
\begin{equation*}
\frac{\dee }{\dee x}f(x)g(x) = g(x)\frac{\dee}{\dee x}f(x) + f(x)\frac{\dee}{\dee x}g(x)
\end{equation*}
is rearranged as
\begin{equation*}
f(x)\frac{\dee}{\dee x}g(x) = \frac{\dee }{\dee x}f(x)g(x) - g(x)\frac{\dee}{\dee x}f(x) 
\end{equation*}
and then integrated to reveal
\begin{equation*}
\int f(x)\frac{\dee}{\dee x}g(x)\dee x = f(x)g(x) - \int g(x)\frac{\dee}{\dee x}f(x) \dee x.
\end{equation*}
\item Putting $u=f(x)$ and $v=g(x)$, we usually remember this in the convenient form
\begin{equation}
\int u\dee v = uv - \int v\dee u.
\end{equation}
\end{itemize}
\end{remark}

\begin{example}
Evaluate $\DS\int x\cos x\dee x$.
\end{example}
\ifdefined\SOLUTION
\SOLUTION[Solution]{
Let $u = x$ and $\dee v = \cos{x}.$  It follows that $\dee u = \dee x$ and $v = \sin{x}.$  Making the substitution and using integration by parts gives 
\begin{equation*}
    \int x\cos x\dee x
    = x\sin{x} - \int\sin{x}\dee x
    = x\sin{x} + \cos{x} + C.
\end{equation*}
Note that if you let $u = \cos{x}$ and $\dee v = x \dee x$, you get $\dee u = -\sin{x}\dee x$ and $v = \frac{1}{2}x^2.$  Making the substitution and using integration by parts gives 
\begin{equation*}
    \int x\cos x\dee x
    = \frac{1}{2}x^2\cos{x} + \int \frac{1}{2}x^2\sin{x}\dee x.
\end{equation*}
This is a more complicated integral than the original. So, when you're picking $u$ and $\dee v$, you want
\begin{enumerate}
    \item A $\dee v$ that we can integrate on its own.
    \item A $u$ so that $v\dee u$ is simpler than $u\dee v.$
\end{enumerate}
}
\else
\fi
\vfill

\newpage

\begin{example}
Evaluate $\DS\int\ln x\dee x$.
\end{example}
\ifdefined\SOLUTION
\SOLUTION[Solution]{
Let $u = \ln{x}$ and let $\dee v = \dee x.$  So, it follows that $\dee u = \frac{\dee x}{x}$ and $v = x.$  Integrating by parts gives 
\begin{equation*}
    \int\ln x\dee x
    = x\ln{x} - \int x \frac{\dee x}{x}
    = x\ln{x} - \int \dee x
    = x\ln{x} - x + C.
\end{equation*}
}
\else
\fi
\vfill
\begin{remark}\,
\begin{itemize}
\item The logarithm is such an important function that it is probably worth memorizing its integral.
\item The technique of deriving the integral of the logarithm has an even greater lesson to teach us.
If we know how to differentiate a function, but not how to integrate it, maybe integration by parts can help.
\end{itemize}
\end{remark}

\newpage

\begin{example}
Evaluate the following.
\begin{enumerate}
\item $\DS\int x^2\E^x\dee x$
\ifdefined\SOLUTION
\SOLUTION[Solution]{
Let $u = x^2$ and $\dee v = \E^x \dee x.$  It follows that $\dee u = 2x\dee x$ and $v = \E^x.$  Integrating by parts gives 
\begin{equation*}
    \int x^2\E^x\dee x
    = x^2\E^x - \int 2x\E^x\dee x
    = x^2\E^x - 2\int x\E^x\dee x.
\end{equation*}
To integrate $x\E^x,$ integration by parts needs to be used again.  So, let $u = x$ and $\dee v = \E^x\dee x.$  It follows that $\dee u = \dee x$ and $v = \E^x.$  Integrating by parts again gives 
\begin{equation*}
    \int x^2\E^x\dee x
    =x^2\E^x - 2\int x\E^x\dee x
    = x^2\E^x - 2\left( x\E^x - \int e^x \dee x \right)
    = x^2\E^x - 2x\E^x + 2\E^x + C.
\end{equation*}
}
\else
\fi
\vfill

\item $\DS\int\E^x\cos x\dee x$
\ifdefined\SOLUTION
\SOLUTION[Solution]{
Let $u = \E^x$ and let $\dee v = \cos{x} \dee x$.  It follows that $\dee u = \E^x\dee x$ and $v = \sin{x}$.  Integrating by parts gives 
\begin{equation*}
    \int \E^x\cos{x}\dee x
    = \E^x\sin{x} - \int \E^x\sin{x} \dee x
\end{equation*}
Integration by parts needs to be used again.  Let $u = \E^x$ and  $\dee v = \sin{x}.$  So, $\dee u = \E^x\dee x$ and $v = -\cos{x}.$  Integrating by parts again gives 
\begin{equation*}
\begin{split}
    \int \E^x\cos{x}\dee x
    &=\E^x\sin{x} - \left( -\E^x\cos{x} + \int \E^x\cos{x}\dee x\right)\\
    &= \E^x\sin{x} + \E^x\cos{x} - \int \cos{x}\E^x\dee x.
\end{split}
\end{equation*}
At first glance, it looks like we're back where we started, but we have a $-1$ in front of the integral on the right and not a $+1$.
This allows us to solve the above equation for the integral.
In particular, we get
\begin{equation*}
    2\int \E^x\cos{x}\dee x = \E^x\cos{x} + \E^x\sin{x} + C,
\end{equation*}
and so
\begin{equation*}
    \int \E^x\cos{x}\dee x = \frac{1}{2}\E^x\cos{x} + \frac{1}{2}\E^x\sin{x} + C.
\end{equation*}
}
\else
\fi
\vfill

\end{enumerate}
\end{example}

\newpage

\begin{remark}
Integration by parts works for definite integral as well so long as you remember to evaluate and subtract appropriately.
In particular, the definite integral formula can be written as
\begin{equation*}
    \int_{a}^{b} u \,\dee v\ 
    = uv\Bigg|_a^b -  \int_{a}^{b} v \,\dee u\ 
\end{equation*}
Alternatively, one may compute the indefinite integral first, and then apply the result.
\end{remark}

\begin{example}
Find the area under the curve $y=x\E^x$ over the interval $[0,4]$.
\end{example}
\ifdefined\SOLUTION
\SOLUTION[Solution]{
Let $u = x$ and $\dee u = \E^x \dee x$.  So, $\dee u = \dee x$ and $v = \E^x$.  We proceed by integration by parts.
\begin{equation*}
    A =  \int_{0}^{4} x\E^x \,\dee x\
    = x\E^x\Bigg|_0^4 - \int_{0}^{4} \E^x \, \dee x\
    = 4\E^4 - (\E^x)\Bigg|_0^4
    = 3\E^4 + 1.
\end{equation*}
}
\else
\fi
\vfill
