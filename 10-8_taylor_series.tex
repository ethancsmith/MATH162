%!TEX root =  main.tex
\setcounter{chapter}{10}
\setcounter{section}{8}
\setcounter{theorem}{0}
\setcounter{equation}{0}

\lectureheader{162}{Calculus II}{Taylor series}{\textit{Thomas' Calculus}  \thesection}

\begin{definition}
Suppose that $f$ is $d$ times differentiable at $x=a$.
Then the \textbf{Taylor polynomial of order $d$ about $x=a$ for $f$} is
\begin{equation*}
P_f(x; d, a) = \sum_{n=0}^d\frac{f^{(n)}(a)}{n!}(x-a)^n.
\end{equation*}
\end{definition}
\begin{remark}
The Taylor polynomial of order $1$ about $x=a$ is the linearization of $f$ at $x=a$.
It would be reasonable to hope that Taylor polynomials provide good approximations to $f(x)$ near $x=a$.
We will take up this topic later.
\end{remark}

\vfill 

\begin{definition}
Suppose that $f$ has derivatives of all orders at $x=a$.
Then the \textbf{Taylor series about $x=a$ for $f$} is
\begin{equation*}
T_f(x; a) = \sum_{n=0}^\infty \frac{f^{(n)}(a)}{n!}(x-a)^n.
\end{equation*}
The special case when $a=0$ is also called a \textbf{Maclaurin series}.
\end{definition}

\begin{remark}
The Taylor polynomial of order $d$ is the $d$th partial sum of the Taylor series.
\end{remark}

\vfill

\newpage

\begin{example}
Compute the Taylor polynomial of order $3$ about $x=2$ for $f(x) = 1/x$.
Then compute the Taylor series.
\end{example}

\ifdefined\SOLUTION
\SOLUTION{
First we observe that
\begin{align*}
f(x) &= x^{-1},\\
f'(x) &= -x^{-2},\\
f''(x) &= (-1)(-2)x^{-3},\\
f'''(x) &= (-1)(-2)(-3)x^{-4},\\
f^{(4)}(x) &= (-1)(-2)(-3)(-4)x^{-5},\\
&\vdots
\end{align*}
So, we surmise that
\begin{equation*}
f^{(n)}(x) = (-1)^{n}n!x^{-(n+1)} \quad (n\ge 0).
\end{equation*}
Whence,
\begin{equation*}
\frac{f^{(n)}(2)}{n!} = \frac{(-1)^{n}n!2^{-(n+1)}}{n!} = \frac{(-1)^n}{2^{n+1}} \quad (n\ge 0).
\end{equation*}
Therefore, the Taylor polynomial of order $3$ for $f(x)=1/x$ about $x=2$ is
\begin{equation*}
\begin{split}
P_f(x;3,2) &= \sum_{n=0}^3\frac{f^{(n)}(2)}{n!}(x-2)^n
= \sum_{n=0}^3\frac{(-1)^{n}}{2^{n+1}}(x-2)^n\\
&=\frac{1}{2}-\frac{1}{4}(x-2)+\frac{1}{8}(x-2)^2-\frac{1}{16}(x-2)^3,
\end{split}
\end{equation*}
and the corresponding Taylor series is
\begin{equation*}
T_f(x;2) = \sum_{n=0}^\infty\frac{(-1)^{n}}{2^{n+1}}(x-2)^n.
\end{equation*}
}
\fi

\newpage

\begin{example}
Compute the Maclaurin series for $f(x) = \exp x$.
\end{example}

\ifdefined\SOLUTION
\SOLUTION{
Since $f^{(n)}(x) = \exp x$ for all $n\ge 0$, we have that
\begin{equation*}
\frac{f^{(n)}(0)}{n!} = \frac{\exp(0)}{n!} = \frac{1}{n!}\quad (n\ge 0).
\end{equation*}
Whence the Maclaurin series for $f(x)=\exp(x)$ is
\begin{equation*}
T_f(x;0) = \sum_{n=0}^\infty\frac{f^{(n)}(0)}{n!}x^n = \sum_{n=0}^\infty\frac{x^n}{n!}.
\end{equation*}
}
\fi

\newpage

\begin{example}
Compute the Maclaurin series for $f(x) = \cos x$.
\end{example}

\ifdefined\SOLUTION
\SOLUTION{
We observe that
\begin{align*}
f(x) &=\cos x,\\
f'(x) &=-\sin x,\\
f''(x) &=-\cos x,\\
f'''(x) &=\sin x,\\
f^{(4)}(x) &=\cos x\\
&\vdots
\end{align*}
and so we surmise that
\begin{equation*}
f^{(n)}(x) = \begin{cases}
(-1)^k\cos x & \text{if } n=2k,\\
(-1)^{k+1}\sin x & \text{if } n=2k+1.
\end{cases}
\end{equation*}
Whence
\begin{equation*}
\frac{f^{(n)}(0)}{n!} = \begin{cases}
\frac{(-1)^k}{(2k)!} & \text{if } n=2k,\\
0 & \text{if } n=2k+1.
\end{cases}
\end{equation*}
Therefore, the Maclaurin series for $f(x)=\cos x$ is
\begin{equation*}
T_f(x;0) = \sum_{k=0}^\infty (-1)^k\frac{x^{2k}}{(2k)!}.
\end{equation*}
}
\fi

\newpage

\begin{example}
Compute the Maclaurin series for $f(x) =\sqrt{1+x}$.
\end{example}

\ifdefined\SOLUTION
\SOLUTION{
First, we observe that
\begin{align*}
f(x) &= (1+x)^{1/2},\\
f'(x) &= (1/2)(1+x)^{-1/2},\\
f''(x) &= (-1/2)(1/2)(1+x)^{-3/2},\\
f'''(x) &= (-3/2)(-1/2)(1/2)(1+x)^{-5/2},\\
f^{(4)}(x) &= (-5/2)(-3/2)(-1/2)(1/2)(1+x)^{-7/2},\\
&\vdots
\end{align*}
Thus, we surmise that
\begin{equation*}
f^{(n)}(x) = (-1)^{n-1}\frac{\prod_{j=1}^{n-1}(2j-1)}{2^n}(1+x)^{-(2n-1)/2} \quad (n\ge 1)
\end{equation*}
so that
\begin{equation*}
\frac{f^{(n)}(0)}{n!}
=\begin{cases}
1& \text{if } n=0,\\
(-1)^{n-1}\frac{\prod_{j=1}^{n-1}(2j-1)}{2^n}& \text{if } n\ge 1.
\end{cases}
\end{equation*}
Therefore, the Maclaurin series for $f(x)=\sqrt{1+x}$ is
\begin{equation*}
T_f(x;0) = 1 + \sum_{n=1}^\infty(-1)^{n-1}\frac{\prod_{j=1}^{n-1}(2j-1)}{2^{n}n!}x^n.
\end{equation*}
}
\fi
