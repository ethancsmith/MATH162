%!TEX root =  main.tex

\lectureheader{162}{Calculus II}{Taylor series}{\textit{Thomas' Calculus} \textsection 10.8}

\begin{definition}
Suppose that $f$ is $d$ times differentiable at $x=a$.
Then the \textbf{Taylor polynomial of order $d$ about $x=a$ for $f$} is
\begin{equation*}
P_f(x; d, a) = \sum_{n=0}^d\frac{f^{(n)}(a)}{n!}(x-a)^n.
\end{equation*}
\end{definition}
\begin{remark}
The Taylor polynomial of order $1$ about $x=a$ is the linearization of $f$ at $x=a$.
It would be reasonable to hope that Taylor polynomials provide good approximations to $f(x)$ near $x=a$.
We will take up this topic later.
\end{remark}

\vfill 

\begin{definition}
Suppose that $f$ has derivatives of all orders at $x=a$.
Then the \textbf{Taylor series of order $d$ about $x=a$ for $f$} is
\begin{equation*}
T_f(x; a) = \sum_{n=0}^\infty \frac{f^{(n)}(a)}{n!}(x-a)^n.
\end{equation*}
The special case when $a=0$ is also called a \textbf{Maclaurin series}.
\end{definition}

\begin{remark}
The Taylor polynomial of order $d$ is the $d$th partial sum of the Taylor series.
\end{remark}

\vfill

\newpage

\begin{example}
Compute the Taylor polynomial of order $3$ about $x=2$ for $f(x) = 1/x$.
Then compute the Taylor series.
\end{example}

\newpage

\begin{example}
Compute the Maclaurin series for $f(x) = \exp x$.
\end{example}

\newpage

\begin{example}
Compute the Maclaurin series for $f(x) = \cos x$.
\end{example}

\newpage

\begin{example}
Compute the Maclaurin series for $f(x) =\sqrt{1+x}$.
\end{example}
