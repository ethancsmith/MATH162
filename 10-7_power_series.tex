%!TEX root =  main.tex

\lectureheader{162}{Calculus II}{Power series}{\textit{Thomas' Calculus}  10.7}

\begin{definition}
Given an infinite sequence $\{c_n\}_{n=0}^\infty$ of ``coefficients," an expression of the form
\begin{equation*}
\sum_{n=0}^\infty c_n(x-a)^n = c_0 + c_1(x-a) + c_2(x-a)^2+\dots
\end{equation*}
is called a \textbf{power series in $x$ about $x=a$}.
\end{definition}

\begin{remark}\,
\begin{itemize}
\item Given a power series, we can define a function
\begin{equation*}
f(x) = \sum_{n=0}^\infty c_n(x-a)^n.
\end{equation*}
\item The domain of the function is the set of $x$ for which the series converges.
%\item The only $x$ for which convergence is guaranteed \textit{a priori} is $x=a$ where we have $f(a)=c_0$.
\end{itemize}
\end{remark}

\vfill

\begin{theorem}
Given a power series $\DS\sum_{n=k}^\infty c_n(x-a)^n$, there are only 3 possibilities.
\begin{enumerate}
\item There is an $R>0$ so that the series converges absolutely when $|x-a|<R$ and diverges when $|x-a|>R$.
\item The series converges for all $x\in\R$.
\item The series converges only for $x=a$.
\end{enumerate}
\end{theorem}

\begin{definition}\,
The \textbf{interval of convergence} is the set of $x$ for which the power series converges.
\begin{enumerate}
\item In the first case of the theorem, the number $R$ is called the \textbf{radius of convergence}.
\item In the second case of the theorem, we say that the series has \textbf{radius of convergence} $R=\infty$.
\item In the third case of the theorem, we say that the series has \textbf{radius of convergence} $R=0$.
\end{enumerate}
\end{definition}

\begin{remark}
In the first case of the theorem (when $0<R<\infty$), the series may or may not converge at the endpoints $x=a\pm R$.
At $x=a\pm R$,
\begin{equation*}
\sum_{n=k}^\infty\left|c_n(x-a)^n\right| 
=\sum_{n=k}^\infty\left|c_n(\pm R)^n\right| 
=\sum_{n=k}^\infty|c_n|R^n,
\end{equation*}
and so the power series either converges absolutely at both endpoints or it does not converge absolutely at either endpoint.
When the series fails to converge absolutely at the endpoints, we must still check each endpoint for conditional convergence.
\end{remark}

\newpage

\begin{example}
For what values of $x$ does the power series $\DS\sum_{n=1}^\infty (-1)^{n-1}\frac{(2x+3)^n}{n}$ converge?
\end{example}

\ifdefined\SOLUTION
\SOLUTION{
Observe that
\begin{equation*}
\begin{split}
\rho =\lim_{n\to\infty}\left|\frac{a_{n+1}}{a_n}\right|
&=\lim_{n\to\infty}\left|\frac{(-1)^{n}(2x+3)^{n+1}n}{(n+1)(-1)^{n-1}(2x+3)^n}\right|\\
&=\lim_{n\to\infty}|2x+3|\frac{n}{n+1}\\
&=\lim_{n\to\infty}|2x+3|\frac{1}{1+1/n}\\
&=|2x+3|.
\end{split}
\end{equation*}
By RaT, the power series converges absolutely if $\rho=|2x+3|<1$ and it diverges if $|2x+3|>1$.
Now $|2x+3|<1$ if and only if $2|x+3/2|<1$ if and only if $|x+3/2|<1/2$.
Thus, the radius of convergence is $R=1/2$, and the center of the interval is $x=-3/2$.
At this point, we know that the power series converges absolutely on $(-2,-1)$.
Convergence at $x=-2$ and at $x=-1$ is still ``up in the air."
At $x=-2$,
\begin{equation*}
\sum_{n=1}^\infty (-1)^{n-1}\frac{(2x+3)^n}{n}
=\sum_{n=1}^\infty (-1)^{n-1}\frac{(-1)^n}{n}
=\sum_{n=1}^\infty \frac{-1}{n}
\end{equation*}
diverges since it is the negative of the harmonic series which diverges.
At $x=-1$,
\begin{equation*}
\sum_{n=1}^\infty (-1)^{n-1}\frac{(2x+3)^n}{n}
=\sum_{n=1}^\infty (-1)^{n-1}\frac{(1)^n}{n}
=\sum_{n=1}^\infty \frac{(-1)^{n-1}}{n}
\end{equation*}
converges since it is equal to the alternating harmonic series which we already know converges conditionally.
Therefore, the power series $\DS\sum_{n=1}^\infty (-1)^{n-1}\frac{(2x+3)^n}{n}$ converges on the interval $(-2,-1]$.
}
\fi

\newpage

\begin{example}
For what values of $x$ does the power series $\DS\sum_{n=0}^\infty n!x^n$ converge?
\end{example}

\ifdefined\SOLUTION
\SOLUTION{
Observe that
\begin{equation*}
\rho = \lim_{n\to\infty}\left|\frac{(n+1)!x^{n+1}}{n!x^n}\right|
=\lim_{n\to\infty}(n+1)|x|
=\begin{cases}
0 & \text{if } x = 0,\\
\infty & \text{if } x\ne 0.
\end{cases}
\end{equation*}
By RaT, this power series converges only at $x=0$, where it converges to
\begin{equation*}
\sum_{n=0}^\infty n! 0^n = 1 + 0 + 0 +\dots = 1.
\end{equation*}
}
\fi

\newpage

\begin{example}
For what values of $x$ does the power series $\DS\sum_{n=0}^\infty \frac{x^n}{n!}$ converge?
\end{example}

\ifdefined\SOLUTION
\SOLUTION{
Observe that
\begin{equation*}
\rho = \lim_{n\to\infty}\left|\frac{x^{n+1}n!}{x^n(n+1)!}\right|
=\lim_{n\to\infty}\frac{|x|}{n+1}
=0.
\end{equation*}
By RaT, this power series converges absolutely for all $x\in\R=(-\infty,\infty)$.
The radius of convergence is $R=\infty$.
}
\fi
\newpage

\begin{remark}
Once we know where a power series converges, we can try to determine its sum.
\end{remark}

\begin{example}
Determine the interval of convergence for each of the power series
\begin{equation*}
f(x) = \sum_{n=0}^\infty x^n\quad\text{ and }\quad g(x) =\sum_{n=0}^\infty (-1)^{n+1}(x-2)^n.
\end{equation*}
How do they relate to the function $F(x)=\DS\frac{1}{1-x}$?
\end{example}

\ifdefined\SOLUTION
\SOLUTION{
By GST, the series 
\begin{equation*}
f(x) = \sum_{n=0}^\infty x^n
\end{equation*}
converges if and only if $|x|<1$.
Thus, $f(x)$ has radius of convergence $R_f=1$ and interval of convergence $D_f=(-1,1)$.
By GST, the series 
\begin{equation*}
g(x)=\sum_{n=0}^\infty (-1)^{n+1}(x-2)^n
=\sum_{n=0}^\infty (-1)(2-x)^n
\end{equation*}
converges if and only if $|x-2|=|2-x|<1$.
Thus, $g(x)$ has radius of convergence $R_g=1$ and interval of convergence $D_g = (1,3)$.
Also by GST, we have
\begin{equation*}
f(x) = \sum_{n=0}^\infty x^n = \frac{1}{1-x} = F(x),
\end{equation*}
but only for those $x\in D_f = (-1,1)$.
Similarly, GST tell us that
\begin{equation*}
g(x)=\sum_{n=0}^\infty (-1)^{n+1}(x-2)^n
=\sum_{n=0}^\infty (-1)(2-x)^n
=-\frac{1}{1-(2-x)}
=\frac{-1}{x-1}
=\frac{1}{1-x} 
= F(x)
\end{equation*}
but only for those $x\in D_g = (1,3)$.
It is important to understand that $f(x)=F(x)$ for certain values of $x$ and $g(x)=F(x)$ for certain other values of $x$.
There are no values of $x$ for which $f(x)=g(x)$.
For example,
\begin{equation*}
f(0) = 1 = F(0),
\end{equation*}
but $g(0)$ is undefined/diverges.
Similarly,
\begin{equation*}
g(3/2) = F(3/2) = \frac{1}{1-3/2} = -2,
\end{equation*}
but $f(3/2)$ is not a number.
The power series $f(x)$ and $g(x)$ give us ``snapshots" of $F(x)$ but they aren't the same picture.
}
\fi

\newpage

\begin{theorem}
If the power series
\begin{equation*}
f(x) = \sum_{n=0}^\infty c_n(x-a)^n
\end{equation*}
has radius of convergence $R>0$, then
\begin{align*}
\frac{\dee }{\dee x}f(x) &= \sum_{n=1}^\infty nc_n(x-a)^{n-1},\\
\int f(x)\dee x &= C+\sum_{n=0}^\infty \frac{c_n}{n+1}(x-a)^{n+1}
\end{align*}
wherever $|x-a|<R$.
\end{theorem}

\begin{example}
Determine the interval of convergence for $\DS\sum_{n=1}^\infty\frac{x^n}{n}$.
Then determine the sum of the series for each $x$ in the interior of the interval.
\end{example}

\ifdefined\SOLUTION
\SOLUTION{
First observe that
\begin{equation*}
\rho = \lim_{n\to\infty}\left|\frac{x^{n+1}n}{x^n(n+1)}\right|
= \lim_{n\to\infty}|x|\frac{n}{n+1}
= \lim_{n\to\infty}|x|\frac{1}{1+1/n}
= |x|.
\end{equation*}
By RaT, the power series converges absolutely if $|x|<1$, i.e., $x\in (-1,1)$.
At $x=1$,
\begin{equation*}
\sum_{n=1}^\infty\frac{x^n}{n}
=\sum_{n=1}^\infty\frac{1}{n},
\end{equation*}
the harmonic series, which diverges.
At $x=-1$,
\begin{equation*}
\sum_{n=1}^\infty\frac{x^n}{n}
=\sum_{n=1}^\infty\frac{(-1)^n}{n}
=-\sum_{n=1}^\infty\frac{(-1)^{n+1}}{n}
\end{equation*}
is a multiple of the alternating harmonic series, and so converges conditionally.
Therefore, the interval of convergence is $[-1,1)$.
By the above theorem,
\begin{equation*}
\frac{\dee}{\dee x}\sum_{n=1}^\infty\frac{x^n}{n}
=\sum_{n=1}^\infty x^{n-1}
=\sum_{k=0}^\infty x^k
=\frac{1}{1-x}\quad (|x|<1),
\end{equation*}
and so
\begin{equation*}
\sum_{n=1}^\infty\frac{x^n}{n}
=\int\frac{\dee x}{1-x} 
= -\ln|1-x| + C
\end{equation*}
for all $x\in (-1,1)$ and some $C\in\R$.
Evaluating both sides at $x=0$, we find that
\begin{equation*}
0=\sum_{n=1}^\infty\frac{0^n}{n}
= -\ln|1| + C
=C.
\end{equation*}
Therefore,
\begin{equation*}
\sum_{n=1}^\infty\frac{x^n}{n}
= -\ln|1-x| 
\end{equation*}
for $x\in (-1,1)$.
It turns out that the two sides are equal at $x=-1$ as well, but that requires a much more delicate argument.
}
\fi

\newpage

\begin{example}
Determine the interval of convergence for $\DS\sum_{n=0}^\infty (-1)^n\frac{x^{2n+1}}{2n+1}$.
Then determine the sum of the series for each $x$ in the interior of the interval.
\end{example}

\ifdefined\SOLUTION
\SOLUTION{
Observe that
\begin{equation*}
\rho = \lim_{n\to\infty}\left|\frac{(-1)^{n+1}x^{2n+3}(2n+1)}{(2n+3)(-1)^nx^{2n+1}}\right|
= \lim_{n\to\infty}x^2\frac{2n+1}{2n+3}
= \lim_{n\to\infty}x^2\frac{2+1/n}{2+3/n}
=x^2.
\end{equation*}
By RaT, the power series converges absolutely if $x^2<1$.
Therefore, the power series converges absolutely on $(-1,1)$ and the radius of convergence is $R=1$.
At $x=\pm 1$,
\begin{equation*}
\sum_{n=0}^\infty (-1)^n\frac{x^{2n+1}}{2n+1}
=\sum_{n=0}^\infty (-1)^n\frac{(\pm 1)^{2n+1}}{2n+1}
=\pm\sum_{n=0}^\infty \frac{(-1)^n}{2n+1}.
\end{equation*}
\begin{enumerate}
\item The series is clearly alternating when $x=\pm 1$.
\item $|a_n| = \frac{1}{2n+1}>\frac{1}{2n+3} = |a_{n+1}|$ for all $n\ge 0$, and so the sequence of terms is nonincreasing.
\item $\DS\lim_{n\to\infty}|a_n| = \lim_{n\to\infty}\frac{1}{2n+1} = 0$.
\end{enumerate}
Therefore, the power series convergence when $x=\pm 1$ by AST, and so the interval of convergence is $[-1,1]$.
Now for $x\in (-1,1)$, we have
\begin{equation*}
\frac{\dee}{\dee x}\sum_{n=0}^\infty (-1)^n\frac{x^{2n+1}}{2n+1}
=\sum_{n=0}^\infty (-1)^nx^{2n}
=\sum_{n=0}^\infty (-x)^{2n}
=\frac{1}{1+x^2},
\end{equation*}
and so
\begin{equation*}
\sum_{n=0}^\infty (-1)^n\frac{x^{2n+1}}{2n+1}
=\int\frac{\dee x}{1+x^2} 
=\arctan(x)+C
\end{equation*}
for $x\in (-1,1)$ and some $C\in\R$.
Evaluating both sides at $x=0$, we see that
\begin{equation*}
0=\sum_{n=0}^\infty (-1)^n\frac{0^{2n+1}}{2n+1}
=\arctan(0)+C
=C.
\end{equation*}
Therefore,
\begin{equation*}
\sum_{n=0}^\infty (-1)^n\frac{x^{2n+1}}{2n+1}
=\arctan(x)
\end{equation*}
for $x\in (-1,1)$.
It turns out that the two sides are equal when $x=\pm 1$ as well, but that requires a much more delicate argument.
}
\fi
