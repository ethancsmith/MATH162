%!TEX root =  main.tex

\lectureheader{162}{Calculus II}{Trigonometric substitution}{\textit{Thomas' Calculus}  8.4}

\begin{remark}
Integrals often arise with integrands containing expressions of the form $\sqrt{a^2-x^2}$.
It is convenient to make the substitution $x=a\sin\theta$, really $\theta = \arcsin(x/a)$.  
Why?
\end{remark}

\ifdefined\SOLUTION
\SOLUTION{
Assume $a>0$, and let $x=a\sin\theta$, i.e., let $\theta = \arcsin(x/a)\in [-\pi/2, \pi/2]$.
Then the usual Pythagorean identity gives that
\begin{equation*}
\sqrt{a^2-x^2} = \sqrt{a^2 - a^2\sin^2\theta}
=a\sqrt{1-\sin^2\theta} 
=a\sqrt{\cos^2\theta}
=a|\cos\theta|.
\end{equation*}
However, since $\theta\in [-\pi/2, -\pi/2]$, it follows that $\cos\theta\ge 0$.
Therefore,
\begin{equation*}
\sqrt{a^2-x^2} = a\cos\theta,
\end{equation*}
and so we can trade integrating squareroots for trigonometric integration, which we just practiced.
However, once the integration is all over, we have to convert our result to an expression in terms of $x$, not $\theta$.
Usually the easiest way to do that is to draw/imagine the reference triangle for the relation $x=a\sin\theta$ which has opposite side $x$, hypotenuse $a$, and adjacent side $\sqrt{a^2-x^2}$.
}
\fi

\newpage

\begin{example}
Evaluate $\DS\int\frac{x^2\dee x}{\sqrt{9-x^2}}$.
\end{example}

\ifdefined\SOLUTION
\SOLUTION{
Let $x=3\sin\theta$ so that $\dee x = 3\cos\theta \dee\theta$ and
\begin{equation*}
\sqrt{9-x^2} = \sqrt{9-9\sin^2\theta} = 3\cos\theta.
\end{equation*}
Whence
\begin{equation*}
\int\frac{x^2\dee x}{\sqrt{9-x^2}} = \int\frac{9\sin^2\theta}{3\cos\theta}3\cos\theta\dee\theta
=9\int\sin^2\theta\dee\theta
=\frac{9}{2}\int(1-\cos 2\theta)\dee\theta
=\frac{9}{2}\left(\theta -\frac{1}{2}\sin 2\theta\right)+C.
\end{equation*}
Now we need to express our result in terms of $x$.
We already know that $\sin\theta = x/3$, 
and from that we deduce that $\theta = \arcsin(x/3)$ and $\cos\theta =\frac{1}{3}\sqrt{9-x^2}$.
Thus, we can use the trigonometric identity $\sin 2\theta = 2\sin\theta\cos\theta$ to see that
\begin{equation*}
\sin 2\theta = \frac{2}{9}x\sqrt{9-x^2}.
\end{equation*}
Therefore, 
\begin{equation*}
\int\frac{x^2\dee x}{\sqrt{9-x^2}} 
=\frac{9}{2}\left(\theta -\frac{1}{2}\sin 2\theta\right)+C
=\frac{9}{2}\left(\arcsin(x/3) - \frac{x\sqrt{9-x^2}}{9}\right) +C.
\end{equation*}
}
\fi

\newpage

\begin{remark}
For integrands containing expressions of the form $\sqrt{a^2+x^2}$, it is convenient to make the substitution $x=a\tan\theta$, really $\theta=\arctan(x/a)$.
Why?
\end{remark}

\ifdefined\SOLUTION
\SOLUTION{
Assume $a>0$, and let $x=a\tan\theta$, i.e., let $\theta = \arctan(x/a)\in (-\pi/2, \pi/2)$
Then the usual Pythagorean identity gives that
\begin{equation*}
\sqrt{a^2+x^2} = \sqrt{a^2 + a^2\tan^2\theta}
=a\sqrt{1+\tan^2\theta} 
=a\sqrt{\sec^2\theta}
=a|\sec\theta|.
\end{equation*}
However, since $\theta\in (\pi/2, -\pi/2)$ it follows that $\sec\theta>0$.
Therefore,
\begin{equation*}
\sqrt{a^2+x^2} = a\sec\theta.
\end{equation*}
Again, once the integration is all over, we have to convert our result to an expression in terms of $x$, not $\theta$.
Usually the easiest way to do that is to draw/imagine the reference triangle for the relation $x=a\tan\theta$ which has opposite side $x$, adjacent side $a$, and hypotenuse $\sqrt{a^2+x^2}$.
}
\fi

\newpage

\begin{example}
Compute $\DS\int\frac{\dee x}{\sqrt{4+x^2}}$.
\end{example}

\ifdefined\SOLUTION
\SOLUTION{
Let $x=2\tan\theta$ so that $\dee x = 2\sec^2\theta\dee\theta$ and
\begin{equation*}
\sqrt{4+x^2} = \sqrt{4+4\tan^2\theta} = 2\sec\theta.
\end{equation*}
Whence, 
\begin{equation*}
\int\frac{\dee x}{\sqrt{4+x^2}}
=\int\frac{2\sec^2\theta\dee\theta}{2\sec\theta}
=\int\sec\theta\dee\theta
=\ln|\sec\theta + \tan\theta| + C
=\ln\left|\frac{\sqrt{4+x^2}}{2} + \frac{x}{2}\right| + C.
\end{equation*}
}
\fi

\newpage

\begin{example}
What can a tangent substitution teach you about $\arcsinh x$?
Hint: What is the derivative?
\end{example}

\ifdefined\SOLUTION
\SOLUTION{
Recall that
\begin{equation*}
\frac{\dee}{\dee x}\arcsinh x = \frac{1}{\sqrt{1+x^2}}\quad (-\infty<x<\infty).
\end{equation*}
Now let $x=\tan\theta$ so that $\dee x = \sec^2\theta\dee\theta$ and $\sqrt{1+x^2} = \sqrt{1+\tan^2\theta} = \sec\theta$.
Whence, 
\begin{equation*}
\begin{split}
\arcsinh x 
&= \int\frac{\dee x}{\sqrt{1+x^2}}\\
&=\int\frac{\sec^2\theta\dee\theta}{\sec\theta}\\
&=\int\sec\theta\dee\theta\\
&=\ln|\sec\theta + \tan\theta| + C\\
&=\ln\left|\sqrt{1+x^2} + x\right| + C
\end{split}
\end{equation*}
for some particular constant $C$.
To determine $C$, we substitute $x=0$ which reveals that
\begin{equation*}
0=\arcsinh(0) = \ln|\sqrt{1+0^2} + 0| + C = C.
\end{equation*}
Therefore,
\begin{equation*}
\arcsinh x 
=\ln\left|\sqrt{1+x^2} + x\right|.
\end{equation*}
}
\fi

\newpage

\begin{remark}
For integrands containing expressions of the form $\sqrt{x^2-a^2}$, it is convenient\footnote{Some people prefer $x=\pm a\cosh\theta$.} to make the substitution $x=a\sec\theta$, really $\theta=\arcsec(x/a)$.
Why?
\end{remark}

\ifdefined\SOLUTION
\SOLUTION{
Assume $a>0$, and let $x=a\sec\theta$, i.e., let $\theta = \arcsec(x/a)\in [0, \pi/2)\cup (\pi/2, \pi]$
Then the usual Pythagorean identity gives that
\begin{equation*}
\sqrt{x^2-a^2} = \sqrt{a^2\sec^2\theta - a^2}
=a\sqrt{\sec^2\theta-1}
=a|\tan\theta|.
\end{equation*}
This time we aren't so fortunate when we try to remove the absolute value.
Rather we have that
\begin{equation*}
\sqrt{x^2-a^2} =
\begin{cases}
a\tan\theta & \text{if }\theta\in [0,\pi/2), \text{ i.e., } x\ge a\\
-a\tan\theta & \text{if }\theta\in (\pi/2, \pi], \text{ i.e. }, x\le a.
\end{cases}
\end{equation*}
Again, once the integration is all over, we have to convert our result to an expression in terms of $x$, not $\theta$.
Usually the easiest way to do that is to draw/imagine the reference triangle for the relation $x=a\sec\theta$ which has hypotenuse $|x|$, adjacent side $a$, and opposite side $\sqrt{x^2-a^2}$.
}
\fi

\newpage

\begin{example}
Calculate $\DS\int\frac{\dee x}{\sqrt{25x^2-4}}$.
\end{example}

\ifdefined\SOLUTION
\SOLUTION{
This time we take $5x = 2\sec\theta$ so that $5\dee x = 2\sec\theta\tan\theta\dee\theta$ and
\begin{equation*}
\sqrt{25x^2-4} = \sqrt{4\sec^2\theta - 4} = 2\sqrt{\sec^2\theta-1} = 2\sqrt{\tan^2\theta} = 2|\tan\theta|.
\end{equation*}
Whence,
\begin{equation*}
\begin{split}
\int\frac{\dee x}{\sqrt{25x^2-4}} 
&= \int\frac{\frac{2}{5}\sec\theta\tan\theta\dee\theta}{2|\tan\theta|}\\
&=\begin{cases}
\frac{1}{5}\int\sec\theta\dee\theta & \text{if } \theta\in (0,\pi/2),\\
-\frac{1}{5}\int\sec\theta\dee\theta & \text{if }\theta\in (\pi/2, \pi)
\end{cases}\\
&=\begin{cases}
\frac{1}{5}\ln|\sec\theta + \tan\theta| + C & \text{if } \theta\in (0,\pi/2),\\
-\frac{1}{5}\ln|\sec\theta + \tan\theta| + C & \text{if }\theta\in (\pi/2, \pi).
\end{cases}\\
&=\begin{cases}
\frac{1}{5}\ln\left|\frac{5x}{2} + \frac{\sqrt{25x^2-4}}{2}\right| + C & \text{if } x>2/5,\\
-\frac{1}{5}\ln\left|\frac{5x}{2} - \frac{\sqrt{25x^2-4}}{2}\right| + C & \text{if } x<-2/5.\\
\end{cases}
\end{split}
\end{equation*}
Perhaps a bit surprisingly, it turns out that expressions in the two different cases are algebraically equivalent so that
\begin{equation*}
\int\frac{\dee x}{\sqrt{25x^2-4}} 
=\frac{1}{5}\ln\left|\frac{5x}{2} + \frac{\sqrt{25x^2-4}}{2}\right| + C\quad (|x|>2/5).
\end{equation*}
}
\fi
