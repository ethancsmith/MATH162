%!TEX root =  main.tex

\lectureheader{162}{Calculus II}{Trigonometric substitution}{\textit{Thomas' Calculus} \textsection 8.4}

\begin{remark}
Integrals often arise with integrands containing expressions of the form $\sqrt{a^2-x^2}$.
It is convenient to make the substitution $x=a\sin\theta$, really $\theta = \arcsin(x/a)$.  
Why?
\end{remark}

\newpage

\begin{example}
Evaluate $\DS\int\frac{x^2\dee x}{\sqrt{9-x^2}}$.
\end{example}

\newpage

\begin{remark}
For integrands containing expressions of the form $\sqrt{a^2+x^2}$, it is convenient to make the substitution $x=a\tan\theta$, really $\theta=\arctan(x/a)$.
Why?
\end{remark}

\newpage

\begin{example}
Compute $\DS\int\frac{\dee x}{\sqrt{4+x^2}}$.
\end{example}

\newpage

\begin{example}
What can a tangent substitution teach you about $\arcsinh x$?
Hint: What is the derivative?
\end{example}

\newpage

\begin{remark}
For integrands containing expression of the form $\sqrt{x^2-a^2}$, it convenient\footnote{Some people prefer $x=\pm a\cosh\theta$.} to make the substitution $x=a\sec\theta$, really $\theta=\arcsec(x/a)$.
Why?
\end{remark}

\newpage

\begin{example}
For $x>2/5$, calculate $\DS\int\frac{\dee x}{\sqrt{25x^2-4}}$.
\end{example}
