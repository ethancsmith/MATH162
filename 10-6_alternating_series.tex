%!TEX root =  main.tex

\lectureheader{162}{Calculus II}{Alternating series}{\textit{Thomas' Calculus} \textsection 10.6}

\begin{definition}
We say that $\sum a_n$ is an \textbf{alternating series} if the sequence of terms alternates between positive and negative values, 
i.e., if $a_n = \pm(-1)^n|a_n|$ and $|a_n|>0$ for all $n$. 
\end{definition}

\begin{theorem}[Alternating series test]
If $\sum a_n$ is an alternating series so that
\begin{itemize}
\item $|a_n|$ is nonincreasing for all $n$ sufficiently large, and
\item $|a_n|\to 0$ as $n\to\infty$,
\end{itemize}
then $\sum a_n$ converges.
\end{theorem}

\begin{example}
The series $\DS\sum_{n=1}^\infty\frac{(-1)^{n+1}}{n}$ is called the \textbf{alternating harmonic series}.
Explain why the series converges.
\end{example}

\vfill

\begin{definition}
We say that $\sum a_n$ is \textbf{conditionally convergent} if $\sum a_n$ is convergent but $\sum |a_n|$ is divergent. 
\end{definition}

\newpage

\begin{example}
Show that the series $\DS\sum_{n=0}^\infty\frac{(-1)^n 10n}{n^2+16}$ converges.
\end{example}

\newpage

\begin{remark}
Recall that if $\sum a_n$ is a convergent series, then the $N$th partial sum
\begin{equation*}
S_N = \sum_{n\le N} a_n
\end{equation*}
approximates the sum of the series, and the $N$th remainder
\begin{equation*}
 R_N = \sum_{n=N+1}^\infty a_n
\end{equation*}
is the error in the approximation.
\end{remark}

\begin{theorem}[Alternating series estimation theorem]
If $\DS\sum_{n=0}^\infty a_n$ is an alternating series so that
\begin{itemize}
\item $|a_n|$ is nonincreasing for all $n\ge 0$, and
\item $|a_n|\to 0$ as $n\to\infty$,
\end{itemize}
then
\begin{enumerate}
\item $S_N$ underestimates the sum of the series if $a_{N+1}>0$, and
\item $S_N$ overestimates the sum of the series if $a_{N+1}<0$.
\end{enumerate}
Moreover, 
\begin{equation*}
\Big| R_N \Big| = \left| S_N - \sum_{n=0}^\infty a_n\right| < \Big| a_{N+1}\Big|.
\end{equation*}
\end{theorem}

\begin{example}
We know that $\DS\sum_{n=1}^\infty\frac{(-1)^{n+1}}{n^3}$ converges to something.
Thanks to Roger Ap\'ery we know that it is an irrational number, but that is about all we know about this number.
How many terms do we need to include so that we are sure that our approximation to the sum is accurate to within $10^{-3}$?
\end{example}

\newpage

\begin{remark}\,
\begin{itemize}
\item When performing a large-scale calculation, it is often possible to speed-up the computation by ``reordering the work."
\item The following theorem tells us that it is okay to reorder the work of summing an absolutely convergent series.
\end{itemize}
\end{remark}

\begin{theorem}[Series rearrangement theorem]
If $\sum a_n$ converges absolutely and the sequence $\{b_n\}$ is any (re)arrangement of the sequence of terms $\{a_n\}$, then $\sum b_n$ converges absolutely, and $\sum b_n = \sum a_n$.
\end{theorem}

\begin{example}
Recall that Euler showed that
\begin{equation*}
\sum_{n=1}^\infty\frac{1}{n^2} = \frac{\pi^2}{6}.
\end{equation*}
Use this fact together with the rearrangement theorem to compute the value of $\DS\sum_{n=1}^\infty\frac{(-1)^{n+1}}{n^2}$.
\end{example}

\newpage

\begin{example}
Which of $\DS\sum_{n=1}^\infty\frac{1}{n^2}$ and  $\DS\sum_{n=1}^\infty\frac{(-1)^{n+1}}{n^2}$ converges faster?
In other words, if you wanted to approximate 
\begin{equation*}
\pi^2 = 6 \sum_{n=1}^\infty\frac{1}{n^2} = 12\sum_{n=1}^\infty\frac{(-1)^{n+1}}{n^2},
\end{equation*}
which sum requires the fewest additions to achieve the same accuracy?
\end{example}


\newpage

\begin{remark}\,
\begin{itemize}
\item The rearrangement theorem is not true if $\sum a_n$ is conditionally convergent
\item With enough skill, it possible to rearrange the terms of a conditionally convergent series so that the resulting series diverges or converges to whatever we want.
\item That means that we need to be very careful when manipulating conditionally convergent series so that we do not deceive ourselves.
\end{itemize}
\end{remark}

\begin{example}
Later we will show that
\begin{equation*}
\sum_{n=1}^\infty\frac{(-1)^{n+1}}{n} = \ln 2.
\end{equation*}
Show how a misuse of the rearrangement theorem can lead to an \underline{invalid proof} that $2\ln 2 = \ln 2$.
\end{example}
