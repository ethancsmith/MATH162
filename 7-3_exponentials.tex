%!TEX root =  main.tex
\setcounter{chapter}{7}
\setcounter{section}{3}
\setcounter{theorem}{0}
\setcounter{equation}{0}

\lectureheader{162}{Calculus II}{The natural exponential}{\textit{Thomas' Calculus}  \thesection}

\begin{theorem}
The natural logarithm is invertible.
\end{theorem}
\begin{definition}
The \textbf{natural exponential function} $\exp x $ is the inverse of the natural logarithm $\ln x$, i.e.,
\begin{align*}
\exp(\ln x) &= x \quad (x>0),\\
\ln (\exp x) &= x \quad (x\in\R).
\end{align*}
\end{definition}


\begin{remark}
Have you ever thought about what exponentiation really means?
\begin{itemize}
\item What does $2^3$ mean?

\ifdefined\SOLUTION
\SOLUTION {
    $2^3 = 2\cdot2\cdot2$ (Repeated multiplication.)
}
\else
\vspace{1in}
\fi

\item What about $(5/4)^{2/3}$?
\ifdefined\SOLUTION
\SOLUTION {
    \begin{equation*}
        \left(\frac{5}{4}\right)^{2/3}
        = \left(\frac{5}{4}\cdot\frac{5}{4}\right)^{1/3}
        =  \left(\frac{25}{16}\right)^{1/3},
    \end{equation*}
    which is the unique real solution to $x^3 = \frac{25}{16}$.
}
\else
\vspace{1in}
\fi

\item What about $\E^\pi$? What about ${\sqrt 2}^{\sqrt 2}$?
\ifdefined\SOLUTION
\SOLUTION {
   Hmmm\dots Let's put in a pin that for a moment. 
}
\else
\vspace{1in}
\fi
\end{itemize}
\end{remark}

\newpage

\begin{definition}
If $x$ and $y$ are real numbers with $x>0$, then we make the definition
\begin{equation*}
x^y = \exp(y\ln x).
\end{equation*}
In particular, $\E^y := \exp y$.
\end{definition}

\begin{remark}
This definition can sometimes be incredibly useful when evaluating limits, derivatives, and comparing the relative growth of functions.
It also implies the following ``precalculus facts."
\end{remark}

\begin{theorem}
If $a, b, c$ are real numbers with $a>0$, then
\begin{enumerate}
\item $\ln(a^b) = b\ln a$,
\item $a^{b+c} = a^ba^c$,
\item $a^{b-c} = \frac{a^b}{a^c}$,
\item $a^{bc} = (a^b)^c$.
\end{enumerate}
\end{theorem}

\begin{example}
So what does $\E^\pi$ really mean?
\end{example}

\ifdefined\SOLUTION
\SOLUTION[Solution] {
    $x = \E^\pi = \exp(\pi) = \ln{^{-1}(\pi)}.$
    It follows that $\ln{x} = \pi$.  So, $\E^\pi$ is the unique solution to the equation
    $\int_{1}^{x} \frac{\dee t}{t} \, = \pi$.
}
\else
\fi
\vfill

\begin{example}
Solve $\E^{2x-6}=4$ for $x$.
\end{example}

\ifdefined\SOLUTION
\SOLUTION[Solution] {
    $\E^{2x-6}=4.$  Take the natural logarithm of both sides.  So, 
    \begin{equation*}
        \ln{e^{2x-6}} = \ln{4}
    \end{equation*} 
    \begin{equation*}
        2x-6 = \ln{4}
    \end{equation*} 
    \begin{equation*}
        2x = \ln{4} + 6
    \end{equation*} 
    \begin{equation*}
        x = \frac{\ln{4} + 6}{2}
    \end{equation*} 
}
\else
\fi
\vfill

\begin{theorem}
$\DS\lim_{x\to 0}\left(1+x\right)^{1/x} = \E$.
\end{theorem}
\begin{remark}
We will develop a technique to evaluate limits like this in section 7.5. 
\end{remark}

\newpage

\begin{theorem}
\begin{equation*}
\frac{\dee }{\dee x}\exp x = \exp x.
\end{equation*}
\end{theorem}
\ifdefined\SOLUTION
\SOLUTION[Solution] {
By definition of inverses, 
\begin{equation*}
    \ln{(\exp x)} = x
\end{equation*}
for all real $x$.  By the chain rule, 
\begin{equation*}
    \frac{1}{\exp{x}}\cdot\frac{\dee}{\dee x}(\exp x) = 1.
\end{equation*}
Clearing the denominator, we get 
\begin{equation*}
    \frac{\dee}{\dee x}(\exp x) = \exp x.
\end{equation*}
}
\else
\begin{proof}\,

\vspace{3in}
\end{proof}
\fi
%\vfill

\begin{corollary}
The natural exponential function is continuous, differentiable, strictly increasing, and concave up on its domain $(-\infty, \infty)$.
\end{corollary}

\newpage

\begin{example}
Compute $\DS\frac{\dee}{\dee x}\E^{\sqrt{3x+1}}$.
\end{example}
\ifdefined\SOLUTION
\SOLUTION[Solution] {
\begin{equation*}
\frac{\dee}{\dee x}e^{\sqrt{3x+1}}
= e^{\sqrt{3x+1}}\cdot \frac{\dee}{\dee x}\sqrt{3x+1}
=  e^{\sqrt{3x+1}}\cdot \frac{1}{2}(3x+1)^{-\frac{1}{2}}\cdot 3
= \frac{3e^{\sqrt{3x+1}}}{2\sqrt{3x+1}}.
\end{equation*}
%Or,
%\begin{equation*}
%\frac{\dee}{\dee x} e^{\sqrt{3x+1}}
%= \frac{\dee}{\dee x} \exp (\sqrt{3x+1})\cdot \frac{\dee}{\dee x} \sqrt{3x+1}
%= \exp (\sqrt{3x+1}) \cdot \frac{1}{2}(3x+1)^{-\frac{1}{2}}\cdot 3
%\end{equation*}
}
\else
\fi
\vfill

\begin{theorem}
\begin{equation*}
\int\E^x\dee x = \E^x + C.
\end{equation*}
\end{theorem}

\begin{example}
Evaluate $\DS\int_0^{\pi/2}\E^{\sin x}\cos x\dee x$.
\end{example}
\ifdefined\SOLUTION
\SOLUTION[Solution] {
Let $u = \sin{x}.$  It follows that $\dee u = \cos{x}\dee x.$  For the limits of integration, $u(0) = \sin{0} = 0,$ and $u(\frac{\pi}{2} = \sin{\frac{\pi}{2}} = 1$.  So, 
\begin{equation*}
    \int_{0}^{\frac{\pi}{2}} \E^{\sin{x}}\cos{x}\dee x
    = \int_{0}^{1}\E^u \, \dee u
    = \E^u \Biggr|_{0}^{1}
    = \E^1 - \E^0 = \E - 1.
\end{equation*}
Or alternatively, 
\begin{equation*}
\int_{0}^{\frac{\pi}{2}} \E^{\sin{x}}\cos{x} \dee x
= \int_{x=0}^{x = \frac{\pi}{2}} \E^u \dee u
= \E^u \Biggr|_{x = 0}^{x = \frac{\pi}{2}}
= \E^{\sin{x}} \Biggr|_{x = 0}^{x = \frac{\pi}{2}} 
= \E^{\sin{\frac{\pi}{2}}} - \E^{\sin{0}} = \E - 1.
\end{equation*}
}
\else
\fi
\vfill



\newpage

\begin{definition}
For any fixed $a>0$, the \textbf{exponential with base} $a$ is $a^x = \exp(x\ln a)$.
\end{definition}

\begin{theorem}
For a fixed real number $a>0$, we have
\begin{align*}
\frac{\dee }{\dee x}a^x &= (\ln a) a^x,\\
\int a^x\dee x &= \frac{a^x}{\ln a} + C.
\end{align*}
\end{theorem}

\begin{example}
Calculate $\DS\frac{\dee }{\dee x}3^{\sin x}$.
\end{example}

\ifdefined\SOLUTION
\SOLUTION[Solution] {
\begin{equation*}
    \frac{\dee}{\dee x}3^{\sin{x}} = (\ln{3})3^{\sin{x}}\cdot\frac{\dee}{\dee x}\sin{x}
    = (\ln{3})3^{\sin{x}}\cos{x}.
\end{equation*}
}
\else
\fi

\vfill

\begin{example}
Compute $\DS\int 2^{\sin x}\cos x\dee x$.
\end{example}

\ifdefined\SOLUTION
\SOLUTION[Solution]{
Let $u = \sin{x}.$  It follows that $\dee u = \cos{x} \dee x$.  So, 
\begin{equation*}
    \int_{}^{} 2^{\sin{x}}\cos{x}\dee x 
    = \int_{}^{} 2^u \dee u
    = \frac{2^u}{\ln{2}} + C
    = \frac{2^{\sin{x}}}{\ln{2}} + C.
\end{equation*}
}
\else
\fi
\vfill

\newpage

\begin{definition}
For any fixed $a\in (0,1)\cup (1,\infty)$, the \textbf{logarithm with base} $a$ is the inverse of $a^x$, i.e.,
\begin{align*}
a^{\log_a x } &= x \quad (x>0),\\
\log_a(a^x) &= x\quad (x\in\R).
\end{align*}
\end{definition}

\begin{theorem}
For any $a,b>0$ with $b\ne 1$, $\log_ba = \DS\frac{\ln a}{\ln b}$.
\end{theorem}

\begin{corollary}
For any fixed $a\in (0,1)\cup (1,\infty)$, we have
\begin{equation*}
\frac{\dee }{\dee x}\log_a x = \frac{1}{x\ln a}.
\end{equation*}
\end{corollary}

\ifdefined\SOLUTION
\SOLUTION[Examples] {
\begin{equation*}
    \frac{\dee}{\dee x}\log_2{x}
    = \frac{\dee}{\dee x} \frac{\ln{x}}{\ln{2}}
    = \frac{1}{\ln{2}}\cdot \frac{\dee}{\dee x} \ln{x} 
    = \frac{1}{\ln{2}} \cdot \frac{1}{x}.
\end{equation*}
\vspace{.5in}
\begin{equation*}
    \frac{\dee}{\dee x}(47)^x 
    =\frac{\dee}{\dee x} \exp (x\ln{47}) 
    =\exp (x\ln{47}) \cdot \frac{\dee}{\dee x} x\ln{47}
    =\exp (x\ln{47}) \cdot\ln{47}
    = (47)^x\ln{47}.
\end{equation*}
}
\else
\fi
