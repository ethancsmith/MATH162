%!TEX root =  main.tex

\lectureheader{162}{Calculus II}{The natural exponential}{\textit{Thomas' Calculus} \textsection 7.3}

\begin{theorem}
The natural logarithm is invertible.
\end{theorem}
\begin{definition}
The \textbf{natural exponential function} $\exp x $ is the inverse of the natural logarithm $\ln x$, i.e.,
\begin{align*}
\exp(\ln x) &= x \quad (x>0),\\
\ln (\exp x) &= x \quad (x\in\R).
\end{align*}
\end{definition}


\begin{remark}
Have you ever thought about what exponentiation really means?
\begin{itemize}
\item What does $2^3$ mean?
\vspace{2in}
\item What about $(5/4)^{2/3}$?
\vspace{2.5in}
\item What about $\E^\pi$? What about ${\sqrt 2}^{\sqrt 2}$?
\vspace{0.1in}
\end{itemize}
\end{remark}

\newpage

\begin{definition}
If $x$ and $y$ are real numbers with $x>0$, then we make the definition
\begin{equation*}
x^y = \exp(y\ln x).
\end{equation*}
In particular, $\E^y := \exp y$.
\end{definition}

\begin{remark}
This definition can sometimes be incredibly useful when evaluating limits, derivatives, and comparing the relative growth of functions.
It also implies the following ``precalculus facts."
\end{remark}

\begin{theorem}
If $a, b, c$ are real numbers with $a>0$, then
\begin{enumerate}
\item $\ln(a^b) = b\ln a$,
\item $a^{b+c} = a^ba^c$,
\item $a^{b-c} = \frac{a^b}{a^c}$,
\item $a^{bc} = (a^b)^c$.
\end{enumerate}
\end{theorem}

\begin{example}
So what does $\E^\pi$ really mean?
\end{example}

\vfill


\begin{example}
Solve $\E^{2x-6}=4$ for $x$.
\end{example}

\vfill

\begin{theorem}
$\DS\lim_{x\to 0}\left(1+x\right)^{1/x} = \E$.
\end{theorem}
\begin{remark}
We will develop a technique to evaluate limits like this in \textsection 7.5. 
\end{remark}

\newpage

\begin{theorem}
\begin{equation*}
\frac{\dee }{\dee x}\exp x = \exp x.
\end{equation*}
\end{theorem}
\begin{proof}\,

\vspace{3in}
\end{proof}

\begin{corollary}
The natural exponential function is continuous, differentiable, strictly increasing, and concave up on its domain $(-\infty, \infty)$.
\end{corollary}

\newpage

\begin{example}
Compute $\DS\frac{\dee}{\dee x}\E^{\sqrt{3x+1}}$.
\end{example}
\vfill

\begin{theorem}
\begin{equation*}
\int\E^x\dee x = \E^x + C.
\end{equation*}
\end{theorem}

\begin{example}
Evaluate $\DS\int_0^{\pi/2}\E^{\sin x}\cos x\dee x$.
\end{example}
\vfill



\newpage

\begin{definition}
For any fixed $a>0$, the \textbf{exponential with base} $a$ is $a^x = \exp(x\ln a)$.
\end{definition}

\begin{theorem}
For a fixed real number $a>0$, we have
\begin{align*}
\frac{\dee }{\dee x}a^x &= (\ln a) a^x,\\
\int a^x\dee x &= \frac{a^x}{\ln a} + C.
\end{align*}
\end{theorem}

\begin{example}
Calculate $\DS\frac{\dee }{\dee x}3^{\sin x}$.
\end{example}

\vfill

\begin{example}
Compute $\DS\int 2^{\sin x}\cos x\dee x$.
\end{example}
\vfill

\newpage

\begin{definition}
For any fixed $a\in (0,1)\cup (1,\infty)$, the \textbf{logarithm with base} $a$ is the inverse of $a^x$, i.e.,
\begin{align*}
a^{\log_a x } &= x \quad (x>0),\\
\log_a(a^x) &= x\quad (x\in\R).
\end{align*}
\end{definition}

\begin{theorem}
For any $a,b>0$ with $b\ne 1$, $\log_ba = \DS\frac{\ln a}{\ln b}$.
\end{theorem}

\begin{corollary}
For any fixed $a\in (0,1)\cup (1,\infty)$, we have
\begin{equation*}
\frac{\dee }{\dee x}\log_a x = \frac{1}{x\ln a}.
\end{equation*}
\end{corollary}

