%!TEX root =  main.tex

\lectureheader{162}{Calculus II}{Infinite sequences}{\textit{Thomas' Calculus} \textsection 10.1}

\begin{definition}\,
\begin{itemize}
\item A \textbf{sequence} is a function whose domain is a subset of $\Z$ (the set of whole numbers).
\item The values of the function are called \textbf{terms} of the sequence.
\item An \textbf{infinite sequence} is a sequence whose domain is infinite.
\end{itemize}
\end{definition}

\begin{remark}
In practice, we view a sequence as an ordered list of real numbers
\begin{equation*}
a_1, a_2, a_3,\dots.
\end{equation*}
In calculus, we are mostly interested in sequences that can be described by simple rules/formulas or by recursion.
\end{remark}

\begin{definition}
Let $\{a_n\}$ be an infinite sequence, and let $L$ be a real number.
\begin{itemize}
\item We say that $\{a_n\}$ \textbf{converges} to $L$, and we write $a_n\to L \text{ as } n\to\infty$ or
\begin{equation*}
\lim_{n\to\infty}a_n = L,
\end{equation*}
if for every $\epsilon>0$, there is an integer $N$ so that
\begin{equation*}
n>N \implies |a_n - L|<\epsilon.
\end{equation*}
\item If $\{a_n\}$ does not converge to any real number $L$, then we say that $\{a_n\}$ \textbf{diverges}.
\end{itemize}
\end{definition}

\begin{remark}\,
\begin{itemize}
\item For many applications, the notation $\DS\lim_{n\to\infty} a_n= L$ makes the interesting part of the limit implicit.
\item Often (though not always) it is the interplay between the $\epsilon$ and the $N$ in the definition that is truly interesting.
\item Imagine a sequence as list of approximate answers to an interesting question (e.g., what really is $\ln(2)$?).
\item We should think of the $\epsilon$ as an acceptable ``error tolerance" that is subject to change depending on the day-to-day whims of our customer (or zany math prof).
\item The only thing the customer is not allowed to ask for is zero error.  Any $\epsilon>0$ is allowed.
\item We should think of the $N$ as the ``cutoff" past which all approximations are guaranteed to be within requirements (i.e., have acceptable errors).
\item To say $\DS\lim_{n\to\infty} a_n= L$, is to say that no matter how unreasonably small we are asked to get the errors, we can always ``win the game" of finding a finite cutoff $N$ so that if we ``turn $n$ up past $N$" the error is guaranteed to be within tolerance.
\end{itemize}
\end{remark}

\newpage

\begin{example}
Let $T_n$ denote the $n$-step trapezoidal approximation to $\ln 2 = \int_1^2\frac{\dee t}{t}$ for $n\ge 1$.
Then
\begin{equation*}
T_1 = \frac{3}{2},\quad
T_2 = \frac{13}{6},\quad
T_3 =\frac{57}{20},\quad
T_4=\frac{743}{210},\quad
T_5 = \frac{2131}{504},\quad \dots.
\end{equation*}
The trapezoidal approximation theorem tells us that
\begin{equation*}
\left| T_n - \ln 2\right| \le \frac{1}{6n^2}.
\end{equation*}
Use this fact to show that $T_n\to \ln 2$ as $n\to\infty$.
\end{example}

\newpage

\begin{definition}
Let $\{a_n\}$ be an infinite sequence, and let $L$ be a real number.
\begin{itemize}
\item We say that $\{a_n\}$ \textbf{diverges} to $\infty$, and we write $a_n\to\infty$ as $n\to\infty$ or
\begin{equation*}
\lim_{n\to\infty}a_n = \infty
\end{equation*}
if for every $M>0$, there is an integer $N$ so that
\begin{equation*}
n>N\implies a_n>M.
\end{equation*}
\item We say that $\{a_n\}$ \textbf{diverges} to $-\infty$, and we write $a_n\to -\infty$ as $n\to\infty$ or
\begin{equation*}
\lim_{n\to\infty}a_n = -\infty
\end{equation*}
if for every $M<0$, there is an integer $N$ so that
\begin{equation*}
n>N\implies a_n<M.
\end{equation*}
\end{itemize}
\end{definition}

\begin{example}
If $n$ is a nonnegative integer, then $n$-\textbf{factorial} is 
\begin{equation*}
n! = \prod_{j=0}^{n-1}\left(n-j\right).
\end{equation*}
List the first 5 terms, and then determine whether the sequence converges or diverges.
\end{example}

\newpage

\begin{theorem}
For all integers $n\ge 1$, 
\begin{equation*}
\E\left(\frac{n}{\E}\right)^n\le n!\le n^{n-1}.
\end{equation*}
\end{theorem}
\begin{remark}
Stirling is famous for proving a much more precise relationship, viz., for every $\epsilon>0$, there is an $N>0$ so that
\begin{equation*}
(1-\epsilon) \sqrt{2\pi n}\left(\frac{n}{\E}\right)^n\le n!\le (1+\epsilon) \sqrt{2\pi n}\left(\frac{n}{\E}\right)^n
\end{equation*}
for all $n>N$.
\end{remark}
\begin{proof}\,

\vspace{6in}
\end{proof}



\newpage

\begin{remark}\,
\begin{itemize}
\item Sequences are discrete (not continuous) and so they can't have derivatives.
\item That's bad news for L'H\^opital's rule, but the following theorem tells us that if our sequence is the restriction of a function to integer inputs, then pretty much everything that we're used doing with limits at $\infty$ for functions still works including L'H\^opital's rule (when it applies).
\end{itemize}
\end{remark}

\begin{theorem}
Let $\{a_n\}$ be an infinite sequence, and let $f(x)$ be a function that is defined for all sufficiently large $x$.
If $f(n) = a_n$ for all sufficiently large integers $n$, then
\begin{equation*}
\lim_{x\to\infty}f(x) = L \implies \lim_{n\to\infty} a_n=  L.
\end{equation*}
\end{theorem}

\begin{example}
Compute the following.
\begin{enumerate}
\item $\DS\lim_{n\to\infty}\frac{n}{n+1}$
\vfill
\item $\DS\lim_{n\to\infty}\frac{\ln n}{n}$
\vfill
\item $\DS\lim_{n\to\infty}\left(\frac{n+1}{n-1}\right)^n$
\vfill
\end{enumerate}
\end{example}


\newpage

\begin{theorem}
If $\{a_n\}$ is a sequence of real numbers and $\DS\lim_{n\to\infty}|a_n|=0$, then $\DS\lim_{n\to\infty}a_n=0$.
\end{theorem}

\begin{example}
Show that the sequence $\{a_n\}$ whose terms are defined by 
\begin{equation*}
a_n = \frac{(-1)^n}{n}\quad (n\ge 1)
\end{equation*}
converges.
\end{example}

\vfill


\begin{example}
Does the sequence $\{a_n\}$ whose terms are defined by
\begin{equation*}
a_n=(-1)^n\quad (n\ge 0)
\end{equation*}
converge or diverge?
\end{example}

\vfill

\newpage

\begin{theorem}[Squeeze/sandwich theorem]
Let $\{a_n\}$, $\{b_n\}$, and $\{c_n\}$ be sequences of real numbers.
If $a_n\le b_n\le c_n$ for all $n$ sufficient large, then
\begin{equation*}
\lim_{n\to\infty}a_n = L = \lim_{n\to\infty}c_n \implies \lim_{n\to\infty}b_n= L.
\end{equation*}
\end{theorem}

\begin{example}
Determine $\DS\lim_{n\to\infty}\frac{\cos n}{n}$.
\end{example}
\vfill

\begin{example}
Determine if the sequence $\DS a_n=\frac{n!}{n^n}$ converges.
\end{example}
\vfill

\newpage

\begin{example}
For what values of $x$ is the sequence $a_n=x^n$ convergent?
\end{example}
\vfill

\newpage

\begin{theorem}
For every (fixed) $x\in\R$,
\begin{equation*}
\lim_{n\to\infty}\frac{x^n}{n!} = 0.
\end{equation*}
\end{theorem}
\begin{proof}\,

\vspace{6in}
\end{proof}

\begin{remark}
It is sometimes helpful to think about the above theorem in the following equivalent forms.
\begin{enumerate}
\item If $x\in\R$, then $x^n\lll n!$ as $n\to\infty$.
\item If $x\in\R$, then for every $\epsilon>0$, there is an $N>0$ so that
\begin{equation*}
n>N\implies |x|^n \le\epsilon n!.
\end{equation*}
\end{enumerate}
\end{remark}

\newpage

\begin{definition}
Let $\{a_n\}$ be a sequence of real numbers.
\begin{itemize}
\item We say that $\{a_n\}$ is \textbf{bounded above} if there is a real number $M$ so that $a_n\le M$ for all $n$, in which case we say that $M$ is an \textbf{upper bound} for $\{a_n\}$.
\item If $L$ is an upper bound for $\{a_n\}$ and no smaller number is also an upper bound for $\{a_n\}$, then we say that $L$ is the \textbf{least upper bound} for $\{a_n\}$.
\item We say that $\{a_n\}$ is \textbf{bounded below} if there is a real number $m$ so that $a_n\ge m$ for all $n$, in which case we say that $m$ is an \textbf{lower bound} for $\{a_n\}$.
\item If $G$ is a lower bound for $\{a_n\}$ and no larger number is also a lower bound for $\{a_n\}$, then we say that $G$ is the \textbf{greatest lower bound} for $\{a_n\}$.
\item We say that $\{a_n\}$ is \textbf{bounded} if it is bounded above and below.
If $\{a_n\}$ is not bounded, we say that it is \textbf{unbounded}.
\end{itemize}
\end{definition}

\begin{example}
Determine if the following sequences are bounded.
\begin{enumerate}
\item $a_n = n\quad (n\ge 0)$
\vfill
\item $a_n=(-1)^n n\quad (n\ge 0)$
\vfill
\item $a_n =\DS \frac{n}{n+1}\quad (n\ge 0)$
\vfill
\end{enumerate}
\end{example}


\newpage

\begin{definition}
Let $\{a_n\}$ be a sequence of real numbers.
\begin{itemize}
\item We say that the sequence is \textbf{nondecreasing} if $a_n\le a_{n+1}$ for all $n$.
\item We say that the sequence is \textbf{nonincreasing} if $a_n\ge a_{n+1}$ for all $n$.
\item If the sequence is nondecreasing or nonincreasing, then we say that it is \textbf{monotonic}.
\end{itemize}
\end{definition}

\begin{example}
Determine whether each sequence below is monotonic
\begin{enumerate}
\item $a_n=n\quad (n\ge 0)$
\vfill
\item $\DS a_n=\frac{n}{n+1}\quad (n\ge 0)$
\vfill
\item $a_n=3\quad (n\ge 0)$
\vfill
\item $a_n=(-1)^n\quad (n\ge 0)$
\vfill
\end{enumerate}
\end{example}

\newpage

\begin{theorem}[Monotonic sequence theorem]
Let $\{a_n\}$ be a monotonic sequence.  
Then
\begin{equation*}
\{a_n\} \text{ is bounded} \iff \{a_n\} \text{ is convergent}.
\end{equation*}
\end{theorem}
\begin{proof}[Proof when $\{a_n\}$ is nondecreasing ($\implies$)]\,

\vspace{7in}
\end{proof}
