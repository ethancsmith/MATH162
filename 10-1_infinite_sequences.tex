%!TEX root =  main.tex

\lectureheader{162}{Calculus II}{Infinite sequences}{\textit{Thomas' Calculus}  10.1}

\begin{definition}\,
\begin{itemize}
\item A \textbf{sequence} is a function whose domain is a subset of $\Z$ (the set of whole numbers).
\item The values of the function are called \textbf{terms} of the sequence.
\item An \textbf{infinite sequence} is a sequence whose domain is infinite.
\end{itemize}
\end{definition}

\begin{remark}
In practice, we view a sequence as an ordered list of real numbers
\begin{equation*}
a_1, a_2, a_3,\dots.
\end{equation*}
In calculus, we are mostly interested in sequences that can be described by simple rules/formulas or by recursion.
\end{remark}

\begin{definition}
Let $\{a_n\}$ be an infinite sequence, and let $L$ be a real number.
\begin{itemize}
\item We say that $\{a_n\}$ \textbf{converges} to $L$, and we write $a_n\to L \text{ as } n\to\infty$ or
\begin{equation*}
\lim_{n\to\infty}a_n = L,
\end{equation*}
if for every $\epsilon>0$, there is an integer $N$ so that
\begin{equation*}
n>N \implies |a_n - L|<\epsilon.
\end{equation*}
\item If $\{a_n\}$ does not converge to any real number $L$, then we say that $\{a_n\}$ \textbf{diverges}.
\end{itemize}
\end{definition}

\begin{remark}\,
\begin{itemize}
\item For many applications, the notation $\DS\lim_{n\to\infty} a_n= L$ makes the interesting part of the limit implicit.
\item Often (though not always) it is the interplay between the $\epsilon$ and the $N$ in the definition that is truly interesting.
\item Imagine a sequence as list of approximate answers to an interesting question (e.g., what really is $\ln(2)$?).
\item We should think of the $\epsilon$ as an acceptable ``error tolerance" that is subject to change depending on the day-to-day whims of our customer (or zany math prof).
\item The only thing the customer is not allowed to ask for is zero error.  Any $\epsilon>0$ is allowed.
\item We should think of the $N$ as the ``cutoff" past which all approximations are guaranteed to be within requirements (i.e., have acceptable errors).
\item To say $\DS\lim_{n\to\infty} a_n= L$, is to say that no matter how unreasonably small we are asked to get the errors, we can always ``win the game" of finding a finite cutoff $N$ so that if we ``turn $n$ up past $N$" the error is guaranteed to be within tolerance.
\end{itemize}
\end{remark}

\newpage

\begin{example}
Let $T_n$ denote the $n$-step trapezoidal approximation to $\ln 2 = \int_1^2\frac{\dee t}{t}$ for $n\ge 1$.
Then
\begin{equation*}
T_1 = \frac{3}{4},\quad
T_2 = \frac{17}{24},\quad
T_3 =\frac{7}{10},\quad
T_4=\frac{1171}{1680},\quad
T_5 = \frac{1753}{2520},\quad \dots.
\end{equation*}
The trapezoidal approximation theorem tells us that the absolute error in the approximation $T_n\approx \ln 2$ satisfies
\begin{equation*}
\left| T_n - \ln 2\right| \le \frac{1}{6n^2}.
\end{equation*}
Use this fact to show that $T_n\to \ln 2$ as $n\to\infty$.
\end{example}
\ifdefined\SOLUTION
\SOLUTION[Solution]{
We need to show that the error in the approximation $\ln 2\approx T_n$ can be made as small as we like by taking $n$ to be sufficiently large.
In other words, we need to show that for every $\epsilon>0$, there is an integer $N$ so that if $n>N$, then 
\begin{equation*}
\left| T_n - \ln 2\right| <\epsilon.
\end{equation*}
Let $\epsilon>0$.
Since we know that $|T_n-\ln 2|\le 1/(6n^2)$, we simply force $1/(6n^2)<\epsilon$.
Solving, we see that the latter inequality is satisfied if and only if $n>1/\sqrt{6\epsilon}$.
Since we need an integer, we take $N=\ceil{\frac{1}{\sqrt{6\epsilon}}}$.
Therefore, if $n>N=\ceil{\frac{1}{\sqrt{6\epsilon}}}>\frac{1}{\sqrt{6\epsilon}}$, it follows that
\begin{equation*}
\left| T_n - \ln 2\right|\le\frac{1}{6n^2} <\epsilon.
\end{equation*}
}
\fi

\vfill

\begin{remark}
Given an infinite sequence, we generally have 3 major questions.
\begin{enumerate}
\item Does $a_n$ converge or diverge, i.e., is $\lim_{n\to\infty} a_n$ a number or not a number?
\item If $a_n$ converges, to which value does it converge, i.e., what number is $\DS L=\lim_{n\to\infty} a_n$?
\item If $a_n\to L$ as $n\to\infty$, how large do we need $n$ to be so that the approximation $a_n\approx L$ is guaranteed to be within a given tolerance?
\end{enumerate}
Sometimes it is difficult to answer all 3 questions, and we settle for less.
\end{remark}

\newpage

\begin{remark}\,
\begin{itemize}
\item Sequences are discrete (not continuous) and so they can't have derivatives.
\item That's bad news for L'H\^opital's rule, but the following theorem tells us that if our sequence is the restriction of a function to integer inputs, then pretty much everything that we're used doing with limits at $\infty$ for functions still works including L'H\^opital's rule (when it applies).
\end{itemize}
\end{remark}

\begin{theorem}
Let $\{a_n\}$ be an infinite sequence, and let $f(x)$ be a function that is defined for all sufficiently large $x$.
If $f(n) = a_n$ for all sufficiently large integers $n$, then
\begin{equation*}
\lim_{x\to\infty}f(x) = L \implies \lim_{n\to\infty} a_n=  L.
\end{equation*}
\end{theorem}

\begin{example}
Compute the following.
\begin{enumerate}
\item $\DS\lim_{n\to\infty}\frac{n}{n+1}$
\ifdefined\SOLUTION
\SOLUTION[Solution]{
By L'H\^opital's rule,
\begin{equation*}
\lim_{n\to\infty}\frac{n}{n+1} =\lim_{n\to\infty} \frac{1}{1} = 1.
\end{equation*}
}
\fi
\vfill
\item $\DS\lim_{n\to\infty}\frac{\ln n}{n}$
\ifdefined\SOLUTION
\SOLUTION[Solution]{
By L'H\^opital's rule,
\begin{equation*}
\lim_{n\to\infty}\frac{\ln n}{n} = \lim_{n\to\infty}\frac{1/n}{1} = 0.
\end{equation*}
}
\fi
\vfill

\newpage

\item $\DS\lim_{n\to\infty}\left(\frac{n+1}{n-1}\right)^n$
\ifdefined\SOLUTION
\SOLUTION[Solution]{
By L'H\^opital's rule,
\begin{equation*}
\begin{split}
\lim_{n\to\infty}\left(\frac{n+1}{n-1}\right)^n 
&=\exp\left(\lim_{n\to\infty}n\ln\left(\frac{n+1}{n-1}\right)\right) \\
&=\exp\left(\lim_{n\to\infty}\frac{\ln{\frac{n+1}{n-1}}}{1/n}\right)\\
&=\exp\left(\lim_{n\to\infty}\frac{\frac{n-1}{n+1}\cdot\frac{(n-1)\cdot 1 - (n+1) \cdot 1)}{(n-1)^2}}{-n^{-2}}\right)\\
&=\exp\left(\lim_{n\to\infty}\frac{2n^2}{(n+1)(n-1)}\right)\\
&=\exp\left(\lim_{n\to\infty}\frac{2}{(1+1/n)(1-1/n)}\right)\\
&=\exp\left(\frac{2}{(1+0)(1-0)}\right)\\
&= \E^2. 
\end{split}
\end{equation*}
}
\fi
\vfill
\end{enumerate}
\end{example}

\newpage

\begin{definition}
Let $\{a_n\}$ be an infinite sequence, and let $L$ be a real number.
\begin{itemize}
\item We say that $\{a_n\}$ \textbf{diverges} to $\infty$, and we write $a_n\to\infty$ as $n\to\infty$ or
\begin{equation*}
\lim_{n\to\infty}a_n = \infty
\end{equation*}
if for every $M>0$, there is an integer $N$ so that
\begin{equation*}
n>N\implies a_n>M.
\end{equation*}
\item We say that $\{a_n\}$ \textbf{diverges} to $-\infty$, and we write $a_n\to -\infty$ as $n\to\infty$ or
\begin{equation*}
\lim_{n\to\infty}a_n = -\infty
\end{equation*}
if for every $M<0$, there is an integer $N$ so that
\begin{equation*}
n>N\implies a_n<M.
\end{equation*}
\end{itemize}
\end{definition}

\begin{remark}
The symbols $\pm\infty$ are not numbers; they are ``directions."
To say that $\DS\lim_{n\to\infty} a_n = \pm\infty$ is to say $a_n$ diverges in a particular way.
\end{remark}

\begin{example}
If $n$ is a nonnegative integer, then $n$-\textbf{factorial} is 
\begin{equation*}
n! = \prod_{j=0}^{n-1}\left(n-j\right).
\end{equation*}
List the first 5 terms, and then determine whether the sequence converges or diverges.
\end{example}
\ifdefined\SOLUTION
\SOLUTION[Solution]{
\begin{align*}
0! &= 1. \text{  This is convention! ``Empty products" are 1, like ``empty sums" are 0.}\\
1! &= \prod_{j=0}^{0}\left(1-j\right) = 1, \\
2! &= \prod_{j=0}^{1}\left(2-j\right) = (2)(1) = 2, \\
3! &= \prod_{j=0}^{2}\left(3-j\right) = (3)(2)(1) = 6, \\
4! &= \prod_{j=0}^{3}\left(4-j\right) = (4)(3)(2)(1) = 24.
\end{align*}
It is clear that if $n\ge 0$, then $n!\ge n$.
Since we can make $n$ larger than any given real number, we can make $n!$ larger than any given real number.
Thus, it follows that $\DS\lim_{n\to \infty} n! = \infty$.
}
\else
\fi
\newpage

\begin{theorem}
For all integers $n\ge 0$, 
\begin{equation*}
\left(\frac{n+1}{\E}\right)^n\le n!\le \sqrt{n+1}\left(\frac{n+1}{\E}\right)^n. 
\end{equation*}
\end{theorem}
\ifdefined\SOLUTION
\SOLUTION[Proof of lower bound]{
Let $n\ge 1$.
The first trick is to transform the product $n!$ into a sum via the logarithm.
Observe that
\begin{equation*}
\ln n! = \ln\left(\prod_{k=1}^{n}k \right) = \sum_{k=1}^{n} \ln k.
\end{equation*}
Since $y = \ln(x)$ is always increasing, the right-hand rule tell us that
\begin{equation*}
\int_k^{k+1}\ln(x)\dee x \le \ln(k+1)
\end{equation*}
for every integer $k\ge 1$.
Therefore,
\begin{equation*}
\begin{split}
\int_1^{n+1}\ln(x)\dee x - \sum_{k=1}^n\ln k
&=\sum_{k=1}^n\left[\int_k^{k+1}\ln(x)\dee x\right] - \sum_{k=1}^n\ln k
=\sum_{k=1}^n\left[\int_k^{k+1}\ln(x)\dee x - \ln k\right]\\
&\le\sum_{k=1}^n\left[\ln(k+1) - \ln(k)\right]
=\ln(n+1) - \ln(1)
=\ln(n+1).
\end{split}
\end{equation*}
Thus, it follows that
\begin{equation*}
\ln n! 
=\sum_{k=1}^n\ln(k)
\ge \int_1^{n+1}\ln(x)\dee x - \ln(n+1)
=(x\ln x - x)\Big|_1^{n+1} - \ln(n+1)
=n\ln(n+1) - n.
\end{equation*}
Since $y = \E^x$ is always increasing, 
\begin{equation*}
n!\ge \E^{n\ln(n+1) - n} 
=\left(\E^{\ln(n+1)}\right)^n \cdot \E^{-n}
=\left(\frac{n+1}{\E}\right)^n.
\end{equation*}
}
\else
\begin{proof}[Proof of lower bound]\,

\vspace{6in}
\end{proof}
\fi

\vfill

\begin{remark}\,
\begin{itemize}
\item The above bounds on $n!$ are not optimal, but the upper bound is ``close" to optimal.
\item You are encouraged to try to prove the upper bound.
Hint: The concavity of $\ln x$ implies that the trapezoidal rule gives an underestimate in the step where the right-hand rule gave an overestimate.
\item Stirling is famous for proving that
\begin{equation*}
n!\sim \sqrt{2\pi n}\left(\frac{n}{\E}\right)^n
\end{equation*}
as $n\to\infty$.
In other words, he showed that, for every $\epsilon>0$, there is an $N>0$ so that
\begin{equation*}
(1-\epsilon) \sqrt{2\pi n}\left(\frac{n}{\E}\right)^n\le n!\le (1+\epsilon) \sqrt{2\pi n}\left(\frac{n}{\E}\right)^n
\end{equation*}
for all $n>N$.
\end{itemize}
\end{remark}

\newpage

\begin{theorem}[Squeeze/sandwich theorem]
Let $\{a_n\}$, $\{b_n\}$, and $\{c_n\}$ be sequences of real numbers.
If $a_n\le b_n\le c_n$ for all $n$ sufficient large, then
\begin{equation*}
\lim_{n\to\infty}a_n = L = \lim_{n\to\infty}c_n \implies \lim_{n\to\infty}b_n= L.
\end{equation*}
\end{theorem}

\begin{example}
Determine $\DS\lim_{n\to\infty}\frac{\cos n}{n}$.
\end{example}
\ifdefined\SOLUTION
\SOLUTION[Solution]{
Since $-1\le\cos n\le 1$ for all $n\ge 1$,
it follows that $\frac{-1}{n}\le \frac{\cos{n}}{n}\le\frac{1}{n}$ for all $n\ge 1$. 
Furthermore,
\begin{equation*}
\lim_{n\to\infty}\frac{1}{n} = 0 = \lim_{n\to\infty} \frac{-1}{n}.
\end{equation*}
Therefore, by the squeeze theorem, $\DS\lim_{n\to\infty}\frac{\cos n}{n} = 0$.
}
\else
\fi
\vfill

\begin{example}
Show that $n!\lll n^n$ as $n\to\infty$.
\end{example}
\ifdefined\SOLUTION
\SOLUTION[Solution]{
We need to show that $n!/n^n\to 0$ as $n\to\infty$.
Recall that $0\le n!\le \sqrt{n+1}\left(\frac{n+1}{\E}\right)^n$ for all $n\ge 0$.
Thus, it follows that
\begin{equation*}
0
\le\frac{n!}{n^n}
\le\frac{\sqrt{n+1}}{\E^n}\left(\frac{n+1}{n}\right)^n
\le\frac{\sqrt{n+1}}{\E^n}\left(1+\frac{1}{n}\right)^n
\end{equation*}
for all $n\ge 1$.  
Since $\DS\lim_{n\to\infty} 0 = 0$ and
\begin{equation*}
\lim_{n\to\infty}\cancelto{0}{\frac{\sqrt{n+1}}{\E^n}}\cancelto{\E}{\left(1+\frac{1}{n}\right)^n} = 0,
\end{equation*}
the sandwich theorem implies that $\DS\lim_{n\to\infty}\frac{n!}{n^n}= 0$.
Thus, it follows that $n!\lll n^n$ as $n\to\infty$.
}
\else
\fi
\vfill

\newpage

\begin{theorem}
If $\{a_n\}$ is a sequence of real numbers and $\DS\lim_{n\to\infty}|a_n|=0$, then $\DS\lim_{n\to\infty}a_n=0$.
\end{theorem}

\begin{theorem}
For every (fixed) $x\in\R$, $x^n\lll n!$ as $n\to\infty$, i.e.,
\begin{equation*}
\lim_{n\to\infty}\frac{x^n}{n!} = 0.
\end{equation*}
\end{theorem}
\ifdefined\SOLUTION
\SOLUTION[Proof]{
Let $x\in\R$.
Note that, by the above theorem, it suffices to show that $\DS\lim_{n\to\infty}\left|\frac{x^n}{n!}\right| = 0$.
Recall that 
\begin{equation*}
n!>\left(\frac{n+1}{\E}\right)^n
\end{equation*}
for all $n\ge 0$.
Therefore,
\begin{equation*}
0\le \left|\frac{x^n}{n!}\right|
<\frac{|x|^n}{((n+1)/\E)^n}
=\left(\frac{|x|\E}{n+1}\right)^n
\le\left(\frac{1}{2}\right)^n
\end{equation*}
for all $n\ge 2|x|\E-1$.
Since 
\begin{equation*}
\lim_{n\to\infty} 0 = 0 = \lim_{n\to\infty}(1/2)^n,
\end{equation*}
the squeeze theorem implies that $\DS\lim_{n\to\infty}\left|\frac{x^n}{n!}\right|=0$.
}
\else
\begin{proof}\,

\vspace{6in}
\end{proof}
\fi

%\begin{example}
%Show that the sequence $\{a_n\}$ whose terms are defined by 
%\begin{equation*}
%a_n = \frac{(-1)^n}{n}\quad (n\ge 1)
%\end{equation*}
%converges.
%\end{example}
%\ifdefined\SOLUTION
%\SOLUTION[Solution]{
%Since $|a_n| = \left| \frac{(-1)^n}{n} \right| = \frac{1}{n} \to 0$ as $n \to \infty$, it follows that
%$a_n=\frac{(-1)^n}{n}\to 0$ as $n\to\infty$.
%In particular, the sequence converges.
%}
%\fi
%\vfill
%
%
%\begin{example}
%Does the sequence $\{a_n\}$ whose terms are defined by
%\begin{equation*}
%a_n=(-1)^n\quad (n\ge 0)
%\end{equation*}
%converge or diverge?
%\end{example}
%\ifdefined\SOLUTION
%\SOLUTION[Solution]{
%If $n = 2m$ is an even number, then $a_n = (-1)^n = (-1)^{2m} = 1$. 
%However, if $n = 2m + 1$ is an odd number, then $a_n = (-1)^n = (-1)^{2m + 1} = -1$. 
%So, the sequence will forever oscillate between $+1$ and $-1$.
%Therefore, $a_n = (-1)^n$ diverges.
%}
%\else
%\fi
%\vfill

\vfill

\begin{example}
For which values of $x$ is the sequence $x^n$ convergent?
\end{example}
\ifdefined\SOLUTION
\SOLUTION[Solution]{ 
If $x\le -1$, then $x^n\le -1$ when $n$ is odd and $x^n\ge 1$ when $n$ is even.
Therefore, the sequence $a_n=x^n$ diverges by oscillation when $x\le -1$.
For the remaining cases, 
\begin{equation*}
\lim_{x\to\infty} x^n
=\begin{cases}
\infty & \text{if } x>1,\\
1 	   & \text{if } x=1,\\
0	   & \text{if } -1<x<1.
\end{cases}
\end{equation*}
Therefore, $x^n$ converges if and only if $-1<x\le 1$.
}
\else
\fi
\vfill


\newpage

\begin{definition}
Let $\{a_n\}$ be a sequence of real numbers.
\begin{itemize}
\item We say that $\{a_n\}$ is \textbf{bounded above} if there is a real number $M$ so that $a_n\le M$ for all $n$, in which case we say that $M$ is an \textbf{upper bound} for $\{a_n\}$.
\item If $L$ is an upper bound for $\{a_n\}$ and no smaller number is also an upper bound for $\{a_n\}$, then we say that $L$ is the \textbf{least upper bound} for $\{a_n\}$.
\item We say that $\{a_n\}$ is \textbf{bounded below} if there is a real number $m$ so that $a_n\ge m$ for all $n$, in which case we say that $m$ is an \textbf{lower bound} for $\{a_n\}$.
\item If $G$ is a lower bound for $\{a_n\}$ and no larger number is also a lower bound for $\{a_n\}$, then we say that $G$ is the \textbf{greatest lower bound} for $\{a_n\}$.
\item We say that $\{a_n\}$ is \textbf{bounded} if it is bounded above and below.
If $\{a_n\}$ is not bounded, we say that it is \textbf{unbounded}.
\end{itemize}
\end{definition}

\begin{example}
Determine if the following sequences are bounded.
\begin{enumerate}
\item $a_n = n^2\quad (n\ge 0)$
\ifdefined\SOLUTION
\SOLUTION[Solution]{
Since $a_n = n^2\ge 0$, for all $n\ge 0$, this sequence is bounded below by $m=0$.
In fact, $G=0$ is the greatest lower bound.
However, the sequence is not bounded above since given any real number $r$, we can find an $n$ so that $a_n=n^2>r$.
Therefore, we say that the sequence is unbounded.
}
\else
\fi
\vfill
\item $a_n=(-1)^n n\quad (n\ge 0)$
\ifdefined\SOLUTION
\SOLUTION[Solution]{
We can find an even-indexed term that is larger than any given real number, 
and we can find an odd-indexed term that is smaller than any given real number.
Therefore, sequence is neither bounded above nor below. 
It is also true to say that the sequence is unbounded.
}
\else
\fi
\vfill
\item $a_n =\DS \frac{n}{n+1}\quad (n\ge 0)$
\ifdefined\SOLUTION
\SOLUTION[Solution]{
Observe that if $n\ge 0$, then
\begin{equation*}
0\le n<n+1.
\end{equation*}
Thus, it follows that
\begin{equation*}
0\le \frac{n}{n+1}<1
\end{equation*}
for all $n\ge 0$.
So, the sequence is bounded below by $m=0$ and above by $M=1$.
Therefore, we say that the sequence is bounded.
In fact, $G=0$ is the greatest lower bound and $L=1$ is the least upper bound.
Note, that here the greatest lower bound is attained since $a_0=0=G$.
However, the least upper bound is never actually attained since $a_n = \frac{n}{n+1}<1$ for all $n\ge 0$.
}
\else
\fi
\vfill
\end{enumerate}
\end{example}


\newpage

\begin{definition}
Let $\{a_n\}$ be a sequence of real numbers.
\begin{itemize}
\item We say that the sequence is \textbf{nondecreasing} if $a_n\le a_{n+1}$ for all $n$.
\item We say that the sequence is \textbf{nonincreasing} if $a_n\ge a_{n+1}$ for all $n$.
\item If the sequence is nondecreasing or nonincreasing, then we say that it is \textbf{monotonic}.
\end{itemize}
\end{definition}

\begin{example}
Determine whether each sequence below is monotonic
\begin{enumerate}
\item $a_n=n\quad (n\ge 0)$
\ifdefined\SOLUTION
\SOLUTION[Solution]{
The sequence is monotonic nondecreasing because $a_n = n \leq n+1 = a_{n+1}$ for all $n\ge 0$.
}
\else
\fi
\vfill
\item $\DS a_n=\frac{n}{n+1}\quad (n\ge 0)$
\ifdefined\SOLUTION
\SOLUTION[Solution]{
If we compute the first several terms of the sequence $0, 1/2, 2/3, 3/4,\dots$, it appears to be nondecreasing.
However, it is never sufficient to compute finitely many terms.
We have to show that $a_n\le a_{n+1}$ for all $n\ge 0$.
Unfortunately, increasing $n$ makes both the numerator and the denominator of $a_n$ larger.
So, it's difficult to be sure if $a_{n+1}$ is bigger or smaller than $a_n$.
One strategy is to try to rewrite $a_n$ so that all the change happens in only the numerator or only in the denominator.
For example, we easily see that
\begin{equation*}
a_n = \frac{n}{n+1} = \frac{1}{1+1/n} < \frac{1}{1+1/(n+1)} = a_{n+1}
\end{equation*}
for all $n\ge 1$.
Combining this with the fact that $a_0=0<1/1 = a_1$, we have proved that $a_n$ is nondecreasing.
But here is another strategy that works when $a_n=f(n)$ for some differentiable function $f$.
Let $f(x) = x/(x+1)$ so that $f(n) = a_n$ and
\begin{equation*}
f'(x) = \frac{(x+1)-x}{(x+1)^2} = \frac{1}{(x+1)^2}>0 \quad (x\ge 0).
\end{equation*}
Thus, it follows that $f(x)$ is increasing for all $x\ge 0$, and hence $a_n=f(n)$ is monotonic nondecreasing.
}
\else
\fi
\vfill
\item $a_n=3\quad (n\ge 0)$
\ifdefined\SOLUTION
\SOLUTION[Solution]{
This sequence is both monotonic nondecreasing and monotonic nonincreasing.  
}
\else
\fi
\vfill
\item $a_n=(-1)^n\quad (n\ge 0)$
\ifdefined\SOLUTION
\SOLUTION[Solution]{
Caution: We cannot put $f(x) = (-1)^x$ and try to use the derivative because this function isn't even defined (let alone differentiable) for most real $x$.
Instead, we notice that the sequence oscillates: $1, -1, 1, -1\dots$.
Since $a_1 = -1 < 1 = a_2$, it follows that $a_n$ is not nonincreasing.
(Careful: The double negative can be a bit confusing.)
Since $a_2 = 1 > -1 = a_3$, it follows that $a_n$ is not nondecreasing.
Therefore, $a_n$ is not monotonic. 
}
\else
\fi
\vfill
\end{enumerate}
\end{example}

\newpage

\begin{theorem}[Monotonic sequence theorem]
Let $\{a_n\}$ be a monotonic sequence.  
Then
\begin{equation*}
\{a_n\} \text{ is bounded} \iff \{a_n\} \text{ is convergent}.
\end{equation*}
\end{theorem}
\ifdefined\SOLUTION
\SOLUTION[Proof when $\{a_n\}$ is nondecreasing ($\implies$)]{
Let $\epsilon>0$, and suppose that $a_n$ is bounded and monotone nondecreasing. 
Since $a_n$ is bounded, there exists a least upper bound $L$ so that $a_n\le L$ for all $n$.  
Since $L-\epsilon < L$, it is not an upper bound for $a_n$.
Thus, there is an $N$ such that $a_N > L - \epsilon$.  
Since $a_n$ is nondecreasing, $a_n\ge a_{N}$ for all $n>N$.
Therefore, if $n > N$, then 
\begin{equation*}
L\ge a_n \ge a_N \ge L-\epsilon, 
\end{equation*}
and hence $|a_n-L|<\epsilon$.
Therefore, $a_n \to L$ as $n \to \infty$.
}
\else
\begin{proof}[Proof when $\{a_n\}$ is nondecreasing ($\implies$)]\,

\vspace{5in}
\end{proof}
\fi

\vfill

\begin{remark}
In the proof above, we needed to use our hypotheses to show that there is a number $L$ so that no matter how small our error tolerance $\epsilon$ is, we can get the approximation $a_n\approx L$ within tolerance and keep it in tolerance by pushing $n$ far enough to the right.
Here's how we did that.
\begin{enumerate}
\item Set our (arbitrary) error tolerance $\epsilon$.
\item Assume our hypotheses (the ``if-part").
\item Get our ``target" (least upper bound) $L$ by unpacking what it means to be bounded.
\item Use what it means to be a least upper bound to show that we can get within tolerance.
\item Use what it means to be nondecreasing to explain why we will continue to stay within tolerance from then on.
\end{enumerate}
\end{remark}

