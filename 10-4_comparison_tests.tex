%!TEX root =  main.tex

\lectureheader{162}{Calculus II}{The comparison tests}{\textit{Thomas' Calculus}  10.4}

\begin{theorem}[Direct comparison test]\,
Suppose that $a_n,b_n\ge 0$ for all $n$ sufficiently large.
\begin{enumerate}
\item If $a_n\ll b_n$ as $n\to\infty$ and $\DS\sum_{n\ge k} b_n$ converges, then $\DS\sum_{n\ge k}a_n$ converges.
\item If $b_n\ll a_n$ as $n\to\infty$ and $\DS\sum_{n\ge k} b_n$ diverges, then $\DS\sum_{n\ge k}a_n$ diverges.\label{direct comparison test}
\end{enumerate}
\end{theorem}
%\begin{proof}[Proof of~\eqref{direct comparison test}]\,
%
%\vspace{6in}
%\end{proof}

\begin{remark}
There is an art to applying the direct comparison test, but there is strategy to this art.
\begin{enumerate}
\item Ask yourself what ``simpler" series $\sum_n b_n$ has terms that look similar to those of $\sum_n a_n$.
\item If $\sum_n b_n$ is known to converge, try to show that $0\le a_n\ll b_n$ as $n\to\infty$.  If you succeed, then $\sum_n a_n$ also converges.
\item If $\sum_n b_n$ is known to diverge, try to show that $0\le b_n\ll a_n$ as $n\to\infty$. If you succeed, then $\sum_n a_n$ also diverges.
\end{enumerate}
As with any art, mastery comes with much experience, and experience can only be earned from a lot of practice.
\end{remark}

\newpage

\begin{example}
Determine if $\DS\sum_{n=1}^\infty\frac{5}{10n-1}$ converges or diverges.
\end{example}
\ifdefined\SOLUTION
\SOLUTION{
We guess that $a_n = \frac{5}{10n-1}$ is comparable to $b_n=\frac{1}{n}$.
Since we know that $\DS\sum_{n=1}^\infty\frac{1}{n}$ diverges by the $p$-test with $p=1$, we hope to show that $a_n\gg b_n$ as $n\to\infty$.
Observe then that if $n\ge 1$, then
\begin{equation*}
a_n = \frac{5}{10n-1}\ge \frac{5}{10n} = \frac{1}{2n} = \frac{1}{2}b_n>0,
\end{equation*}
i.e., $b_n\le 2a_n$ for all $n\ge 1$.
Therefore, $b_n = \frac{1}{n}\ll \frac{5}{10n-1} = a_n$ as $n\to\infty$.
Thus, we conclude that $\DS\sum_{n=1}^\infty\frac{5}{10n-1}$ diverges by the direct comparison test (DCT).
}
\else
\fi

\newpage

\begin{example}
Determine whether the series $\DS\sum_{n=0}^\infty\frac{1}{n!}$  converges or diverges.
\end{example}
\ifdefined\SOLUTION
\SOLUTION{
Recall that $n!\ge\left(\frac{n+1}{\E}\right)^n$ for all $n\ge 0$, and therefore
\begin{equation*}
0<\frac{1}{n!}\le \left(\frac{\E}{n+1}\right)^n\le\left(\frac{1}{2}\right)^n \quad (n\ge 2\E-1),
\end{equation*}
i.e., $\frac{1}{n!}\ll \frac{1}{2^n}$ as $n\to\infty$.
Since $\DS\sum_{n=0}^\infty (1/2)^{n}$ converges by GST (with $r = 1/2$), 
it follows that $\DS\sum_{n=0}^\infty \frac{1}{n!}$ converges by DCT. 
}
\else
\fi
\newpage

\begin{example}
Determine whether the series $\DS\sum_{n=0}^\infty\frac{1}{2^n+\sqrt n}$  converges or diverges.
\end{example}
\ifdefined\SOLUTION
\SOLUTION{
Here there are two ``obvious" choices of comparator: $b_n=\frac{1}{2^n}$ or $b_n=\frac{1}{\sqrt n}$.
We want to make the less drastic or less greedy change.
So, we need to ask ourselves which term is more important when it comes to determining the size of the denominator for large $n$.
We can try to drop the other one.\\
\vspace{2\baselineskip}
Observe that for all $n\ge 0$,
\begin{equation*}
0 \le \frac{1}{2^n + \sqrt{n}} \le \frac{1}{2^n}
\end{equation*}
Since $\DS\sum_{n=0}^\infty \frac{1}{2^n}$ converges by GST (with $r = 1/2$), 
it follows that $\DS\sum_{n=0}^\infty \frac{1}{2^n + \sqrt{n}}$ converges by DCT.\\
\vspace{2\baselineskip}
What would have happened if we had guessed that $\frac{1}{\sqrt n}$ was a good comparator?
We would still have
\begin{equation*}
0 \le \frac{1}{2^n + \sqrt{n}} \le \frac{1}{\sqrt{n}}
\end{equation*}
for all $n\ge 1$.
Unfortunately, since $\DS\sum_{n=1}^\infty\frac{1}{\sqrt n}$ diverges by $p$-test (with $p=1/2$),
the DCT tells us nothing about $\DS\sum_{n=0}^\infty\frac{1}{2^n+\sqrt n}$.
}
\fi

\newpage

\begin{remark}\,
\begin{itemize}
\item The test for divergence tells us that if $\sum a_n$ is going to have a chance at convergence, then we must have $a_n\to 0$ as $n\to\infty$.
\item The difference between series convergence and divergence is ``how fast" the terms $a_n$ tend to zero.
\item All the other series tests are trying to get a sense of whether the terms $a_n$ are going to zero ``fast enough" to ensure convergence.
\item The comparison tests are about determining convergence by comparing the relative sizes of the terms.
\item For the direct comparison test we have to be careful to get the inequalities correct for all sufficiently large $n$.
\item The limit comparison test allows for ``cruder" comparison using limits (relative orders at infinity).
\end{itemize}
\end{remark}

\begin{theorem}[Limit comparison test]
Suppose that $a_n, b_n>0$ for all $n\ge N$.
\begin{enumerate}
\item If $a_n\asymp b_n$ as $n\to\infty$, in particular if $\DS\lim_{n\to\infty}\frac{a_n}{b_n}=L\in (0,\infty)$, 
then $\DS\sum_{n\ge k} a_n$ and $\DS\sum_{n\ge k} b_n$ both converge or both diverge.
\item If $a_n\lll b_n$ as $n\to\infty$ (i.e., if $\DS\lim_{n\to\infty}\frac{a_n}{b_n}=0$) and $\DS\sum_{n\ge k} b_n$ converges, then $\DS\sum_{n\ge k} a_n$ converges.
\item If $a_n\ggg b_n$ as $n\to\infty$ (i.e., if $\DS\lim_{n\to\infty}\frac{a_n}{b_n}=\infty$) and $\DS\sum_{n\ge k} b_n$ diverges, then $\DS\sum_{n\ge k} a_n$ diverges.
\end{enumerate}
\end{theorem}

\newpage

\begin{example}
Determine if the series $\DS\sum_{n=1}^\infty\frac{2n+1}{(n+1)^2}$ converges or diverges.
\end{example}
\ifdefined\SOLUTION
\SOLUTION{
First, we guess that
\begin{equation*}
    a_n = \frac{2n+1}{(n+1)^2} \asymp \frac{n}{n^2} = \frac{1}{n} = b_n \text{ as } n \to \infty.
\end{equation*}
Next, we verify this by computing that
\begin{equation*}
\begin{split}
\lim_{n\to\infty} \frac{a_n}{b_n} 
&= \lim_{n\to\infty} \frac{2n+1}{(n+1)^2}\Big/ \left(\frac{1}{n}\right) \\
&= \lim_{n\to\infty} \frac{n(2n+1)}{(n+1)^2} \\
&= \lim_{n\to\infty} \frac{n(2n+1)\frac{1}{n^2}}{(n+1)^2\frac{1}{n^2}} \\
&= \lim_{n\to\infty} \frac{1(2+\frac{1}{n})}{\left( 1 + \frac{1}{n}\right)^2} \\
&= 2 \in (0,\infty).
\end{split}
\end{equation*}
\\Since $\DS\sum_{n=1}^\infty b_n = \sum_{n=1}^\infty \frac{1}{n}$ diverges by $p$-test with $p = 1$,
it follows that $\DS\sum_{n=1}^\infty \frac{2n+1}{(n+1)^2}$ also diverges by the limit comparison test (LTC).
}
\else
\fi
\newpage

\begin{example}
Determine if the series $\DS\sum_{n=1}^\infty\frac{1}{2^n-1}$ converges or diverges.
\end{example}
\ifdefined\SOLUTION
\SOLUTION{
First, we guess that
\begin{equation*}
a_n=\frac{1}{2^n - 1} \asymp \frac{1}{2^n} = b_n \text{ as } n\to\infty.
\end{equation*}
Next we verify that
\begin{equation*}
\lim_{n\to\infty}   \frac{1}{2^n-1}\Big/ \frac{1}{2^n} 
= \lim_{n\to\infty} \frac{2^n}{2^n - 1} 
= \lim_{n\to\infty} \frac{2^n\frac{1}{2^n}}{(2^n - 1)\frac{1}{2^n}} 
= \lim_{n\to\infty} \frac{1}{1-\frac{1}{2^n}} = 1 \in (0,\infty).
\end{equation*}
Since $\DS\sum_{n=1}^\infty b_n = \sum_{n=1}^\infty \frac{1}{2^n}$ converges by GST (with $r=1/2$),
$\DS\sum_{n=1}^\infty \frac{1}{2^n-1}$ converges by LCT.
}
\else
\fi
\newpage

\begin{example}
Determine if the series $\DS\sum_{n=1}^\infty\frac{\ln n}{n^{3/2}}$ converges or diverges.
\end{example}
\ifdefined\SOLUTION
\SOLUTION{
Recall that $\ln{n} \lll n^\epsilon$ as $n\to\infty$ for every $\epsilon > 0$.
So in particular,
\begin{equation*}
a_n = \frac{\ln n}{n^{3/2}} \lll \frac{n^{1/4}}{n^{3/2}} = \frac{1}{n^{5/4}} = b_n \text{ as } n\to\infty.
\end{equation*}
In other words,
\begin{align*}
\lim_{n\to\infty} \frac{a_n}{b_n} = \frac{\ln n}{n^{3/2}}\Big/\frac{1}{n^{5/2}} = 0.
\end{align*}
Since $\DS\sum_{n=1}^\infty b_n = \sum_{n=1}^\infty \frac{1}{n^{5/4}}$ converges by $p$-test (with $p = 5/4$), 
$\DS\sum_{n=1}^\infty \frac{\ln n}{n^{3/2}}$ also converges by LCT.\\
\vspace{2\baselineskip}\\
Note: Because $a_n$ is of smaller order than $b_n$, we needed $\sum b_n$ to converge.  
If it had diverged, the LCT would tell us nothing about the original series.
}
\else
\fi
\newpage

\begin{example}
Determine if the series $\DS\sum_{n=1}^\infty\frac{1+n\ln n}{n^2+5}$ converges or diverges.
\end{example}
\ifdefined\SOLUTION
\SOLUTION{
First, we guess that
\begin{equation*}
a_n = \frac{1+n\ln n}{n^2+5} \asymp \frac{n\ln n}{n^2} = \frac{\ln n}{n} \ggg \frac{1}{n} = b_n \text{ as } n\to\infty.
\end{equation*}
Next we verify that
\begin{equation*}
\begin{split}
\lim_{n\to\infty}\frac{a_n}{b_n}
&=\lim_{n\to\infty}\frac{1+n\ln n}{n^2+5}\cdot\frac{n}{1}\\
&=\lim_{n\to\infty}\frac{n+n^2\ln n}{n^2+5}\\
&=\lim_{n\to\infty}\frac{n^2\ln n\left(\frac{1}{n\ln n} + 1\right)}{n^2\left(1+\frac{5}{n^2}\right)}\\
&=\lim_{n\to\infty}\frac{\ln n\left(\frac{1}{n\ln n} + 1\right)}{\left(1+\frac{5}{n^2}\right)}\\
&=\infty.
\end{split}
\end{equation*}
Therefore, $a_n = \frac{1+n\ln n}{n^2+5}\ggg\frac{1}{n} = b_n$ as $n\to\infty$.
Since $\DS\sum_{n=1}^\infty\frac{1}{n}$ diverges by $p$-test with $p=1$,
$\DS\sum_{n=1}^\infty\frac{1+n\ln n}{n^2+5}$ diverges by LCT.\\
\vspace{2\baselineskip}\\
Note: Because $a_n$ is of greater order than $b_n$, we needed $\sum b_n$ to diverge.  
If it had converged, the LCT would tell us nothing about the original series.
}
\else
\fi

\newpage

\begin{example}
Determine if the series $\DS\sum_{n=1}^\infty\frac{1}{n^{\arctan n}}$ converges or diverges.
\end{example}

\ifdefined\SOLUTION
\SOLUTION{
Since $\arctan n \to \pi/2$ as $n\to\infty$, it is reasonable to guess that
\begin{equation*}
\frac{1}{n^{\arctan n}} \asymp \frac{1}{n^{\pi/2}} \text{ as }n\to\infty.
\end{equation*}
To verify this we compute
\begin{equation*}
\begin{split}
\lim_{n\to\infty}\frac{1}{n^{\arctan n}} \Big/\frac{1}{n^{\pi/2}} 
&=\lim_{n\to\infty}n^{\pi/2-\arctan n}\\
&=\exp\left(\lim_{n\to\infty}\left(\pi/2-\arctan n\right)\ln n\right)\\
&=\exp\left(\lim_{n\to\infty}\frac{\pi/2-\arctan n}{(\ln n)^{-1}}\right)\\
&=\exp\left(\lim_{n\to\infty}\frac{\frac{-1}{n^2+1}}{-(\ln n)^{-2}\frac{1}{n}}\right)\\
&=\exp\left(\lim_{n\to\infty}\frac{n(\ln n)^2}{n^2+1}\right)\\
&=\exp\left(\lim_{n\to\infty}\frac{(\ln n)^2}{n(1+1/n^2)}\right)\\
&=\exp\left(\lim_{n\to\infty}\frac{\left(\frac{\ln n}{\sqrt n}\right)^2}{1+1/n^2}\right)\\
&=\exp(0/1)\\
&=\E\in (0,\infty).
\end{split}
\end{equation*}
Therefore, since $\DS\sum_{n=1}^\infty\frac{1}{n^{\pi/2}}$ converges by $p$-test (with $p=\pi/2>1$), 
it follows that $\DS\sum_{n=1}^\infty\frac{1}{n^{\arctan n}}$ converges by LCT.
}
\fi

\newpage

\begin{example}
Determine if the series $\DS\sum_{n=2}^\infty\frac{1}{(\ln n)^{\ln n}}$ converges or diverges.
\end{example}

\ifdefined\SOLUTION
\SOLUTION{
Here the terms sort of look like a $p$-series, but not quite since the base $\frac{1}{\ln n}$ decays to zero much slower than $\frac{1}{n}$.
However, rather than having a constant $p$ for the exponent we have $\ln n\to\infty$ as $n\to\infty$, which should help speed up the decay.
To compare the two, we apply our usual logarithm trick:
\begin{equation*}
\begin{split}
\lim_{n\to\infty}\frac{1}{(\ln n)^{\ln n}} \Big/\frac{1}{n^p} 
&=\lim_{n\to\infty}\frac{n^p}{(\ln n)^{\ln n}}\\
&=\lim_{n\to\infty}\frac{\exp(p\ln n)}{\exp(-(\ln n)\ln\ln n)}\\
&=\exp\left(\lim_{n\to\infty}(p\ln n-(\ln n)\ln\ln n)\right)\\
&=\exp\left(\lim_{n\to\infty}-(\ln n)\ln\ln n\left(1-\frac{p}{\ln\ln n}\right)\right)\\
&=0.
\end{split}
\end{equation*}
Thus we have shown that
\begin{equation*}
\frac{1}{(\ln n)^{\ln n}}
\lll 
\frac{1}{n^p} 
\text{ as } n\to\infty
\end{equation*}
for every fixed real number $p$.
Since $\DS\sum_{n=1}^\infty\frac{1}{n^2}$ converges by $p$-test with $p=2$,
it follows that $\DS\sum_{n=2}^\infty\frac{1}{(\ln n)^{\ln n}}$ converges by LCT.
}
\fi

\newpage

\begin{example}
Determine if the series $\DS\sum_{n=10}^\infty\frac{1}{(\ln n)^{\ln\ln n}}$ converges or diverges.
\end{example}

\ifdefined\SOLUTION
\SOLUTION{
Again the terms look sort of, but not quite, like a $p$-series.
However, this time the exponent $\ln\ln n\to\infty$ as $n\to\infty$, but it does so at a much slower rate.
So, it is not immediately clear if this new series converges or diverges.
Proceeding in a similar manner to the last example,
\begin{equation*}
\begin{split}
\lim_{n\to\infty}\frac{1}{(\ln n)^{\ln\ln n}} \Big/\frac{1}{n^p} 
&=\lim_{n\to\infty}\frac{n^p}{(\ln n)^{\ln\ln n}}\\
&=\lim_{n\to\infty}\frac{\exp(p\ln n)}{\exp(-(\ln\ln n)^2}\\
&=\exp\left(\lim_{n\to\infty}(p\ln n-(\ln\ln n)^2\right)\\
&=\exp\left(\lim_{n\to\infty}(\ln n)\left(p-\frac{(\ln\ln n)^2}{\ln n}\right)\right)\\
&=\exp\left(\lim_{m\to\infty}m\left(p-\left(\frac{\ln m}{\sqrt m}\right)^2\right)\right)\\
&=\infty.
\end{split}
\end{equation*}
Thus we have shown that
\begin{equation*}
\frac{1}{(\ln n)^{\ln\ln n}}
\ggg 
\frac{1}{n^p} 
\text{ as } n\to\infty
\end{equation*}
for every fixed real number $p$.
Since $\DS\sum_{n=1}^\infty\frac{1}{n}$ diverges by $p$-test with $p=1$,
it follows that $\DS\sum_{n=2}^\infty\frac{1}{(\ln n)^{\ln\ln n}}$ diverges by LCT.
}
\fi
