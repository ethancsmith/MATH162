%!TEX root =  main.tex

\lectureheader{162}{Calculus II}{The comparison tests}{\textit{Thomas' Calculus} \textsection 10.4}

\begin{theorem}[Direct comparison test]\,
\begin{enumerate}
\item If $0\le a_n\le b_n$ for all $n\ge N$ and $\DS\sum_{n\ge N} b_n$ converges, then $\DS\sum_{n\ge N}a_n$ converges.
\item If $0\le b_n\le a_n$ for all $n\ge N$ and $\DS\sum_{n\ge N} b_n$ diverges, then $\DS\sum_{n\ge N}a_n$ diverges.\label{direct comparison test}
\end{enumerate}
\end{theorem}
%\begin{proof}[Proof of~\eqref{direct comparison test}]\,
%
%\vspace{6in}
%\end{proof}

\begin{remark}
There is an art to applying the direct comparison test, but there is strategy to this art.
\begin{enumerate}
\item Ask yourself what ``simpler" series $\sum b_n$ has terms that look similar to $\sum a_n$.
\item If $\sum b_n$ is known to converge, try to show that $0\le a_n\le b_n$ eventually.  If you succeed, then $\sum a_n$ also converges.
\item If $\sum b_n$ is known to diverge, try to show that $0\le b_n\le a_n$ eventually. If you succeed, then $\sum a_n$ also diverges.
\end{enumerate}
As with any art, mastery comes with much experience, and experience is only got by a lot of practice.
\end{remark}

\newpage

\begin{example}
Determine if $\DS\sum_{n=1}^\infty\frac{5}{10n-1}$ converges or diverges.
\end{example}

\newpage

\begin{example}
Determine whether the series $\DS\sum_{n=0}^\infty\frac{1}{n!}$  converges or diverges.
\end{example}

\newpage

\begin{example}
Determine whether the series $\DS\sum_{n=0}^\infty\frac{1}{2^n+\sqrt n}$  converges or diverges.
\end{example}

\newpage

\begin{remark}\,
\begin{itemize}
\item The test for divergence tells us that if $\sum a_n$ is going to have a chance at convergence, then we must have $a_n\to 0$ as $n\to\infty$.
\item The difference between series convergence and divergence is ``how fast" the terms $a_n$ tend to zero.
\item All the other series tests are trying to get a sense of whether the terms $a_n$ are going to zero ``fast enough" to ensure convergence.
\item The comparison tests are about determining convergence by comparing the relative sizes of the terms.
\item For the direct comparison test we have to be careful to get the inequalities correct for all sufficiently large $n$.
\item The limit comparison test allows for ``cruder" comparison using limits (relative orders at infinity).
\end{itemize}
\end{remark}

\begin{theorem}[Limit comparison test]
Suppose that $a_n, b_n>0$ for all $n\ge N$.
\begin{enumerate}
\item If $a_n\asymp b_n$ as $n\to\infty$, in particular if $\DS\lim_{n\to\infty}\frac{a_n}{b_n}=L\in (0,\infty)$, 
then $\sum a_n$ and $\sum b_n$ both converge or both diverge.
\item If $a_n\lll b_n$ as $n\to\infty$ (i.e., if $\DS\lim_{n\to\infty}\frac{a_n}{b_n}=0$) and $\sum b_n$ converges, then $\sum a_n$ converges.
\item If $a_n\ggg b_n$ as $n\to\infty$ (i.e., if $\DS\lim_{n\to\infty}\frac{a_n}{b_n}=\infty$) and $\sum b_n$ diverges, then $\sum a_n$ diverges.
\end{enumerate}
\end{theorem}

\newpage

\begin{example}
Determine if the series $\DS\sum_{n=1}^\infty\frac{2n+1}{(n+1)^2}$ converges or diverges.
\end{example}

\newpage

\begin{example}
Determine if the series $\DS\sum_{n=1}^\infty\frac{1}{2^n-1}$ converges or diverges.
\end{example}

\newpage

\begin{example}
Determine if the series $\DS\sum_{n=1}^\infty\frac{\ln n}{n^{3/2}}$ converges or diverges.
\end{example}

\newpage

\begin{example}
Determine if the series $\DS\sum_{n=1}^\infty\frac{1+n\ln n}{n^2+5}$ converges or diverges.
\end{example}

\newpage

\begin{example}
Determine if the series $\DS\sum_{n=2}^\infty\frac{1}{(\ln n)^{\ln n}}$ converges or diverges.
\end{example}

\newpage

\begin{example}
Determine if the series $\DS\sum_{n=10}^\infty\frac{1}{(\ln n)^{\ln\ln n}}$ converges or diverges.
\end{example}

\newpage

\begin{example}
Determine if the series $\DS\sum_{n=10}^\infty\frac{1}{n^{\arctan n}}$ converges or diverges.
\end{example}
