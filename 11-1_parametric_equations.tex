%!TEX root =  main.tex

\lectureheader{162}{Calculus II}{Parametric equations}{\textit{Thomas' Calculus}  11.1}

\begin{definition}
Suppose that $x=f(t)$ and $y=g(t)$ are functions of some third variable $t$, called a \textbf{parameter}.
We say that $x$ and $y$ are given by \textbf{parametric equations}.
The set of points $(x,y)$ traced in the plane as $t$ varies is called a \textbf{parametric curve}.
\end{definition}

\begin{example}
Consider the curve given by the parametric equations
\begin{align*}
x&=t^2-2t,\\
y&=t+1
\end{align*}
for $t\in [0,\infty)$.
Plot the points on the curve corresponding to $t=0, 1, 2, 3$.
Then eliminate the parameter to and identify the curve and finish the sketch.
\end{example}

\ifdefined\SOLUTION
\SOLUTION{
We make a table of values:
\begin{center}
\begin{tabular}{c|c|c}
$t$ & $x$ & $y$\\
\hline
$0$ &  $0$ & $1$\\
$1$ & $-1$ & $2$\\
$2$ &  $0$ & $3$\\
$3$ &  $3$ & 4
\end{tabular}
\end{center}
Solving the equation $y=t+1$ for $t$ and substituting we have
\begin{equation*}
\begin{split}
x &= (y-1)^2 -2(y-1)\\
  &= y^2-4y+3.
\end{split}
\end{equation*}
Now as $t$ varies in the interval $[0,\infty)$, $y=t+1$ varies in the interval $[1,\infty)$.
Therefore, the parametric equations trace out 
\begin{equation*}
x = y^2-4y+3\quad (1\le y<\infty).
\end{equation*}
This is part of a parabola that opens toward the right.
However, the parametrization also imposes an orientation: 
the parabola is traversed from the point $(x,y)=(0,1)$ upwards.
}
\fi

\newpage

\begin{example}
Consider the curve given by the parametric equations
\begin{align*}
x &= t + \frac{1}{t},\\
y &= t - \frac{1}{t} \quad (t>0).
\end{align*}
Eliminate the parameter to identify the curve.
\end{example}

\ifdefined\SOLUTION
\SOLUTION{
Adding the equations, we find that
\begin{equation*}
x+y = 2t,
\end{equation*}
and subtracting the equation, we have
\begin{equation*}
x-y = \frac{2}{t}.
\end{equation*}
Multiplying these two new equations, we find that
\begin{equation*}
x^2-y^2 = (x+y)(x-y) = 2t\frac{2}{t} = 4.
\end{equation*}
As $t$ varies in the interval $(0,\infty)$, 
we see that $x=t + 1/t$ also varies in the interval $(0,\infty)$ and $y=t-1/t$ varies in the interval $(-\infty,\infty)$.
Therefore, the parametric equations trace out the right half of the hyperbola
\begin{equation*}
x^2-y^2 = 4 \quad (x>0)
\end{equation*}
but it does so from bottom to top.
}
\fi

\newpage

\begin{remark}\,
\begin{enumerate}
\item One advantage of parametric equations is their ``explicit" (as opposed to ``implicit") nature.
\begin{enumerate}
\item To find a point on the curve given by 
\begin{equation*}
\frac{x^2}{16}+\frac{y^2}{9}=1,
\end{equation*}
we have to pick a value for one of the variables and then \underline{try} to solve for the other.
\item To find a point on the curve given by
\begin{align*}
x &=  4\cos t,\\
y &=  3\sin t\quad (0\le t\le 2\pi)
\end{align*}
we simply choose any $t\in [0,2\pi]$, and we immediately get a point on the curve every time.
\end{enumerate}
\item Another advantage of parametrization is that it imposes ``orientation" on a curve (direction of traverse) which some applications require.
For example, we may want to model the``position" of some object that depends on some non-spatial parameter (e.g., time).
\item A third advantage of parametrization is that each coordinate is given as a function of the parameter $t$, and it is far easier to apply the tools of calculus to functions.
\end{enumerate}
\end{remark}

\newpage

\begin{example}
Two particles start at the same instant in time.
The first travels along the path 
\begin{equation*}
C_1: x = \frac{16}{3}-\frac{8}{3}t,\quad y = 4t-5 \quad (t\ge 0)
\end{equation*}
while the second travels along the path 
\begin{equation*}
C_2: x = 2\sin\left(\frac{\pi}{2}t\right),\quad y = -3\cos\left(\frac{\pi}{2}t\right) \quad (t\ge 0).
\end{equation*}
At what points, if any, do the paths intersect?  At what points, if any, do the particles collide?
\end{example}

\ifdefined\SOLUTION
\SOLUTION{
The equations for $C_1$ may be rewritten as
\begin{align*}
3x &= 16 - 8t,\\
2y &= 8t - 10.
\end{align*}
Adding these equations, we find that
\begin{equation*}
3x+2y = 6\quad (y\ge -5),
\end{equation*}
which is a ``half-line."
The equations for $C_2$ may be rewritten as $C_2$ as
\begin{align*}
\frac{x}{2} &= \sin\left(\frac{\pi}{2}t\right),\\
\frac{y}{-3}&= \cos\left(\frac{\pi}{2}t\right).
\end{align*}
Whence, squaring and adding, the usual Pythagorean identity gives
\begin{equation*}
\frac{x^2}{4} + \frac{y^2}{9} = 1,
\end{equation*}
which is an ellipse that is traversed counterclockwise infinitely many times starting at the point $(x,y) = (0,-3)$.
To determine if the curves intersect, we simultaneously solve the cartesian equations.
We rewrite the cartesian equation for $C_1$ as $3x = 6-2y$ 
and the cartesian equation for $C_2$ as $(3x)^2 + 4y^2 = 36$.
Substituting the first into the later, we have
\begin{equation*}
(6-2y)^2 +4y^2 = 36.
\end{equation*}
Expanding, we have 
\begin{equation*}
36 - 24y + 8y^2 = 36.
\end{equation*}
Whence,
\begin{equation*}
8y(y-3) = 0,
\end{equation*}
and so $y=0$ or $y=3$.
Substituting these values back into the equation for $C_1$, we find that $x=2$ in the first case and $x=0$ in the latter.
Therefore, the curves intersect at $(x,y)=(2,0)$ and $(x,y)=(0,3)$.
The particle on $C_1$ is at $(x,y)=(2,0)$ if and only if $t=5/4$, and it is at $(x,y)=(0,3)$ if and only if $t=2$.
When $t=5/4$, the particle on $C_2$ is at $(x,y) = \Big(2\sin(5\pi/8), -3\cos(5\pi/8)\Big)\ne (2,0)$.
So the particles do not collide at $(2,0)$.
However, when $t=2$, the particle on $C_2$ is at $(x,y) = (0,3)$, and so the particles do collide at this point.
}
\fi

\newpage

\begin{remark}
The following ``bread and butter" parametrizations are important to know.
\begin{itemize}
\item
Given distinct points $(a,b), (c,d)\in\R^2$, the unique line passing through $(a,b)$ and $(c,d)$ may be parametrized by the equations
\begin{align*}
x & = a(1-t) + ct,\\
y &=  b(1-t) + dt
\end{align*}
for $t\in (-\infty, \infty)$.
%\begin{remark}
%The line segment \underline{from} $(a,b)$ \underline{to} $(c,d)$ is traversed as $t$ varies in $[0,1]$.
%\end{remark}
\item
Given $a, b\in \R$, the ellipse
\begin{equation*}
\frac{x^2}{a^2}+\frac{y^2}{b^2}=1
\end{equation*}
is parametrized with counterclockwise orientation by the equations
\begin{align*}
x & = a\cos t,\\
y &= b\sin t
\end{align*}
for $t\in [0,2\pi]$.
\item
Given $a, b\in \R$, the the hyperbola
\begin{equation*}
\frac{x^2}{a^2}-\frac{y^2}{b^2}=1
\end{equation*}
is parametrized by the equations
\begin{align*}
x & = a\sec t,\\
y &= b\tan t
\end{align*}
for $t\in [0,2\pi]$.
\item
Given $a, b\in \R$, the right branch of the hyperbola
\begin{equation*}
\frac{x^2}{a^2}-\frac{y^2}{b^2}=1
\end{equation*}
is parametrized by the equations
\begin{align*}
x & = a\cosh t,\\
y &= b\sinh t
\end{align*}
for $t\in (-\infty,\infty)$.
\item If $y=f(x)$ is a function of $x$, then the curve
\begin{equation*}
y=f(x)\quad (a\le x\le b)
\end{equation*}
is parametrized by the equations
\begin{align*}
x &= t,\\
y &=f(t)
\end{align*}
for $t\in [a,b]$.
\end{itemize}
\end{remark}

