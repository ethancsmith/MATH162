%!TEX root =  main.tex

\lectureheader{162}{Calculus II}{Separable differential equations}{\textit{Thomas' Calculus} \textsection 7.4}

\begin{definition}
An \textbf{ordinary differential equation} (ODE) is an equation relating an unknown function $y$, its derivatives, and some given functions of its independent variable.
To solve the equation is to find all functions $y$ which satisfy the equation, though in practice it is sometimes sufficient to find a single solution.
\end{definition}

\begin{remark}
The full solution set to a differential equation of the form
\begin{equation*}
\frac{\dee y}{\dee x} = f(x)
\end{equation*}
is (by definition) the indefinite integral $y = \int f(x)\dee x$.
\end{remark}

\begin{definition}
A quantity $y$ is said to exhibit \textbf{exponential change} if its rate of change is proportional to its size at a given time $t$, i.e., 
\begin{equation}\label{exponential change de}
\frac{\dee y}{\dee t} = ky
\end{equation}
for some constant $k$.
\end{definition}

\begin{remark}
Examples of exponential change include unchecked population growth, amount of a decaying radioactive substance, and the temperature difference between a hot object and its surrounding medium.
\end{remark}

\begin{example}
Explain the name ``exponential change" by solving the differential equation~\eqref{exponential change de}.
\end{example}

\newpage

\begin{definition}
A differential equation is said to be \textbf{separable} if it takes the form
\begin{equation}\label{separable de}
\frac{\dee y}{\dee x} = g(x)h(y),
\end{equation}
where $g$ is a function of $x$ alone and $h$ is a function of $y$ alone.
\end{definition}

\begin{remark}\,
Many separable equations~\eqref{separable de} can be solved (at least implicitly) by first observing that
\begin{equation*}
\dee y = \frac{\dee y}{\dee x}\dee x = g(x)h(y)\dee x
\end{equation*}
so that
\begin{equation*}
\int\frac{\dee y}{h(y)} = \int g(x)\dee x.
\end{equation*}
\end{remark}

\begin{example}
Solve the ODE
\begin{equation*}
\frac{\dee y}{\dee x} = (1+y)\E^x\quad (y>-1).
\end{equation*}
\end{example}

\newpage

\begin{example}
Solve the ODE
\begin{equation*}
y(x+1)\frac{\dee y}{\dee x} = x(y^2+1).
\end{equation*}
\end{example}

