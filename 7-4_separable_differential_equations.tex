%!TEX root =  main.tex
\setcounter{chapter}{7}
\setcounter{section}{4}
\setcounter{theorem}{0}
\setcounter{equation}{0}

\lectureheader{162}{Calculus II}{Separable differential equations}{\textit{Thomas' Calculus}  \thesection}

\begin{definition}
An \textbf{ordinary differential equation} (ODE) is an equation relating an unknown function $y$, its derivatives, and some given functions of its independent variable.
To solve the equation is to find all functions $y$ which satisfy the equation, though in practice it is sometimes sufficient to find a single solution.
\end{definition}

\begin{remark}
The full solution set to a differential equation of the form
\begin{equation*}
\frac{\dee y}{\dee x} = f(x)
\end{equation*}
is (by definition) the indefinite integral $y = \int f(x)\dee x$.
\end{remark}

\begin{definition}
A quantity $y$ is said to exhibit \textbf{exponential change} if its rate of change is proportional to its size at a given time $t$, i.e., 
\begin{equation}\label{exponential change de}
\frac{\dee y}{\dee t} = ky
\end{equation}
for some constant $k$.
\end{definition}

\begin{remark}
Examples of exponential change include unchecked population growth, amount of a decaying radioactive substance, and the temperature difference between a hot object and its surrounding medium.
\end{remark}

\begin{example}
Explain the name ``exponential change" by solving the differential equation~\eqref{exponential change de}.
\end{example}

\ifdefined\SOLUTION
\SOLUTION[Explanation] {
\begin{align*}
    \dee y &= \frac{\dee y}{\dee t}\dee t \\
    \dee y &= ky\dee t \\
    \frac{\dee y}{y} &= k\dee t \\
    \int_{}^{} \frac{\dee y}{y} \,\ &= \int_{}^{} k \,\dee t\ \\
    \ln{y} &= kt + C_0 \\    
    |y| &= \E^{kt + C_0} = C_1\E^{kt}, \text{ where } C_1 = \E^{C_0}.
\end{align*}
Thus, it follows that 
\begin{equation*}
    y = 
    \begin{cases}
        C_1\E^{kt} & \text{if } y > 0\\
        -C_1\E^{kt} & \text{if } y < 0\\
    \end{cases}
\end{equation*}
So, $y = C\E^{kt}$, where $C \neq 0$ is the full solution set to (9).
}
\else
\fi


\newpage

\begin{definition}
A differential equation is said to be \textbf{separable} if it takes the form
\begin{equation}\label{separable de}
\frac{\dee y}{\dee x} = g(x)h(y),
\end{equation}
where $g$ is a function of $x$ alone and $h$ is a function of $y$ alone.
\end{definition}

\begin{remark}\,
Many separable equations~\eqref{separable de} can be solved (at least implicitly) by first observing that
\begin{equation*}
\dee y = \frac{\dee y}{\dee x}\dee x = g(x)h(y)\dee x
\end{equation*}
so that
\begin{equation*}
\int\frac{\dee y}{h(y)} = \int g(x)\dee x.
\end{equation*}
\end{remark}

\begin{example}
Solve the ODE
\begin{equation*}
\frac{\dee y}{\dee x} = (1+y)\E^x\quad (y>-1).
\end{equation*}
\end{example}

\ifdefined\SOLUTION
\SOLUTION[Solution] {
\begin{align*}
    \frac{\dee y}{\dee x} &= (1+y)\E^x \\
    \frac{\dee y}{\dee x}\dee x &= (1+y)\E^x \dee x \\
    \dee y &= (1+y)\E^x \dee x \\
    \frac{\dee y}{1+y} &= \E^x \dee x \\
    \int_{}^{} \frac{\dee y}{1+y} \, \ &= \int_{}^{} \E^x \, \dee x \ \\
    \ln{|1+y|} &= \E^x + C_0 \\
    1+y &= \E^{(\E^x + C_0)} = C_1\E^{\E^x}, \text{ where } C_1 = \E^{C_0} > 0.
\end{align*}
Since we are given $y>1$, we conclude that $1+y = C_1\E^{\E^x}$.  Therefore, $y = C_1\E^{\E^x}-1, C_1 > 0$ is the full solution set.
}
\else
\fi
\newpage

\begin{example}
Solve the ODE
\begin{equation*}
y(x+1)\frac{\dee y}{\dee x} = x(y^2+1).
\end{equation*}
\end{example}

\ifdefined\SOLUTION
\SOLUTION[Solution] {
\begin{align*}
    y(x+1)\frac{\dee y}{\dee x} &= x(y^2+1) \\
    y(x+1)\frac{\dee y}{\dee x}\dee x &= x(y^2+1) \dee x \\
    y(x+1)\dee y &= x(y^2+1) \dee x \\
    \frac{y}{y^2 + 1}\dee y &= \frac{x}{x+1}\dee x \\ 
    \int_{}^{} \frac{y}{y^2 + 1} \, \dee y\ &= \int{}^{} \frac{x}{x+1} \, \dee x \ \\
\end{align*}
Let $u = y^2 + 1$. So, $\dee u = 2y\dee y$.  Let $w = x+1$.  So, $\dee w = \dee x$. Rewriting the integrals with the substitutions gives
\begin{align*}
    \frac{1}{2}\int_{}^{} \frac{\dee u}{u} \, \ &= \int{}^{} \frac{w-1}{w} \, \dee w \ \\
    \frac{1}{2}\int_{}^{} \frac{\dee u}{u} \, \ &= \int{}^{} 1 - \frac{1}{w} \, \dee w \ \\
    \frac{1}{2}\ln{|u|} &= w - \ln{|w|} + C \\
    \frac{1}{2}\ln{(y^2 + 1)} &= x+1 - \ln{|x+1|} + C \\
\end{align*}
\textit{Note:} this gives y implicitly as a function of x. 
}
\else
\fi
