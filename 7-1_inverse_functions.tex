%!TEX root =  main.tex

\lectureheader{162}{Calculus II}{Inverse functions}{\textit{Thomas' Calculus} 7.1}

\begin{definition}
Let $A$ and $B$ be subsets of $\R$, and let $f$ be a function with domain $\dom(f)=A$ and range $\rng(f)=B$.
We say that $f$ is \textbf{invertible} if there is a function $f^{-1}$ with domain $\dom(f^{-1})=B$ and range $\rng(f^{-1})=A$ so that
\begin{align*}
f^{-1}(f(a)) &= a\quad (a\in A),\\
f(f^{-1}(b)) &= b\quad (b\in B),\\
\end{align*}
i.e.,
\begin{equation*}
f^{-1}(b)=a \iff f(a)=b.
\end{equation*}
If $f^{-1}$ exists, we say that $f^{-1}$ is the \textbf{inverse} of $f$.
\end{definition}

\begin{remark}
If $f^{-1}$ exists, then
\begin{enumerate}
\item the graph of the function $y=f^{-1}(f(x))$ coincides with the graph of $y=x$ for $x\in A$, 
\item the graph of $y=f(f^{-1}(x))$ coincides with the graph of $y=x$ for $x\in B$, and
\item the graph of $y=f^{-1}(x)$ is the reflection of the graph of $y=f(x)$ across the line $y=x$.
\end{enumerate}
\end{remark}

\begin{example}
Explain why $g(x) = \sqrt x$ is \underline{not} the inverse of $f(x)=x^2$ over their natural domains.
\end{example}

\ifdefined\SOLUTION
\SOLUTION{The two are not inverses for two different reasons (either of which is sufficient).
\begin{enumerate}
\item Firstly, $\dom(f) = (-\infty, \infty) \ne [0,\infty) = \rng(g)$.
\item Secondly, 
\begin{equation*}
g(f(x)) = g(x^2) = \sqrt{x^2} = |x| \ne x
\end{equation*}
for at least one $x\in\dom(f)=(-\infty, \infty)$.
In particular, $-1\in\dom(f)$ and 
\begin{equation*}
g(f(-1)) = 1\ne -1.
\end{equation*}
Note: It is wrong to say that $\sqrt{x^2} = \pm x$.
The reason is that the expression on the right is \underline{not} a function while the one on the left is.
For every $x$ other than zero, the expression on the left returns exactly one number, while the expression on the right returns two numbers.
For example, $\sqrt{(-1)^2} = 1$ whereas $\pm (-1) = \mp 1$, both $-1$ and $1$.
Put another way, the graph of $y=\sqrt{x^2}$ is a V-shape, which passes the vertical line test;
the graph of $y=\pm x$ is an X-shape, which does not pass vertical line test.
\end{enumerate}
}
\else
\vfill
\fi

\newpage

\begin{example}
Compute the inverse of $f(x)=3x+1$ over its natural domain.
\end{example}
\ifdefined\SOLUTION
\SOLUTION[Solution]{
First, note that $\dom(f) = \R$ since $f$ is a polynomial.
Now let $x = f(y)$, and observe that
\begin{align*}
    x = 3y + 1 &\iff x - 1 = 3y \\
    &\iff \frac{x-1}{3} = y.
\end{align*}
So put $f^{-1}(x) = \frac{x-1}{3}$, and observe that $\dom(f) = \R = \rng(f^{-1})$, and $\rng(f) = \R = \dom(f^{-1})$.  
Finally, we verify that
\begin{equation*}
f(f^{-1}(x)) = 3\left(\frac{x-1}{3}\right) + 1 = (x-1)+1 = x
\end{equation*}
for all $x\in\R = \dom(f^{-1})$, and
\begin{equation*}
f^{-1}(f(x)) = \frac{(3x+1)-1}{3} = \frac{3x}{3} = x
\end{equation*}
for all $x\in \R = \dom(f)$.
}
\else
\fi
\newpage

\begin{definition}
Let $f$ be a function with domain $A$.
We say that $f$ is \textbf{one-to-one} (or \textbf{injective}) if for all $a,b\in A$
\begin{equation*}
f(a)=f(b)\implies a=b.
\end{equation*}
%or equivalently,
%\begin{equation*}
%a\ne b\implies f(a)\ne f(b).
%\end{equation*}
\end{definition}

\begin{remark}
A function is one-to-one if every time you \textit{think} you've found two inputs that map to the \textit{same output}, it turns out that the inputs are really the \textit{same input}.
In other words, the definition of one-to-one is just a formal statement of the ``horizontal line test" that you may have learned in precalculus.
\end{remark}

\begin{example}
Use the definition (and algebra) to show that $f(x)=3x+1$ is one-to-one on its natural domain.
\end{example}

\ifdefined\SOLUTION
\SOLUTION{
Let $a$ and $b$ be elements of the real numbers. Then
\begin{align*}
    f(a) = f(b) &\implies 3a + 1 = 3b + 1 \\
    &\implies 3a = 3b \\
    &\implies a = b.
\end{align*}
Therefore, $f$ is one-to-one.
}
\else
\fi
\vfill

\begin{remark}
To show that an implication ($\implies$ statement) is \underline{not} true, 
we demonstrate a \underline{specific} situation where the hypothesis (i.e., ``if"-part) is true and yet the conclusion (i.e., ``then"-part) is false.
\end{remark}

\begin{example}
Use the definition to show that $f(x)=x^2$ is \underline{not} one-to-one on its natural domain.
\end{example}
\ifdefined\SOLUTION
\SOLUTION{
Note that $f(2) = 2^2 = 4 = (-2)^2 = f(-2)$ and yet $2 \neq -2$.
}
\else
\fi
\vfill

\newpage

\begin{theorem}
If $f$ is strictly monotone (increasing or decreasing) over an interval, then $f$ is one-to-one on that interval.
\end{theorem}

\ifdefined\SOLUTION
\SOLUTION[Proof when $f$ is strictly increasing]{
Suppose that $f$ is strictly increasing over the interval $I$, and let $a,b\in I$ so that $f(a)=f(b)$.
To show that $f$ is one-to-one, it suffices to show that $a=b$.
We will show this by showing that it is impossible that $a\ne b$.
First suppose that $a<b$.
Then since $f$ is strictly increasing, it follows that $f(a)<f(b)$, contradicting the standing assumption that $f(a)=f(b)$.
Therefore, we must conclude that $a\ge b$.
Similarly, if $a>b$, then it follows that $f(b)<f(a)$ and we again have a contradiction.
Whence, we must conclude that $a=b$.
Therefore, $f$ is one-to-one over $I$.}
\vfill
\else
\begin{proof}[Proof when $f$ is strictly increasing]\,

\vspace{3.5in}

\end{proof}
\fi

\begin{example}
Use the above theorem to show that $f(x)=3x+1$ is one-to-one on its domain.
\end{example}

\ifdefined\SOLUTION
\SOLUTION{
Observe that $f'(x) = 3 > 0$ for all real $x$.  So, $f$ is increasing and therefore one-to-one on the real numbers.
}
\else
\fi
\vfill

\newpage

\begin{theorem}
A function is invertible if and only if it is one-to-one.
\end{theorem}

\begin{remark}
Students taking discrete math or combinatorics may notice that the above theorem does not mention that an invertible function also needs to be \textit{onto} (i.e., \textit{surjective}).
That is because we can afford to be a little less precise with our definition of invertible in calc II.
\end{remark}

\begin{corollary}
Let $a\in\R$, and let $I$ be an open interval containing $a$.
If $f$ is differentiable on $I$, $f'(a)\ne 0$, and $f'$ does not change sign on $I$, then $f$ is invertible on $I$.
\end{corollary}

\ifdefined\SOLUTION
\SOLUTION[Proof]{Suppose that $f$ is differentiable on $I$, $f'(a)\ne 0$ and $f'$ does not change sign on $I$.
If $f'(a)>0$, then $f$ is strictly increasing on $I$, and if $f'(a)<0$, then $f$ is strictly decreasing on $I$. 
In either case, $f$ is one-to-one on $I$, and by the above theorem, it follows that $f$ is invertible.
}
\else
\begin{proof}\,

\vspace{4in}

\end{proof}
\fi

\newpage

\begin{example}
Show that the function
\begin{equation*}
f(x)=x^2\quad (x\ge 0)
\end{equation*}
is invertible on its domain.
What is its inverse?
\end{example}
\ifdefined\SOLUTION
\SOLUTION{
Note that $f'(x) = 2x > 0$ for all $x\in (0,\infty)$.  Therefore, $f(x)  = x^2$ is one-to-one and invertible for $x\in (0,\infty)$.
Since $f(x) = x^2 = 0 \iff x = 0, f$ is one-to-one on $[0,\infty)$.  So, $f$ is invertible on $[0,\infty)$, and its inverse is $f^{-1}(x) = \sqrt{x}$.\\
\indent In fact, the symbol $\sqrt x$ is just the ``invented name" for this inverse function that we know must exist.
Note that this is kind of like ``cheating."
If we ask ``What is $\sqrt{3}$?", the answer is that it is the solution to the equation $x^2=3$.
And if we ask ``What is the solution so the equation $x^2=3$?", we answer that it is $x=\sqrt 3$?
Our analytic reasoning tells us that such a number must exist and since its not one that we already know, we simply ``invent a name for it."
We're going to do a lot of this sort of ``inventing" throughout chapter 7.
}
\else
\fi
\vfill

\begin{example}
Show that the function
\begin{equation*}
f(x)=x^2\quad (x\le 0)
\end{equation*}
is invertible on its domain.
What is its inverse?
\end{example}
\ifdefined\SOLUTION
\SOLUTION{
Note that $f'(x) = 2x < 0$ for $x \in (-\infty, 0).$  Also, $f(x) = x^2 = 0 \iff x = 0$. So, $f$ is one-to-one on $(-\infty, 0],$ and therefore invertible on $(-\infty, 0]$.  
And $\dom(f) = (-\infty, 0] = \rng(f^{-1})$, and $\rng(f) = [0, \infty) = \dom(f^{-1})$. 
The inverse function is $f^{-1}(x) = -\sqrt{x}$.
}
\else
\fi
\vfill

\newpage

\begin{theorem}
Let $a\in\R$.
If $f$ is continuously differentiable at $x=a$ and $f'(a)\ne 0$, then $f^{-1}$ is differentiable at $x=b=f(a)$ and
 \begin{equation*}
 \left(f^{-1}\right)'(b) = \frac{1}{f'\big(f^{-1}(b)\big)}.
 \end{equation*}
\end{theorem}

\begin{example}
Show that $f(x)=2x+\cos x$ is invertible on $\R$.
Then compute $\frac{\dee f^{-1}}{\dee x}$ when $x=1$.
\end{example}

\ifdefined\SOLUTION
\SOLUTION{
Note that $f'(x) = 2 - \sin{x} \geq 2 -1 = 1 > 0$ for all $x\in \R$.  So, $f$ is one-to-one and invertible on $\R$.  
Furthermore, since
\begin{align*}
    f^{-1}(1) = y & \iff 1 = f(y) = 2y + \cos{y} \\
    & \iff y = 0,
\end{align*}
we have that
\begin{equation*}
    \left.\frac{\dee f^{-1}}{\dee x}\right|_{x = 1} 
    = \frac{1}{f'(f^{-1}(1))}
    = \frac{1}{2-\sin{(f^{-1}(1))}}
    = \frac{1}{2-\sin{0}}
    = \frac{1}{2}.
\end{equation*}
Note: We know that $f(x)=2x+\cos x$ has an inverse, but not even Mathematica can solve the equation $x=f(y)$ for $y$.
The reason is that there is no way to solve this equation in terms of functions that we already know.
We simply invent a name for this function (we call it $f^{-1}$ in this context), but the function is not generally important enough to give a special name for all time.
}
\else
\fi
\newpage

\begin{example}
Compute the natural domain and range of $f(x)=\sqrt{x^3+x^2+x+1}$.
Show that $f$ is invertible over its entire domain.
Then compute $\left(f^{-1}\right)'(2)$.
\end{example}

\ifdefined\SOLUTION
\SOLUTION[Solution]{
Observe that $\dom(f) = \{x : g(x) \geq 0\}$, where $g(x) = x^3 + x^2 + x + 1$.  
Note that $g'(x) = 3x^2 + 2x + 1,$ and the discriminant of $g'$ is $D = b^2 - 4ac = 2^2 - 4(3)(1) = 4 - 12 < 0$.  
Since $g'(0) = 1,$ we see that $g'(x) \geq 0$ for all $x\in \R$.  
So, $g(x)$ is increasing on $\R$.  
Since $g(-1) = (-1)^3 + (-1)^2 + (-1) + 1 = 0,$ we see that $g(x) \geq 0$ for all $x\in [-1, \infty)$.  
Therefore, $\dom(f) = [-1, \infty)$ and $\rng(f) = [0, \infty)$.
Now, note that $f'(x) = \frac{1}{2}(x^3+x^2+x+1)^{-\frac{1}{2}}\cdot(3x^2+2x+1) > 0$ for $x\in (-1, \infty)$.  
So, $f$ is one-to-one and invertible on $(-1, \infty)$.  
Since $f(x) = 0 \iff x = -1$, $f$ is invertible on $[-1, \infty)$.  
So, 
\begin{align*}
    f^{-1}(2) = y & \iff 2 = f(y) = \sqrt{y^3+y^2+y+1} \\
    & \iff y = 1. 
\end{align*}
And by the previous theorem, 
\begin{align*}
    \left(f^{-1}\right)'(2) & = \frac{1}{f'(f^{-1}(2))} \\
    & = \frac{1}{f'(1)} \\
    & = \frac{2\sqrt{4}}{6} = \frac{2}{3}. 
\end{align*}
}
\else
\fi
\newpage

\begin{example}
Show that 
\begin{equation*}
f(x) = 3+\frac{x^2}{2}+\tan(\pi x/2)\quad (-1<x<1)
\end{equation*}
is invertible over its domain.
Then compute $\frac{\dee f^{-1}}{\dee x}$ at $x=3$.
\end{example}
\ifdefined\SOLUTION
\SOLUTION[Solution]{
Note that $f'(x) = x + \frac{\pi}{2}\sec^2\left({\frac{\pi x}{2}}\right).$  Since $\cos{\theta} \leq 1$ for all $\theta$, $\sec{\theta} \geq 1$ for all $\theta.$  Therefore, $f'(x) = x + \frac{\pi}{2}\sec^2\left({\frac{\pi x}{2}}\right) \geq x + \frac{\pi}{2} > -1 + \frac{\pi}{2} > 0$ for all $x\in (-1,1).$  So, $f$ is increasing and invertible on $(-1,1).$  Hence, the previous theorem can be used.  So, 
\begin{align*}
    f^{-1}(3) = y & \iff 3 = f(y) = 3+\frac{y^2}{2}+\tan(\pi y/2)\\
    & \iff y = 0, 
\end{align*}
and
\begin{equation*}
     \left.\frac{\dee f^{-1}}{\dee x}\right|_{x=3}
     = \frac{1}{f'(f^{-1}(3))}
     = \frac{1}{f'(0)}
     = \frac{1}{0 + \frac{\pi}{2}\sec^2{(0)}}
     = \frac{2}{\pi}.
\end{equation*}
}
\else
\fi
%\end{document}   
