%!TEX root =  main.tex

\lectureheader{162}{Calculus II}{Inverse functions}{\textit{Thomas' Calculus} \textsection 7.1}

\begin{definition}
Let $A$ and $B$ be subsets of $\R$, and let $f$ be a function with domain $A$ and range $B$.
We say that $f$ is \textbf{invertible} if there is a function $f^{-1}$ with domain $B$ and range $A$ so that
\begin{align*}
f^{-1}(f(a)) &= a\quad (a\in A),\\
f(f^{-1}(b)) &= b\quad (b\in B),\\
\end{align*}
i.e.,
\begin{equation*}
f^{-1}(b)=a \iff f(a)=b.
\end{equation*}
If $f^{-1}$ exists, we say that $f^{-1}$ is the \textbf{inverse} of $f$.
\end{definition}

\begin{remark}
If $f^{-1}$ exists, then
\begin{enumerate}
\item the graph of the function $y=f^{-1}(f(x))$ coincides with the graph of $y=x$ for $x\in A$, 
\item the graph of $y=f(f^{-1}(x))$ coincides with the graph of $y=x$ for $x\in B$, and
\item the graph of $y=f^{-1}(x)$ is the reflection of the graph of $y=f(x)$ across the line $y=x$.
\end{enumerate}
\end{remark}

\begin{example}
Explain why $g(x) = \sqrt x$ is \underline{not} the inverse of $f(x)=x^2$ over their natural domains.
\end{example}

\ifdefined\SOLUTION
\SOLUTION{The two are not inverses for two different reasons (either of which is sufficient).
\begin{enumerate}
\item Firstly, $\dom(f) = (-\infty, \infty) \ne [0,\infty) = \rng(g)$.
\item Secondly, 
\begin{equation*}
g(f(x)) = g(x^2) = \sqrt{x^2} = |x| \ne x
\end{equation*}
for some $x\in\dom(f)=(-\infty, \infty)$.
\end{enumerate}
}
\else
\vfill
\fi

\newpage

\begin{example}
Compute the inverse of $f(x)=3x+1$ over its natural domain.
\end{example}

\newpage

\begin{definition}
Let $f$ be a function with domain $A$.
We say that $f$ is \textbf{one-to-one} (or \textbf{injective}) if for all $a,b\in A$
\begin{equation*}
f(a)=f(b)\implies a=b.
\end{equation*}
%or equivalently,
%\begin{equation*}
%a\ne b\implies f(a)\ne f(b).
%\end{equation*}
\end{definition}

\begin{remark}
A function is one-to-one if every time you \textit{think} you've found two inputs that map to the \textit{same output}, it turns out that the inputs are really the \textit{same input}.
In other words, the definition of one-to-one is just a formal statement of the ``horizontal line test" that you may have learned in precalculus.
\end{remark}

\begin{example}
Use the definition (and algebra) to show that $f(x)=3x+1$ is one-to-one on its natural domain.
\end{example}

\vfill

\begin{example}
Use the definition to show that $f(x)=x^2$ is \underline{not} one-to-one on its natural domain.
\end{example}

\vfill

\newpage

\begin{theorem}
If $f$ is strictly monotone (increasing or decreasing) over an interval, then $f$ is one-to-one on that interval.
\end{theorem}
\begin{proof}\,

\vspace{3.5in}

\end{proof}


\begin{example}
Use the above theorem to show that $f(x)=3x+1$ is one-to-one on its domain.
\end{example}

\vfill

\newpage

\begin{theorem}
A function is invertible if and only if it is one-to-one.
\end{theorem}

\begin{remark}
Students taking discrete math or combinatorics may notice that the above theorem does not mention that an invertible function also needs to be \textit{onto} (i.e., \textit{surjective}).
That is because we are using a simplified definition of invertible in calc II.
\end{remark}

\begin{corollary}
Let $a\in\R$, and let $I$ be an open interval containing $a$.
If $f$ is differentiable on $I$, $f'(a)\ne 0$, and $f'$ does not change sign on $I$, then $f$ is invertible on $I$.
\end{corollary}

\begin{proof}\,

\vspace{4in}

\end{proof}

\newpage

\begin{example}
Show that the function
\begin{equation*}
f(x)=x^2\quad (x\ge 0)
\end{equation*}
is invertible on its domain.
What is its inverse?
\end{example}

\vfill

\begin{example}
Show that the function
\begin{equation*}
f(x)=x^2\quad (x\le 0)
\end{equation*}
is invertible on its domain.
What is its inverse?
\end{example}

\vfill

\newpage

\begin{theorem}
Let $a\in\R$.
If $f$ is continuously differentiable at $x=a$ and $f'(a)\ne 0$, then $f^{-1}$ is differentiable at $x=a=f^{-1}(b)$ and
 \begin{equation*}
 \left(f^{-1}\right)'(b) = \frac{1}{f'\big(f^{-1}(b)\big)}.
 \end{equation*}
\end{theorem}

\begin{example}
Show that $f(x)=2x+\cos x$ is invertible on $\R$.
Then compute $\frac{\dee f^{-1}}{\dee x}$ when $x=1$.
\end{example}

\newpage

\begin{example}
Compute the natural domain and range of $f(x)=\sqrt{x^3+x^2+x+1}$.
Show that $f$ is invertible over its entire domain.
Then compute $\left(f^{-1}\right)'(2)$.
\end{example}

\newpage

\begin{example}
Show that 
\begin{equation*}
f(x) = 3+\frac{x^2}{2}+\tan(\pi x/2)\quad (-1<x<1)
\end{equation*}
is invertible over its domain.
Then compute $\frac{\dee f^{-1}}{\dee x}$ at $x=3$.
\end{example}

%\end{document}   
