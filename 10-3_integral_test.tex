%!TEX root =  main.tex

\lectureheader{162}{Calculus II}{Integral test}{\textit{Thomas' Calculus} \textsection 10.3}

\begin{definition}
Given an infinite series $\DS\sum_{n=0}^\infty a_n$, the sequence of \textbf{remainders} (or \textbf{tails}) is
\begin{equation*}
R_N = \sum_{n=N+1}^\infty a_n \quad (N\ge 0).
\end{equation*}
\end{definition}

\begin{remark}\,
\begin{itemize}
\item An infinite series converges if and only if one of its tails converges.
 \item This is because 
 \begin{equation*}
 \sum_{n=0}^\infty a_n = \sum_{n=0}^N a_n + \sum_{n=N+1}^\infty a_n = s_N + R_N
 \end{equation*}
 and $s_N$ is just the sum of finitely many real numbers.
 \item So, if we just want to know if the full series converges or diverges, it doesn't matter where we start $n$.
 \item If we know that the full series converges, then we can use the $N$th partial sum as an approximation for the series and the the $N$th remainder is the error in this approximation.
 \end{itemize}
\end{remark}

\newpage

\begin{theorem}[Integral test]
Suppose that $f$ is positive, continuous, and decreasing for all $x\ge a$.
Then
\begin{equation*}
\sum_{n=a}^\infty f(n) \text{ converges }\iff \int_a^\infty f(x)\dee x \text{ converges.}
\end{equation*}
\end{theorem}
\begin{remark}
As usual, it is important to tell the truth in our writing and distinguish between objects that are related but not necessarily equal.
The integral and the sum are related (if one converges then so does the other), but they are different objects and are most likely not equal.
\end{remark}
\textit{Proof.}

\newpage

\,

\vspace{8in}

\qed

\newpage

\begin{definition}
A $p$\textbf{-series} is a series of the form
\begin{equation*}
\sum_{n=1}^\infty \frac{1}{n^p},
\end{equation*}
where $p\in\R$.
The special case when $p=1$ is called the \textbf{harmonic series}.
\end{definition}

\begin{theorem}[$p$-series test]
The $p$-series converges if and only if $p>1$.
In particular, the harmonic series diverges.
\end{theorem}
\begin{remark}\,
\begin{itemize}
\item Euler developed a technique to determine the value of the series when $p$ is a positive even number.  For example, he showed
\begin{equation*}
\sum_{n=1}^\infty\frac{1}{n^2} = \frac{\pi^2}{6}.
\end{equation*}
\item To this day, no one knows the precise value of the series for odd $p\ge 3$.
\item Euler also showed that the divergence of the harmonic series can be used to give a novel proof that there are infinitely many prime numbers.
\item Riemann would later change the real variable $p$ to a complex variable $s$, and show how the series is connected to a much deeper theory of prime numbers.
\end{itemize}
\end{remark}
\begin{proof}\,

\vspace{4in}
\end{proof}

\newpage

\begin{example}
Determine whether the following converge or diverge.
\begin{enumerate}
\item $\DS\sum_{n=1}^\infty\frac{1}{n^2+1}$
\vfill
\item $\DS\sum_{n=1}^\infty n\E^{-n^2}$
\vfill
\item $\DS\sum_{n=1}^\infty\frac{1}{2^{\ln n}}$
\vfill
\end{enumerate}
\end{example}

\newpage

\begin{remark}\,
\begin{itemize}
\item If we know that a series $\DS\sum_{n=0}^\infty a_n$ converges, then we can approximate its sum by the $N$th partial sum, i.e., 
\begin{equation*}
\sum_{n=0}^\infty a_n \approx s_N = \sum_{n=0}^N a_n
\end{equation*}
when $N$ is large.
\item The error in this approximation 
\begin{equation*}
R_N = \sum_{n=0}^\infty a_n - s_N = \sum_{n=N+1}^\infty a_n
\end{equation*}
is called the \textbf{$N$-th remainder} (or \textbf{tail of the series}).
\end{itemize}
\end{remark}

\begin{theorem}[Integral test estimation theorem]
Suppose that $f$ is positive, continuous, and decreasing for all $x\ge N$, and let $a_n=f(n)$.
If $\DS\sum_{n=0}^\infty f(n)$ converges, then 
\begin{equation*}
0\le  \int_{N+1}^\infty f(x)\dee x\le R_N\le \int_{N}^\infty f(x)\dee x.
\end{equation*}
\end{theorem}

\vfill

\begin{remark}
Because the integral test estimation theorem gives a nonnegative lower bound on the $N$th remainder, we can use it to improve our $N$th partial sum approximation.
\end{remark}

\begin{corollary}
Suppose that $f$ is positive, continuous, and decreasing for all $x\ge N$.
If $\DS\sum_{n=0}^\infty f(n)$ converges, then 
\begin{equation*}
\sum_{n=0}^Nf(n) +  \int_{N+1}^\infty f(x)\dee x\le \sum_{n=0}^\infty f(n)\le \sum_{n=0}^Nf(n) +  \int_{N}^\infty f(x)\dee x.
\end{equation*}
\end{corollary}

\newpage

\begin{example}
Although we know that $\DS\sum_{n=1}^\infty\frac{1}{n^3}$ exists (i.e., is a finite real number), we don't know very much about it except that Ap\'ery proved that it is an irrational number.
\begin{enumerate}
\item Bound the error in the approximation
\begin{equation*}
\sum_{n=1}^\infty\frac{1}{n^3} \approx \sum_{n=1}^{10}\frac{1}{n^3} = \frac{19164113947}{16003008000} = 1.19753\dots.
\end{equation*}
\vfill
\item  Now bound the error in the approximation
\begin{equation*}
\sum_{n=1}^{\infty}\frac{1}{n^3}\approx \sum_{n=1}^{10}\frac{1}{n^3} + \int_{11}^\infty\frac{\dee x}{x^3} =  \frac{2326859291587}{1936363968000} = 1.20166\dots.
\end{equation*}
\vfill
\end{enumerate}
\end{example}

\newpage

\begin{remark}\,
\begin{itemize}
\item The proof of the integral test really shows that if $f$ is positive, continuous, and decreasing for all $x\ge a$, then
\begin{equation*}
\sum_{n=a}^N f(n) \asymp \int_a^N f(x)\dee x\quad \text{as }N\to\infty.
\end{equation*}
\item This means that even in the case of divergence, the integral test may have more to say.
\end{itemize}
\end{remark}
\begin{example}
Determine the rate at which the harmonic series diverges to $\infty$.
\end{example}

\vfill

\begin{remark}
Euler showed that $\DS\lim_{N\to \infty}\left(\sum_{n=1}^N\frac{1}{n} - \ln(N)\right)$ exists.
Today the value of the limit is denoted by $\gamma$ and called the \textit{Euler--Mascheroni constant}.
\end{remark}
