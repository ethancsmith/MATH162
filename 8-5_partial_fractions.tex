%!TEX root =  main.tex
\setcounter{chapter}{8}
\setcounter{section}{5}
\setcounter{theorem}{0}
\setcounter{equation}{0}

\lectureheader{162}{Calculus II}{Partial fraction decomposition}{\textit{Thomas' Calculus}  \thesection}

\begin{example}
Use the fact that
\begin{equation*}
\frac{5x-3}{x^2-2x-3} = \frac{3}{x-3} + \frac{2}{x+1}
\end{equation*}
to compute $\DS \int\frac{5x-3}{x^2-2x-3} \dee x$.
\end{example}
\ifdefined\SOLUTION
\SOLUTION[Solution]{
\begin{align*}
    \int\frac{5x-3}{x^2-2x-3} \dee x 
    &= \int \frac{3}{x-3}\dee x + \int\frac{2}{x+1} \dee x \\
\end{align*}
Let $u = x - 3.$ It follows that $\dee u = \dee x$. Similarly, let $w = x + 1.$  So, $\dee w = \dee x.$  Making both those substitutions gives 
\begin{equation*}
    \int\frac{3\dee u}{u} + \int \frac{2\dee w}{w} 
    = 3\ln{|u|} + 2\ln{|w|} + C
    = 3\ln{|x-3|} + 2\ln{|x+1|} + C.
\end{equation*}
}
\else
\fi

\vfill
\begin{remark}
The process of ``un-common-denominatorizing" a rational function is known as \textit{partial fraction decomposition}.
We will apply the following strategy to integrate a rational function $f(x)/g(x)$.
\begin{enumerate}
\item Use polynomial long division to write 
\begin{equation*}
\frac{f(x)}{g(x)} = q(x) + \frac{r(x)}{g(x)},
\end{equation*}
where $\deg(r)<\deg(g)$.
\item Factor the denominator $g(x)$ into linear factors (of the form $ax+b$) and irreducible quadratic factors (of the form $ax^2+bx+c$ with $b^2-4ac<0$).
\item Write the remaining proper rational function $r(x)/g(x)$ as a sum of terms of the form
\begin{equation*}
\frac{A}{(ax+b)^i}\quad\text{or}\quad\frac{Ax+B}{(ax^2+bx+c)^i}.
\end{equation*}
\end{enumerate}
\end{remark}

\newpage

\begin{remark}
If the denominator $g(x)$ factors into a product of distinct linear factors
\begin{equation*}
\begin{split}
g(x) &= \prod_{j=1}^k(a_jx+b_j)\\
& = (a_1x+b_1)(a_2x+b_2)\cdots (a_kx+b_k),
\end{split}
\end{equation*}
then we solve for constants $A_1, A_2,\dots, A_k$ so that
\begin{equation*}
\begin{split}
\frac{r(x)}{g(x)} &= \sum_{j=1}^k\frac{A_j}{a_jx+b_j}\\ 
&= \frac{A_1}{a_1x+b_1} + \frac{A_2}{a_2x+b_2} + \dots +\frac{A_k}{a_kx+b_k}. 
\end{split}
\end{equation*}
\end{remark}

\begin{example}
Evaluate $\DS\int\frac{x^2+2x-1}{2x^3+3x^2-2x}\dee x$.
\end{example}
\ifdefined\SOLUTION
\SOLUTION[Solution]{
\begin{enumerate}
    \item  Since the degree of the numerator is less than the degree of the denominator, step one (long division) is not needed.
    \item  $g(x) = 2x^3 + 3x^2 - 2x = x(2x^2 + 3x - 2) = x(2x-1)(x+2).$
    \item Now we set up the partial fraction decomposition:
\begin{equation*}
\frac{x^2 + 2x - 1}{x(2x-1)(x+2)} = \frac{A}{x} + \frac{B}{2x-1} + \frac{C}{x+2}.
\end{equation*}
Clearing denominators we get 
\begin{equation*}
x^2 + 2x - 1 = A(2x-1)(x+2) + B(x)(x+2) + C(x)(2x-1).
\end{equation*}
Now we can recover $A$, $B$, and $C$ via clever substitutions for $x$.
\end{enumerate}
\begin{itemize}
    \item $x = 0$:  $-1 = A(-1)(2) + 0 + 0$ implies that $A = 1/2$. 
    \item $x = 1/2$:  $1/4 + 1 - 1 = 1/4 = 0 + B(1/2)(5/2) + 0$ implies that $B = (1/4)(2/1)(2/5) = 1/5$.
    \item $x = -2$:  $4 - 4 - 1 = -1 = 0 + 0 + C(-2)(-5)$ implies that $C = -1/((-2)(-5)) = -1/10$.
\end{itemize}
So, it follows that 
\begin{align*}
\int\frac{x^2+2x-1}{2x^3+3x^2-2x}\dee x 
&= \int\frac{(1/2) \dee x}{x} + \int\frac{(1/5) \dee x}{2x-1} + \int\frac{(-1/10) \dee x}{x + 2} \\
&= \frac{1}{2}\ln{|x|} + \frac{1}{10}\ln{|2x-1|} - \frac{1}{10}\ln|x+2| + C. 
\end{align*}
}
\newpage
\else
\newpage
\,
\newpage
\fi

\begin{remark}
If the denominator $g(x)$ factors into a product of linear factors with some repeated factors, say $(ax+b)^e$ divides $g(x)$, 
then instead of just the term $\frac{A}{ax+b}$, we must include $e$ terms
\begin{equation*}
\sum_{i=1}^e\frac{A_{i}}{(ax+b)^i} = \frac{A_{1}}{ax+b} + \frac{A_{2}}{(ax+b)^2}+\dots +\frac{A_{e}}{(ax+b)^e}.
\end{equation*}
\end{remark}

\begin{example}
Compute $\DS\int\frac{x^4-2x^2+4x+1}{x^3-x^2-x+1}\dee x$.
\end{example}

\ifdefined\SOLUTION
\SOLUTION{
\begin{enumerate}
\item Since the rational function is not proper, we begin with polynomial long division.
%\begin{equation*}
%\polylongdiv{x^4-2x^2+4x+1}{x^3-x^2-x+1} 
%\end{equation*}
The result of long division is that
\begin{equation*}
\frac{x^4-2x^2+4x+1}{x^3-x^2-x+1} = (x+1) + \frac{4x}{x^3-x^2-x+1}.
\end{equation*}
\item Next we factor the denominator
\begin{align*}
g(x) &= x^3-x^2-x+1 = (x^3-x^2) -(x-1) \\
&= x^2(x-1) - (x-1) = (x-1)(x^2-1) \\
&= (x-1)(x-1)(x+1) = (x-1)^2(x+1). 
\end{align*}
\item Third, we set up the partial fraction decomposition
\begin{equation*}
\frac{r(x)}{g(x)} = \frac{4x}{(x-1)^2(x+1)} = \frac{A}{x-1} + \frac{B}{(x-1)^2} + \frac{C}{x+1}. 
\end{equation*}
Clearing denominators gives
\begin{equation*}
4x = A(x-1)(x+1) + B(x+1) + C(x-1)^2. 
\end{equation*}
We can recover most of the coefficients by clever substitution.
\begin{itemize}
\item $x = -1$: $-4 = 0 + 0 +4C$, and so $C = -1$. 
\item $x = 1$: $4 = 0 + 2B + 0$, and so $B = 2$.
\end{itemize}
Since we know all but one coefficient, we could find it by any other substitution using the known values of $B$ and $C$.
However, the following method of ``comparing coefficients" is useful to know.
Using the fact that $B=2$ and $C=-1$, we expand to learn that
\begin{align*}
4x &= Ax^2 - A + Bx + B + C(x^2-2x+1) \\
&= (A+C)x^2 + (B-2C)x + (-A + B + C) \\
&= (A-1)x^2 + 4x + (-A+1). 
\end{align*}
In particular,
\begin{equation*}
4x = (A-1)x^2+4x+(-A+1)
\end{equation*}
as polynomials.
We may therefore compare coefficients of like terms.
\begin{itemize}
\item coefficient of $x^2$: $0 = A-1$,
\item coefficient of $x$:   $4 = 4$,
\item coefficient of $1$:	 $0= -A+1$.
\end{itemize}
Thus, we see (in one of two different ways) that $A=1$.
Therefore we have the partial fraction decomposition
\begin{align*}
\frac{x^4-2x^2+4x+1}{x^3-x^2-x+1} 
&= (x+1) + \frac{4x}{x^3-x^2-x+1} \\
&= (x+1) + \frac{1}{x-1} + \frac{2}{(x-1)^2} + \frac{-1}{x+1}, 
\end{align*}
and hence,
\begin{align*}
\int \frac{x^4-2x^2+4x+1}{x^3-x^2-x+1} \dee x 
&= \int(x+1) \dee x + \int\frac{\dee x}{x-1} + \int\frac{2\dee x}{(x-1)^2} + \int\frac{-\dee x}{x+1} \\
&= \frac{x^2}{2} + x + \ln{|x-1|} + -2(x-1)^{-1} - \ln{|x+1|} + C.
\end{align*}
\end{enumerate}
}
\newpage
\else
\newpage
\,
\newpage
\fi


\begin{remark}
If $g(x)$ contains a (possibly repeated) irreducible quadratic factor $(ax^2+bx+c)^e$, then we must also include $e$ terms of the form
\begin{equation*}
\sum_{i=1}^e\frac{A_{i}}{(ax^2+bx+c)^i} = \frac{A_{1}x+B_1}{ax^2+bx+c} + \frac{A_{2}x+B_2}{(ax^2+bx+c)^2}+\dots +\frac{A_{e}x+B_e}{(ax^2+bx+c)^e}.
\end{equation*}
\end{remark}

\begin{example}
Calculate $\DS\int\frac{\dee x}{x(x^2+1)^2}$.
\end{example}
\ifdefined\SOLUTION
\SOLUTION[Solution]{
\begin{enumerate}
    \item The degree of the numerator is 0, which is less than the degree of the denominator, which is 5.  Hence, no long division is needed.
    \item The denominator $g(x) = x(x^2+1)^2$ is already factored into reducibles.
    Note that the discriminant of $x^2+1$ is $D = b^2 - 4ac = 0^2 - 4(1)(1) = -4 < 0$.
    \item Now set up the decomposition:
\begin{equation*}
\frac{1}{x(x^2+1)^2} = \frac{A}{x} + \frac{Bx+C}{x^2+1} + \frac{Dx+E}{(x^2+1)^2}, 
\end{equation*}
and clear denominators to see that
\begin{equation*}
1 = A(x^2+1)^2 + (Bx+C)(x)(x^2+1) + (Dx+E)(x)(x^2+1). 
\end{equation*}
Substituting $x=0$ gives
\begin{align*}
1 = A + 0 + 0
\end{align*}
and so $A=1$.
\end{enumerate}
Unless we're very practiced at working with complex (imaginary) numbers, we won't be able to determine the remaining coefficients by clever substitution.
Instead, substitute $A=1$ and expand to reveal that
\begin{align*}
    1 &= 1(x^2+1)^2 + (Bx+C)(x^2+x) + (Dx^2+Ex) \\
    &= (x^4 + 2x^2 + 1) + (Bx+C)(x^2+x) + (Dx^2+Ex) \\
    &= (1+B)x^4 + Cx^3 + (2+B+D)x^2 + (C+E)x + 1. 
\end{align*}
Comparing coefficients, we have 
\begin{itemize}
\item coefficient of $x^4$: $0 = 1 + B$, 
\item coefficient of $x^3$: $0 = C$,
\item coefficient of $x^2$: $0 = 2 + B + D$,
\item coefficient of $x$: $0 = C+E$,
\item coefficient of $1$: $1 = 1$.
\end{itemize}
Thus, we learn that $B=-1$, $C=0$, $D=-(2+B)=-(2-1)=-1$, and $E=-C=0$.
Therefore,
\begin{align*}
\int\frac{\dee x}{x(x^2+1)^2}
&= \int\frac{\dee x}{x} + \int \frac{(-1)x \dee x}{x^2 + 1} + \int \frac{(-1)x \dee x}{(x^2+1)^2} \\
&= \ln{|x|} -\frac{1}{2}\ln{|x^2 + 1|} +\frac{1}{2}(x^2+1)^{-1} + C \\
&= \ln{|x|} -\frac{1}{2}\ln{(x^2 + 1)} +\frac{1}{2}(x^2+1)^{-1} + C. 
\end{align*}
}
\else
\newpage
\,
\newpage
\fi
\newpage

\begin{remark}
Sometimes a nonrational integrand may be transformed to a rational one by an appropriate change of variables.
Then the method of partial fractions may be used to integrate the result.
\end{remark}

\begin{example}
Evaluate $\DS\int\frac{\sqrt{x+4}}{x}\dee x$ by first transforming the integrand to a rational expression.
\end{example}
\ifdefined\SOLUTION
\SOLUTION[Solution]{
Let $u = \sqrt{x+4}$ so that $\dee u = \frac{\dee x}{2\sqrt{x+4}} = \frac{\dee x}{2u}$, i.e., $2u\dee u = \dee x$.  
Furthermore, $u^2 = x + 4$, and so $x = u^2 - 4$.
Therefore,
\begin{equation*}
\int\frac{\sqrt{x+4}}{x}\dee x 
= \int \frac{u}{u^2 - 4}2u\dee u
= 2\int\frac{u^2\dee u}{u^2+4},
\end{equation*}
which is ready for partial fraction decomposition.
\begin{enumerate}
\item The rational function is not proper, and so we begin with long division.
%\begin{equation*}
%\polylongdiv{X^2}{X^2-4}
%\end{equation*}
The result of the long division is that
\begin{equation*}
\frac{u^2}{u^2-4} = 1 + \frac{4}{u^2-4}
\end{equation*}
\item The denominator $u^2-4 = (u+2)(u-2)$. 
\item The partial fraction decomposition takes the form
\begin{equation*}
\frac{4}{u^2-4} = \frac{A}{u-2}+\frac{B}{u+2}.
\end{equation*}
Clearing denominators, we have
\begin{equation*}
4 =  A(u+2) + B(u-2).
\end{equation*}
We determine the coefficients by ``clever substitution."
\begin{itemize}
\item $u = -2$: $4 = 0 + (-4)B$, and so $B = -1$. 
\item $u = 2$: $4 = 4A + 0$, and so $A = 1$.  
\end{itemize}
Therefore,
\begin{align*}
    \int\frac{\sqrt{x+4}}{x}\dee x &= 2\int \frac{u^2}{u^2-4} \dee u \\
    &= 2 \int \left[1 + \frac{4}{u^2-4} \right] \dee u \\
    &= 2\left[ \int \dee u + \int \frac{\dee u}{u-2} + \int \frac{-\dee u}{u+2}
    \right] \\
    &= 2u + 2\ln{|u-2|} - 2\ln|u+2| + C \\
    &= 2\sqrt{x+4} + 2\ln\left|\sqrt{x+4}-2\right| - 2\ln\left|\sqrt{x+4}+2\right| + C.
\end{align*}
\end{enumerate}
}
\else
\fi
