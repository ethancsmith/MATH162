%!TEX root =  main.tex

\lectureheader{162}{Calculus II}{The natural logarithm}{\textit{Thomas' Calculus} \textsection 7.2}

\begin{theorem}[The fundamental theorem of calculus, part I]
If $f$ is continuous on $[a,b]$, then there is a solution to the differential equation
\begin{equation*}
y'=f(x).
\end{equation*}
In particular, $y=\DS\int_a^xf(t)\dee t$ is the unique solution so that $y(a)=0$.
\end{theorem}

\begin{remark}\,
\begin{itemize}
\item Recall that if $r\ne -1$, then $y= \frac{x^{r+1}}{r+1}$ is a solution to the differential equation $y'=x^r$.
\item There has to be a solution when $r=-1$ since $f(x)=1/x$ is continuous everywhere except for $x=0$, but it is also clear that $y=\frac{x^{r+1}}{r+1}$ \underline{does not work}!
\item So, what are we to do?
\item Answer: just invent a name for the solution we know must exist, and then try to figure out how it behaves.
\end{itemize}
\end{remark}

\begin{definition}
The \textbf{natural logarithm} is the function defined by the rule
\begin{equation*}
\ln(x) = \int_1^x\frac{\dee t}{t}\quad (x>0).
\end{equation*}
\end{definition}

\begin{corollary}
The natural logarithm is the solution to the IVP
\begin{align*}
y' & = \frac{1}{x},\\
y(1) &=0.
\end{align*}
In particular, we have
\begin{equation*}
\frac{\dee}{\dee x}\ln x = \frac{1}{x},\quad (x>0).
\end{equation*}
\end{corollary}

\begin{remark}
A lot of interesting functions are ``invented" in this way:
\begin{enumerate}
\item Some STEM application leads us to build a model that requires the solution to some differential equation (or some other problem).
\item Some theorem (e.g., FTC1) tells us that a solution exists, but it's not any of the functions that we've seen so far.
\item Give it a new name and study it.
\end{enumerate}
\end{remark}

\begin{corollary}
The natural logarithm is continuous, differentiable, strictly increasing, and concave down on its entire domain $(0,\infty)$.
\end{corollary}

\newpage 

\begin{theorem}[Properties of the logarithm]
For all real $a,b>0$ and rational $r$, we have
\begin{align}
 \ln ab &= \ln a + \ln b,\label{log of product}\\
 \ln\frac{a}{b} &= \ln a - \ln b,\\
 \ln\frac{1}{a} &= -\ln a,\\
 \ln a^r &= r\ln a.
\end{align}
\end{theorem}
\ifdefined\SOLUTION
\SOLUTION[Proof of~\eqref{log of product}]{
Let $a,b>0$.
Differentiating with respect to $a$, we observe that on the one hand
\begin{equation*}
\frac{\dee}{\dee a}\ln a = \frac{1}{a},
\end{equation*}
and on the other hand, the chain rule gives
\begin{equation*}
\frac{\dee}{\dee a}\ln ab = \frac{1}{ab}\frac{\dee}{\dee a}(ab) = \frac{b}{ab} = \frac{1}{a}.
\end{equation*}
Whence it follows that 
\begin{equation*}
\ln ab = \ln a + C
\end{equation*}
for some constant (with respect to $a$) $C$.
Evaluating each side of this equation at $a=1$, we observe that
\begin{equation*}
\ln b = \ln 1 + C = C.
\end{equation*}
Substituting $C=\ln b$ above, we have
\begin{equation*}
\ln ab = \ln a + \ln b.\qedhere
\end{equation*}
}
\else
\begin{proof}[Proof of~\eqref{log of product}]\,

\vspace{6in}
\end{proof}
\fi

\newpage

\begin{remark}
If $y$ is a positive, differentiable function of $x$, then the chain rule gives the formula
\begin{equation*}
\frac{\dee }{\dee x}\ln y = \frac{y'}{y}.
\end{equation*}
This expression is called the \textbf{logarithmic derivative} of $y$.
Logarithmic derivatives have many applications including the computation of ordinary derivatives.
\end{remark}

\begin{example}
Suppose that $y = \frac{(x^2+1)(x+3)^{1/2}}{x-1}$ for $x>1$.
Compute $\dee y/\dee x$ without using the product rule or the quotient rule.
\end{example}

\vfill

\begin{example}
Compute $\dee y/\dee x$ if $y=x^x$ for $x>0$.
\end{example}

\vfill

\newpage

\begin{theorem}
If $n$ is a positive integer, then $\ln 2^n > n/2$.
\end{theorem}
\begin{proof}\,

\vspace{2.5in}
\end{proof}

\begin{corollary}
The range of the natural logarithm is $(-\infty, \infty)$, and moreover
\begin{align*}
\lim_{x\to \infty}\ln x &= \infty,\\
\lim_{x\to 0^+}\ln x &=-\infty.
\end{align*}
\end{corollary}
\begin{proof}\,

\vspace{2.5in}
\end{proof}

\begin{corollary}
The range of the natural logarithm is $(-\infty,\infty)$.
\end{corollary}

\begin{definition}
\textbf{Euler's number} (a.k.a. \textbf{Napier's constant}) is the number $\E$ defined by the equation $\ln(\E)=1$.
\end{definition}

\newpage

\begin{theorem}
On any interval not containing zero, we have
\begin{equation*}
\int\frac{\dee x}{x} = \ln|x| + C.
\end{equation*}
\end{theorem}

\begin{corollary}
Where the integrands are continuous, we have
\begin{align}
\int\tan x \dee x &= \ln |\sec x| + C,\\
\int\cot x \dee x &= \ln|\sin x| + C,\\
\int\sec x \dee x &=\ln |\sec x + \tan x| + C,\\
\int\csc x \dee x &= -\ln |\csc x + \cot x| + C.
\end{align}
\end{corollary}
\begin{proof}\,

\vspace{5in}
\end{proof}
