%!TEX root =  main.tex

\lectureheader{162}{Calculus II}{The natural logarithm}{\textit{Thomas' Calculus}  7.2}

\begin{theorem}[The fundamental theorem of calculus, part I]
If $f$ is continuous on $[a,b]$, then there is a solution to the differential equation
\begin{equation*}
y'=f(x).
\end{equation*}
In particular, $y=\DS\int_a^xf(t)\dee t$ is the unique solution so that $y(a)=0$.
\end{theorem}

\begin{remark}\,
\begin{itemize}
\item Recall that if $r\ne -1$, then $y= \frac{x^{r+1}}{r+1}$ is a solution to the differential equation $y'=x^r$.
\item There has to be a solution when $r=-1$ since $f(x)=1/x$ is continuous everywhere except for $x=0$, but it is also clear that $y=\frac{x^{r+1}}{r+1}$ \underline{does not work}!
\item So, what are we to do?
\item Answer: just invent a name for the solution we know must exist, and then try to figure out how it behaves.
\end{itemize}
\end{remark}

\begin{definition}
The \textbf{natural logarithm} is the function defined by the rule
\begin{equation*}
\ln x = \int_1^x\frac{\dee t}{t}\quad (x>0).
\end{equation*}
\end{definition}

\begin{corollary}
The natural logarithm is the solution to the IVP
\begin{align*}
y' & = \frac{1}{x},\\
y(1) &=0.
\end{align*}
In particular, we have
\begin{equation*}
\frac{\dee}{\dee x}\ln x = \frac{1}{x},\quad (x>0).
\end{equation*}
\end{corollary}

\begin{remark}
A lot of interesting functions are ``invented" in this way:
\begin{enumerate}
\item Some STEM application leads us to build a model that requires the solution to some differential equation (or some other problem).
\item Some theorem (e.g., FTC1) tells us that a solution exists, but it's not any of the functions that we've seen so far.
\item Give it a new name and study it.
\end{enumerate}
\end{remark}

\begin{corollary}
The natural logarithm is continuous, differentiable, strictly increasing, and concave down on its entire domain $(0,\infty)$.
\end{corollary}

\newpage 

\begin{theorem}[Properties of the logarithm]
For all real $a,b>0$ and rational $r$, we have
\begin{align}
 \ln ab &= \ln a + \ln b,\label{log of product}\\
 \ln\frac{a}{b} &= \ln a - \ln b,\\
 \ln\frac{1}{a} &= -\ln a,\\
 \ln a^r &= r\ln a.
\end{align}
\end{theorem}
\ifdefined\SOLUTION
\SOLUTION[Proof of~\eqref{log of product}]{
Let $a,b>0$.
Differentiating with respect to $a$, we observe that on the one hand
\begin{equation*}
\frac{\dee}{\dee a}\ln a = \frac{1}{a},
\end{equation*}
and on the other hand, the chain rule gives
\begin{equation*}
\frac{\dee}{\dee a}\ln ab = \frac{1}{ab}\frac{\dee}{\dee a}(ab) = \frac{b}{ab} = \frac{1}{a}.
\end{equation*}
Since the functions $f(a)=\ln a$ and $g(a)=\ln ab$ have the same derivative (with respect to $a$), it follows that 
\begin{equation*}
\ln ab = \ln a + C
\end{equation*}
for some constant (with respect to $a$) $C$.
Evaluating each side of this equation at $a=1$, we observe that
\begin{equation*}
\ln b = \ln 1 + C = C.
\end{equation*}
Substituting $C=\ln b$ above, we see that we have proven the identity
\begin{equation*}
\ln ab = \ln a + \ln b
\end{equation*}
for all positive $a, b$.
}
\else
\begin{proof}[Proof of~\eqref{log of product}]\,

\vspace{6in}
\end{proof}
\fi

\newpage

\begin{remark}
If $y$ is a positive, differentiable function of $x$, then the chain rule gives the formula
\begin{equation*}
\frac{\dee }{\dee x}\ln y = \frac{y'}{y}.
\end{equation*}
This expression is called the \textbf{logarithmic derivative} of $y$.
Logarithmic derivatives have many applications including the computation of ordinary derivatives.
\end{remark}

\begin{example}
Suppose that $y = \frac{(x^2+1)(x+3)^{1/2}}{x-1}$ for $x>1$.
Compute $\dee y/\dee x$ without using the product rule or the quotient rule.
\end{example}

\ifdefined\SOLUTION
\SOLUTION {
Taking the natural log of both sides of the equation gives 
\begin{equation*}
    \ln{y} = \ln{\frac{(x^2+1)(x+3)^{1/2}}{x-1}}.
\end{equation*}
Using the properties of the logarithm, we get 
\begin{equation*}
    \ln{y} = \ln{(x^2 + 1)} + \ln{(x+3)^{1/2}} - \ln{(x-1)} =  \ln{(x^2 + 1)} + \frac{1}{2}\ln{(x+3)} - \ln{(x-1)}.
\end{equation*}
Next, take the logarithmic derivative. So, 
\begin{equation*}
    \frac{y'}{y} = \frac{\dee}{\dee x}\ln{y} = \frac{2x}{x^2+1} + \frac{1}{2}\cdot \frac{1}{x+3}-\frac{1}{x-1}.
\end{equation*}
Multiplying both sides by y gives 
\begin{equation*}
    y' = y\left[ \frac{2x}{x^2+1} + \frac{1}{2(x+3)}-\frac{1}{x-1} \right]
    = \frac{(x^2+1)(x+3)^{1/2}}{x-1}\left[\frac{2x}{x^2+1} + \frac{1}{2(x+3)}-\frac{1}{x-1}\right].
\end{equation*}
}
\else
\vfill
\fi

\begin{example}
Compute $\dee y/\dee x$ if $y=x^x$ for $x>0$.
\end{example}
\ifdefined\SOLUTION
\SOLUTION {
Take the natural log of both sides of the equation.  So, 
\begin{equation*}
    \ln{y} = \ln{x^x}.
\end{equation*}
Using the properties of logarithms, it follows that 
\begin{equation*}
    \ln{y} = \ln{x^x} = x\ln{x}.
\end{equation*}
Taking the logarithmic derivative and using the product rule, 
\begin{equation*}
    \frac{y'}{y} = \frac{\dee}{\dee x}\ln{y} 
    = \frac{\dee}{\dee x}x\ln{x} 
    =\ln{x} + x\frac{1}{x} = \ln{x} + 1.
\end{equation*}
Therefore, 
\begin{equation*}
    y' = y[\ln{x} + 1] = x^x[\ln{x} + 1].
\end{equation*}
}
\else
\vfill
\fi

\newpage

\begin{remark}\,
\begin{itemize}
\item The number $\ln 2$ is an ``invented" number just like $\sqrt 3$.
\item Our mathematical reasoning tells us that such a number has to exist, but it's still sort of mysterious.
\item To help us feel more comfortable, we remove some of the mystery by estimation.
\item For a lower bound, we have
    \begin{equation*}
        \ln 2 
		= \int_{1}^{2} \frac{\dee t}{t} 
		>\int_1^2\frac{\dee t}{2}
		= \frac{1}{2}.
    \end{equation*} 
\item For an upper bound, we can say
    \begin{equation*}
        \ln 2 
		= \int_{1}^{2} \frac{\dee t}{t} 
		<\int_1^2\frac{\dee t}{1}
		= 1.
    \end{equation*} 
\item So, $\ln 2$ is some number between $0.5$ and $1$.
\item Later will develop numerical techniques that will allow us to estimate logarithms to arbitrary precision.
\end{itemize}
\end{remark}

\begin{theorem}
The range of the natural logarithm is $(-\infty, \infty)$, and moreover
\begin{align*}
\lim_{x\to \infty}\ln x &= \infty,\\
\lim_{x\to 0^+}\ln x &=-\infty.
\end{align*}
\end{theorem}

\ifdefined\SOLUTION
\SOLUTION[Proof] {
    Since $\ln{x}$ is always increasing and 
    \begin{equation*}
        \lim_{n\to\infty} \ln{(2^n)} 
		=\lim_{n\to\infty} n\ln 2
        >\lim_{n\to\infty} \frac{n}{2} = \infty,
    \end{equation*}
    we get 
    \begin{equation*}
        \lim_{x\to\infty} \ln{x} = \infty.
    \end{equation*}
    Also, 
    \begin{equation*}
        \lim_{x\to\ 0^+} \ln{x} 
        = \lim_{t\to\infty} \ln{1/t} 
        = \lim_{t\to\infty} -\ln{t} = -\infty. 
    \end{equation*}
By intermediate value theorem, it now follows that the range of the natural logarithm is $(-\infty, \infty)$.
}
\else
\begin{proof}\,

\vspace{2.5in}
\end{proof}
\fi




\begin{definition}
\textbf{Euler's number} (a.k.a. \textbf{Napier's constant}) is the real number $\E$ defined by the equation $\ln(\E)=1$.
\end{definition}

\newpage

\begin{theorem}
On any interval not containing zero, we have
\begin{equation*}
\int\frac{\dee x}{x} = \ln|x| + C.
\end{equation*}
\end{theorem}

\begin{corollary}
Where the integrands are continuous, we have
\begin{align}
\int\tan x \dee x &= \ln |\sec x| + C,\label{tan integral}\\
\int\cot x \dee x &= \ln|\sin x| + C,\\
\int\sec x \dee x &=\ln |\sec x + \tan x| + C,\\
\int\csc x \dee x &= -\ln |\csc x + \cot x| + C.
\end{align}
\end{corollary}

\ifdefined\SOLUTION
\SOLUTION[Proof of~\eqref{tan integral}]{
Note that if $u = \cos{x}$, then $\dee u = -\sin{x} \dee x$.  
Therefore,
\begin{equation*}
\begin{split}
    \int\tan{x} \dee x 
    &= \int_{}^{} \frac{\sin{x}}{\cos{x}} \dee x \\
    &= -\int_{}^{} \frac{\dee u}{u} \\
    &= - \ln{|u|} + C \\
    &= -\ln{|\cos{x}|} + C\\
    &= \ln{(|\cos{x}|^{-1})} + C\\
    &= \ln{|\sec{x}|} + C.
\end{split}
\end{equation*}
}
\else
\begin{proof}[Proof of~\eqref{tan integral}]\,

\vspace{5in}
\end{proof}
\fi

\newpage

%\section*{Exercises}
%
%The following exercises are transcribed from \textit{Thomas' Calculus}, 15e.
%\begin{enumerate}[label=(\arabic*)]
%\item Express $\ln 0.125$ in terms of $\ln 2$ and/or $\ln 3$.
%\item Use the properties of logarithms to simplify the expression $\ln(\sin\theta)-\ln\left(\frac{\sin\theta}{4}\right)$.
%\item Find the derivative of $y$ with respect to $x$ for $y=\ln(5x)$.
%\item Find the derivative of $y$ with respect to $t$ for $y=\ln(t^{11})$.
%\item Find the derivative of $y$ with respect to $x$.
% \begin{equation*}
%y=6\ln\left(\frac{12}{x}\right)
%\end{equation*}
%\item Find the derivative of $y$ with respect to $x$.
% \begin{equation*}
%y=\ln(x^{14})
%\end{equation*}
%\end{enumerate}
