%!TEX root =  main.tex

\lectureheader{162}{Calculus II}{Convergence of Taylor series}{\textit{Thomas' Calculus} \textsection 10.9}

\begin{definition}
Let $f$ be a function with $d$ derivatives at $x=a$, and let 
\begin{equation*}
P_f(x; d, a) = \sum_{n=0}^d\frac{f^{(n)}(a)}{n!}(x-a)^n
\end{equation*}
denote the Taylor polynomial of order $d$ about $x=a$ for $f$.
The \textbf{remainder} is defined by
\begin{equation*}
R_f(x;d,a) = f(x) - P_f(x;d,a).
\end{equation*}
\end{definition}

\begin{remark}
The remainder is the error in the approximation $f(x)\approx P_f(x;d,a)$.
\end{remark}

\begin{theorem}[Taylor's theorem]
Let $f$ be a function with derivatives of all orders at $x=a$, and let $I$ be an interval containing $x=a$.
If
\begin{equation*}
\lim_{d\to\infty}R_f(x; d, a) = 0
\end{equation*}
for all $x\in I$, then
\begin{equation*}
f(x) = \sum_{n=0}^\infty \frac{f^{(n)}(a)}{n!}(x-a)^n
\end{equation*}
for all $x\in I$.
\end{theorem}

\begin{theorem}[Lagrange's remainder theorem]
Let $f$ be a function with $d+1$ derivatives at $x=a$, and let $I$ be an interval containing $x=a$.
Then for every $x\in I$, there is a real number $c$ between $x$ and $a$ so that
\begin{equation*}
R_f(x;d,a) = \frac{f^{(d+1)}(c)}{(d+1)!}(x-a)^{d+1}.
\end{equation*}
\end{theorem}

\begin{remark}\,
\begin{itemize}
\item We can use this theorem to \dots
\begin{enumerate}
\item determine how large $d$ needs to be to guarantee desired accuracy for Taylor approximations, and
\item determine open intervals for which a function is equal to its Taylor series.
\end{enumerate}
\item The difficulty in applying the theorem is that we typically have no idea what $c$ is.
\item We only know that $c$ is between $x$ and $a$, and so we have to identify and then assume the worst case.
\end{itemize}
\end{remark}

\begin{theorem}
The following identities are true:
\begin{align}
\E^x &= \sum_{n=0}^\infty \frac{x^n}{n!}\quad (-\infty < x <\infty),\\
\sin x &=\sum_{n=0}^\infty (-1)^n\frac{x^{2n+1}}{(2n+1)!}\quad (-\infty < x < \infty),\\
\cos x &=\sum_{n=0}^\infty (-1)^n\frac{x^{2n}}{(2n)!}\quad (-\infty < x <\infty),\\
\arctan x &=\sum_{n=0}^\infty (-1)^n\frac{x^{2n+1}}{2n+1}\quad (-1\le x\le 1),\\
\ln(1+ x) &= \sum_{n=1}^\infty (-1)^{n+1}\frac{x^n}{n}\quad (-1< x\le 1).
\end{align}
\end{theorem}

\newpage 

\begin{example}
Use known series to compute a series expansion for $f(x) = \frac{1}{3}(2x + x\cos x)$.
\end{example}

\newpage 

\begin{example}
Use known series to compute a series expansion for $f(x) = \cos(2x)$.
\end{example}

\newpage 

\begin{example}
Use known series to compute a series expansion for $f(x) = \sinh(x)$.
\end{example}

\newpage 

\begin{example}
Use known series to compute the first 4 terms of the Maclaurin series for $f(x) = \E^x\cos x$.
\end{example}

\newpage

\begin{example}
For what values of $x$ is the error in the approximation
\begin{equation*}
\sin x \approx x - \frac{x^3}{3!}
\end{equation*}
smaller than $3\cdot 10^{-4}$?
%(Hint: The Maclaurin series for $\sin x$ is alternating for all $x$.)
\end{example}
