%!TEX root =  main.tex

\lectureheader{162}{Calculus II}{Convergence of Taylor series}{\textit{Thomas' Calculus}  10.9}

\begin{definition}
Let $f$ be a function with $d$ derivatives at $x=a$, and let 
\begin{equation*}
P_f(x; d, a) = \sum_{n=0}^d\frac{f^{(n)}(a)}{n!}(x-a)^n
\end{equation*}
denote the Taylor polynomial of order $d$ about $x=a$ for $f$.
The \textbf{remainder} is defined by
\begin{equation*}
R_f(x;d,a) = f(x) - P_f(x;d,a).
\end{equation*}
\end{definition}

\begin{remark}
The remainder is the signed error in the approximation $f(x)\approx P_f(x;d,a)$.
\end{remark}

\begin{theorem}[Taylor's theorem]
Let $f$ be a function with derivatives of all orders at $x=a$, and let $I$ be an interval containing $x=a$.
If
\begin{equation*}
\lim_{d\to\infty}R_f(x; d, a) = 0
\end{equation*}
for all $x\in I$, then
\begin{equation*}
f(x) = \sum_{n=0}^\infty \frac{f^{(n)}(a)}{n!}(x-a)^n
\end{equation*}
for all $x\in I$.
\end{theorem}

\begin{theorem}[Lagrange's remainder theorem]
Let $f$ be a function with $d+1$ derivatives at $x=a$, and let $I$ be an interval containing $x=a$.
Then for every $x\in I$, there is a real number $c$ between $x$ and $a$ so that
\begin{equation*}
R_f(x;d,a) = \frac{f^{(d+1)}(c)}{(d+1)!}(x-a)^{d+1}.
\end{equation*}
\end{theorem}

\begin{remark}\,
\begin{itemize}
\item We can use this theorem to \dots
\begin{enumerate}
\item determine how large $d$ needs to be to guarantee desired accuracy for Taylor approximations, and
\item determine intervals for which a function is equal to its Taylor series.
\end{enumerate}
\item The difficulty in applying the theorem is that we typically have no idea what $c$ is.
\item We only know that $c$ is between $x$ and $a$, and so we have to identify and then assume the worst case.
\end{itemize}
\end{remark}

\begin{theorem}
The following identities are true:
\begin{align}
\E^x &= \sum_{n=0}^\infty \frac{x^n}{n!}\quad (-\infty < x <\infty),\\
\sin x &=\sum_{n=0}^\infty (-1)^n\frac{x^{2n+1}}{(2n+1)!}\quad (-\infty < x < \infty),\\
\cos x &=\sum_{n=0}^\infty (-1)^n\frac{x^{2n}}{(2n)!}\quad (-\infty < x <\infty),\\
\arctan x &=\sum_{n=0}^\infty (-1)^n\frac{x^{2n+1}}{2n+1}\quad (-1\le x\le 1),\\
\ln(1+ x) &= \sum_{n=1}^\infty (-1)^{n+1}\frac{x^n}{n}\quad (-1< x\le 1).
\end{align}
\end{theorem}

\newpage 

\begin{example}
Use known series to compute a series expansion for $f(x) = \frac{1}{3}(2x + x\cos x)$.
\end{example}

\ifdefined\SOLUTION
\SOLUTION{
Since
\begin{equation*}
\cos x =\sum_{n=0}^\infty (-1)^n\frac{x^{2n}}{(2n)!}\quad (-\infty < x <\infty),
\end{equation*}
it follows that
\begin{equation*}
\begin{split}
f(x) = \frac{1}{3}(2x + x\cos x)
&=\frac{1}{3}\left(2x+x\sum_{n=0}^\infty (-1)^n\frac{x^{2n}}{(2n)!}\right)\\
&=\frac{2}{3}x+\sum_{n=0}^\infty (-1)^n\frac{x^{2n+1}}{3(2n)!}\\
&=\frac{2}{3}x+\frac{x}{3}+\sum_{n=1}^\infty (-1)^n\frac{x^{2n+1}}{3(2n)!}\\
&=x+\sum_{n=1}^\infty (-1)^n\frac{x^{2n+1}}{3(2n)!}
\end{split}
\end{equation*}
for all $x\in\R=(-\infty,\infty)$.
}
\fi

\newpage 

\begin{example}
Use known series to compute a series expansion for $f(x) = \cos(2x)$.
\end{example}

\ifdefined\SOLUTION
\SOLUTION{
Since
\begin{equation*}
\cos x =\sum_{n=0}^\infty (-1)^n\frac{x^{2n}}{(2n)!}\quad (-\infty < x <\infty),
\end{equation*}
it follows that
\begin{equation*}
\begin{split}
f(x) = \cos(2x)
=\sum_{n=0}^\infty (-4)^n\frac{x^{2n}}{(2n)!}
\end{split}
\end{equation*}
for all $x\in\R=(-\infty,\infty)$.
}
\fi

\newpage 

\begin{example}
Use known series to compute a series expansion for $f(x) = \sinh(x)$.
\end{example}

\ifdefined\SOLUTION
\SOLUTION{
Since
\begin{equation*}
\E^x = \sum_{n=0}^\infty \frac{x^n}{n!}\quad (-\infty < x <\infty)
\end{equation*}
it follows that
\begin{equation*}
\begin{split}
f(x) = \sinh(x) 
&= \frac{\E^x - \E^{-x}}{2}\\
&= \frac{1}{2}\left(\sum_{n=0}^\infty \frac{x^n}{n!} - \sum_{n=0}^\infty \frac{(-x)^n}{n!}\right)\\
&= \frac{1}{2}\sum_{n=0}^\infty\left(\frac{x^n}{n!} - \frac{(-1)^nx^n}{n!}\right)\\
&= \sum_{n=0}^\infty\left(\frac{1 - (-1)^n}{2}\right)\frac{x^n}{n!}\\
&= \sum_{k=0}^\infty\frac{x^{2k+1}}{(2k+1)!}\quad (-\infty <x <\infty)
\end{split}
\end{equation*}
since
\begin{equation*}
\frac{1 - (-1)^n}{2} = \begin{cases}
0 & \text{if }n=2k,\\
1 & \text{if }n=2k+1.
\end{cases}
\end{equation*}
}
\fi

\newpage 

\begin{example}
Use known series to compute the first 4 terms of the Maclaurin series for $f(x) = \E^x\cos x$.
\end{example}

\ifdefined\SOLUTION
\SOLUTION{
We have
\begin{equation*}
\begin{split}
f(x) &=\E^x\cos x\\
&=\left(\sum_{n=0}^\infty\frac{x^n}{n!}\right)\left(\sum_{n=0}^\infty(-1)^n\frac{x^{2n}}{(2n)!}\right)\\
&=\left(1+x+\frac{x^2}{2} +\frac{x^3}{6}+\frac{x^4}{24}+\dots\right)\left(1-\frac{x^2}{2}+\frac{x^4}{24}-\dots\right)\\
&=1+\left(1\right)x+\left(\frac{1}{2}-\frac{1}{2}\right)x^2+\left(-\frac{1}{2}+\frac{1}{6}\right)x^3+\left(\frac{1}{24}-\frac{1}{4}+\frac{1}{24}\right)x^4+\dots\\
&=1+x-\frac{1}{3}x^3-\frac{1}{6}x^4+\dots
\end{split}
\end{equation*}
}
\fi

\newpage

\begin{example}
For what values of $x$ is the absolute error in the approximation
\begin{equation*}
\sin x \approx x - \frac{x^3}{3!} = x-\frac{1}{6}x^3
\end{equation*}
smaller than $3\cdot 10^{-4}$?
%(Hint: The Maclaurin series for $\sin x$ is alternating for all $x$.)

\ifdefined\SOLUTION
\SOLUTION{
Note that with $f(x)=\sin x$, $P_f(x;3,0) = x-\frac{1}{6}x^3$.
However, since the even powered terms of the Maclaurin series for $\sin x$ are all zero, 
it also true to say that
\begin{equation*}
P_f(x;4,0) = x - \frac{1}{6}x^3.
\end{equation*}
In other words, we can view this cubic approximation as the order 3 or the order 4 Taylor approximation, and we can use whichever gives us better error estimation.
Viewing the approximation as order 4, Lagrange's remainder theorem tells us that the error 
\begin{equation*}
R_f(x;4,0) = \frac{f^{(5)}(c)}{5!}x^5
\end{equation*}
for some real number $c$ between $0$ and $x$.
Since $|f^{(5)}(c)| = |\cos(c)|\le 1$ for all $x$, it follows that
\begin{equation*}
|R_f(x;4,0)|=\left|\frac{f^{(5)}(c)}{5!}x^5\right|
\le \frac{|x|^5}{120}.
\end{equation*}
Solving $|x|^5/120 < 3\cdot 10^{-4}$,
we see that if $|x|<\sqrt[5]{\frac{360}{10^4}}\approx 0.514$, then
\begin{equation*}
|R_f(x;4,0)|<3\cdot 10^{-4}.
\end{equation*}
}
\fi
\end{example}
