%!TEX root =  main.tex

\lectureheader{162}{Calculus II}{Infinite series}{\textit{Thomas' Calculus} \textsection 10.2}

\begin{definition}\,
Let $\{a_n\}_{n=0}^\infty$ be a sequence of real numbers.
\begin{itemize}
\item The sequence of \textbf{partial sums} is 
\begin{equation*}
s_n = \sum_{k=0}^na_k.
\end{equation*}
\item The limit 
\begin{equation*}
\sum_{k=0}^\infty a_k = \lim_{n\to\infty}s_n
\end{equation*}
is called an \textbf{infinite series}.
\item We say that the series \textbf{converges} if the above limit exists, otherwise we say that the series \textbf{diverges}.
\end{itemize}
\end{definition}
\begin{remark}\,
\begin{itemize}
\item The thing that makes series more difficult than sequences is the fact that the sequence of partial terms is given to us as the sum of an ever-increasing number of terms.
\item To evaluate a series, it is helpful if we can express the partial sums in ``closed form."
\item We have  very few ``tricks" for evaluating the sum of an infinite series.
\item If the sequence of partial sums $s_n$ can be rewritten as
\begin{equation*}
s_n = \sum_{k=0}^n a_k = \sum_{k=0}^n \big(b_{k+1} - b_k\big),
\end{equation*}
then we say that the series is \textbf{telescoping} and
\begin{equation*}
\begin{split}
s_n &= \sum_{k=0}^n \big(b_{k+1} - b_k\big)\\
& = (b_1 - b_0) + (b_2 - b_1) + \dots + (b_{n+1} - b_n)\\
& = b_{n+1} - b_0.
\end{split}
\end{equation*}
\end{itemize}
\end{remark}

\newpage

\begin{example}
Compute the first 3 partial sums of the series $\DS\sum_{k=0}^\infty \frac{1}{(k+1)(k+2)}$.
Then determine if the series converges or diverges.
\end{example}

\newpage

\begin{definition}
A \textbf{geometric series} is a series of the form $\DS\sum_{n=0}^\infty ar^n$, where $a$ and $r$ are fixed real numbers.
\end{definition}

\begin{theorem}[Geometric series theorem]
Let $a,r\in\R$.
The series $\DS\sum_{n=0}^\infty ar^n$ converges if and only if $|r|<1$, in which case
\begin{equation*}
\sum_{n=0}^\infty ar^n = \frac{a}{1-r}.
\end{equation*}
\end{theorem}
\begin{proof}\,

\vspace{6in}
\end{proof}

\newpage 

\begin{example}
Evaluate $\DS\sum_{n=0}^\infty\frac{(-1)^n5}{4^n}$ if possible.
\end{example}

\newpage

\begin{theorem}[Test for divergence]
If $\DS\sum a_n$ converges, then $a_n\to 0$ as $n\to\infty$.
\end{theorem}
\begin{remark}
In practice, we use this theorem in the form of its contrapositive, viz.,
\begin{equation*}
\lim_{n\to\infty} a_n\ne 0 \implies \sum a_n \text{ diverges}.
\end{equation*}
\end{remark}
\begin{proof}\,

\vspace{3in}
\end{proof}
\begin{example}
Explain why the series below diverge.
\begin{enumerate}
\item $\DS\sum_{n=1}^\infty \frac{n+1}{n}$
\vfill
\item $\DS\sum_{n=0}^\infty (-1)^n$
\vfill
\end{enumerate}
\end{example}

\newpage

\begin{theorem}
If $\sum a_n$ and $\sum b_n$ both converge and $k\in\R$, then
\begin{enumerate}
\item $\sum \big(a_n\pm b_n\big) = \sum a_n \pm \sum b_n$,
\item $\sum ka_n = k\sum a_n$.
\end{enumerate}
\end{theorem}

\begin{example}
Evaluate the series below.
\begin{enumerate}
\item $\DS\sum_{n=1}^\infty \frac{3^{n-1} - 1}{6^{n-1}}$
\vfill
\item $\DS\sum_{n=1}^\infty \frac{4}{2^n}$
\vfill
\end{enumerate}
\end{example}
