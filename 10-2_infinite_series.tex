%!TEX root =  main.tex

\lectureheader{162}{Calculus II}{Infinite series}{\textit{Thomas' Calculus}  10.2}

\begin{definition}\,
Let $\{a_n\}_{n=k}^\infty$ be a sequence of real numbers.
\begin{itemize}
\item The sequence of \textbf{partial sums} is 
\begin{equation*}
S_N = \sum_{n=k}^Na_n.
\end{equation*}
\item The limit 
\begin{equation*}
\sum_{n=k}^\infty a_n = \lim_{N\to\infty}S_N
\end{equation*}
is called an \textbf{infinite series}.
\item We say that the series \textbf{converges} if the above limit exists, otherwise we say that the series \textbf{diverges}.
\item In this context, we refer to the $a_n$ as the sequence of \textbf{terms} of the series.
\end{itemize}
\end{definition}

\vfill

\begin{remark}[Caution]
Students that are sloppy with details are bound to confuse themselves.
There are 3 distinct objects associated with every infinite series.
\begin{enumerate}
\item The sequence of terms.
\item The sequence of partial sums.
\item The series itself, which is the limit of the partial sums.
\end{enumerate}
\end{remark}

\vfill

\begin{remark}
Given an infinite series, we are generally interested in 3 main questions.
\begin{enumerate}
\item Does the series converge or diverge?  
\item If the series converges, to what real number does it converge?
\item If the series converges to a real number $L$, how many terms of the sequence do we need so that the approximation
\begin{equation*}
S_N = \sum_{n=k}^N a_n \approx L
\end{equation*}
is accurate to within some $\epsilon$-tolerance?
\end{enumerate}
It is often the case that the first question is hard enough!
\end{remark}

\newpage

\begin{remark}[Telescoping series]\,
\begin{itemize}
\item The thing that makes series more difficult than sequences is the fact that the sequence of partial terms is given to us as the sum of an ever-increasing number of terms.
\item To evaluate a series, it is helpful if we can express the partial sums in ``closed form."
\item We have  very few ``tricks" for evaluating the sum of an infinite series.
\item If the sequence of partial sums $S_n$ can be rewritten as
\begin{equation*}
S_N = \sum_{n=k}^N a_n = \sum_{n=k}^N \big(b_{n+1} - b_n\big),
\end{equation*}
then we say that the series is \textbf{telescoping} and
\begin{equation*}
\begin{split}
S_N &= \sum_{n=k}^N \big(b_{n+1} - b_n\big)\\
& = (\cancel{b_{k+1}} - b_k) 
  + (\cancel{b_{k+2}} - \bcancel{b_{k+1}}) 
  + \xcancel{\dots} 
  + (\cancel{b_N} - \bcancel{b_{N-1}}) 
  + (b_{N+1} - \bcancel{b_N})\\
& = b_{N+1} - b_k.
\end{split}
\end{equation*}
\item This last expression involves a constant number of operations (viz., 1 substraction), and so we say that it is a ``closed form" expression for $S_N$.
\end{itemize}
\end{remark}

\newpage

\begin{example}
Determine the sum of the series $\DS\sum_{k=0}^\infty \frac{1}{(k+1)(k+2)}$.
\end{example}

\ifdefined\SOLUTION
\SOLUTION{
By the method of partial fractions, we write
\begin{equation*}
\frac{1}{(k+1)(k+2)} = \frac{A}{k+1} + \frac{B}{k+2}. 
\end{equation*}
Clearing denominators, we have
\begin{equation*}
1 = A(k+2) + B(k+1).
\end{equation*}
Substituting $k=-1$ yields $A=1$, and substituting $k=-2$ yields $B=-1$.
Whence,
\begin{equation*}
\begin{split}
S_N &= \sum_{k=0}^N \frac{1}{(k+1)(k+2)} 
=\sum_{k=0}^N \left[ \frac{1}{k+1} - \frac{1}{k+2} \right] \\
&=\left[1-\cancel{1/2}\right]
 +\left[\bcancel{1/2}-\cancel{1/3}\right]
 +\left[\bcancel{1/3}-\cancel{1/4}\right]
 +\left[\bcancel{1/4}-\cancel{1/5}\right]
 +\xcancel{\dots}
 +\left[\bcancel{1/(N+1)}-1/(N+2)\right]\\
&= 1 - \frac{1}{N+2} 
\end{split}
\end{equation*}
since the second term of expression in brackets cancels with the first term of the next expression in brackets,
leaving only the first term of the first bracketed expression and the second term of the last bracketed expression.
Another way to see this same idea is to first add all the positive terms and then add all the negative ones.
In other words, we write
\begin{equation*}
S_N = \sum_{k=0}^N \left[ \frac{1}{k+1} - \frac{1}{k+2} \right] 
= \sum_{k=0}^N\frac{1}{k+1} - \sum_{k=0}^N\frac{1}{k+2}.
\end{equation*}
Now that we have $S_N$ as a difference of two sums, we reindex the sum that is being substracted with the substitution $n=k+1$ so that we have
\begin{equation*}
\sum_{k=0}^N\frac{1}{k+2} = \sum_{n=1}^{N+1}\frac{1}{n+1}.
\end{equation*}
Thus, we see that we are summing the same expression as in the first sum but we start at index $1$ (as opposed to $0$) and we end at index $N+1$ (as opposed to $N$).
So, there is nearly perfect cancelation when we substract.
In particular,
\begin{equation*}
S_N = \sum_{k=0}^N\frac{1}{k+1} - \sum_{n=1}^{N+1}\frac{1}{n+1}
= 1 - \frac{1}{N+1}
\end{equation*}
so that only the $k=0$ term of the first sum and the $n=N+1$ term of the second sum survive.
Either way, we have
\begin{equation*}
\sum_{k=0}^\infty\frac{1}{(k+1)(k+2)} = \lim_{N\to\infty} S_N = \lim_{N\to\infty}\left(1-\frac{1}{N+1}\right) = 1.
\end{equation*}
}
\fi
\vfill

\newpage

\begin{definition}
A \textbf{geometric series} is a series of the form $\DS\sum_{n=0}^\infty ar^n$, where $a$ and $r$ are fixed real numbers.
\end{definition}

\begin{theorem}[Geometric series theorem]
Let $a,r\in\R$.
The series $\DS\sum_{n=0}^\infty ar^n$ converges if and only if $|r|<1$, in which case
\begin{equation*}
\sum_{n=0}^\infty ar^n = \frac{a}{1-r}.
\end{equation*}
\end{theorem}

\ifdefined\SOLUTION
\SOLUTION[Proof]{
Observe that
\begin{align*}
 S_N &= \sum_{n=0}^N ar^n,\\ 
rS_N &=r\sum_{n=0}^N ar^n = \sum_{n=0}^\infty ar^{k+1}. 
\end{align*}
Whence,
\begin{equation*}
rS_N-S_N
=\sum_{n=0}^N ar^{n+1}-\sum_{n=0}^N ar^N 
=\sum_{n=0}^N\left(ar^{n+1}-ar^n\right) 
= ar^{N+1}-a.
\end{equation*}
Factoring on each side, we have 
\begin{equation*}
(r-1)S_N
=a(r^{N+1}-1).
\end{equation*}
Therefore, if $r\ne 1$,
\begin{equation*}
S_N=\frac{a(r^{N+1}-1)}{r-1}.
\end{equation*}
If $|r| < 1$, then 
\begin{equation*}
\sum_{n=0}^\infty ar^n
=\lim_{N\to\infty}S_N
= \lim_{N\to \infty} \frac{a(r^{N+1}-1)}{r-1}
=\frac{a}{1-r}.
\end{equation*}
If $r\le -1$ or $r>1$, then 
\begin{equation*}
\sum_{n=0}^\infty ar^n 
=\lim_{N\to\infty}S_N 
= \lim_{N\to \infty} \frac{a(r^{N+1}-1)}{r-1}\quad \text{DNE}.
\end{equation*}
If $r = 1$, then 
\begin{equation*}
\sum_{n=0}^\infty ar^n 
=\lim_{N\to\infty}\sum_{n=0}^N a 
=\lim_{N\to\infty}(N+1)a\quad \text{DNE}.
\end{equation*} 
}
\else
\begin{proof}\,

\vspace{6in}
\end{proof}
\fi
\newpage 

\begin{example}
Evaluate $\DS\sum_{n=0}^\infty\frac{(-1)^n5}{4^n}$ if possible.
\end{example}
\ifdefined\SOLUTION
\SOLUTION{
\begin{equation*}
\sum_{n=0}^\infty\frac{(-1)^n5}{4^n} 
=\sum_{n=0}^\infty 5\left( -\frac{1}{4} \right)^n  
=\sum_{n=0}^\infty 5\left( \frac{-1}{4} \right)^n 
= \frac{5}{1-\left(-\frac{1}{4}\right)} 
= \frac{5}{5/4} = 4.
\end{equation*}
}
\fi
\newpage

\begin{theorem}[Test for divergence]
If $\DS\sum_{n=k}^\infty a_n$ converges, then $a_n\to 0$ as $n\to\infty$.
\end{theorem}
\begin{remark}
In practice, we use this theorem in the form of its contrapositive, viz.,
\begin{equation*}
\lim_{n\to\infty} a_n\ne 0 \implies \sum_{n=k}^\infty a_n \text{ diverges}.
\end{equation*}
\end{remark}
\ifdefined\SOLUTION
\SOLUTION[Proof]{
Suppose that $\DS\sum_{n=k}^\infty a_n$ converges.
Then there is a real number $L$ so that
\begin{equation*}
\lim_{N\to\infty} S_N = \lim_{N\to\infty}\sum_{n=k}^N a_n = L.
\end{equation*}
Whence
\begin{equation*}
S_{N} - S_{N-1} = \sum_{n=k}^{N}a_n - \sum_{n=k}^{N-1} a_n = a_N,
\end{equation*}
and so
\begin{equation*}
\lim_{N\to\infty}a_N = \lim_{N\to\infty}\left(S_N-S_{N-1}\right) = L-L = 0.
\end{equation*}
}
\else
\begin{proof}\,

\vspace{3in}
\end{proof}
\fi

\begin{example}
Explain why the series below diverge.
\begin{enumerate}
\item $\DS\sum_{n=1}^\infty \frac{n+1}{n}$
\ifdefined\SOLUTION
\SOLUTION{
Since
\begin{equation*}
\lim_{n\to\infty} \frac{n+1}{n} 
=\lim_{n\to\infty}(1+1/n)
= 1 \ne 0,
\end{equation*}
the series $\DS\sum_{n=1}^\infty \frac{n+1}{n}$ diverges by the TFD.
}
\else
\fi
\vfill
\item $\DS\sum_{n=0}^\infty (-1)^n$
\ifdefined\SOLUTION
\SOLUTION{
Since $\DS\lim_{n\to\infty} (-1)^n$ diverges by oscillation, 
the series $\DS\sum_{n=0}^\infty (-1)^n$ diverges by the TFD.
}
\else
\fi
\vfill
\end{enumerate}
\end{example}

\newpage

\begin{theorem}
If $\DS\sum_{n=k}^\infty a_n$ and $\DS\sum_{n=k}^\infty b_n$ both converge and $c\in\R$, then
\begin{enumerate}
\item $\DS\sum_{n=k}^\infty \big(a_n\pm b_n\big) = \sum_{n=k}^\infty a_n \pm \sum_{n=k}^\infty b_n$,
\item $\DS\sum_{n=k}^\infty ca_n = c\sum_{n=k}^\infty a_n$.
\end{enumerate}
\end{theorem}

\begin{example}
Evaluate the series below.
\begin{enumerate}
\item $\DS\sum_{n=1}^\infty \frac{3^{n-1} - 1}{6^{n-1}}$
\ifdefined\SOLUTION
\SOLUTION{
Here we use a technique called ``reindexing" which is similar to $u$-substitution.
Letting $k=n-1$, we see that
\begin{align*}
\sum_{n=1}^\infty \frac{3^{n-1} - 1}{6^{n-1}} 
&= \sum_{n=1}^\infty\left[\left(\frac{3}{6}\right)^{n-1} - \left(\frac{1}{6}\right)^{n-1}\right]\\
&= \sum_{n=1}^\infty\left(\frac{1}{2}\right)^{n-1} - \sum_{n=1}^\infty \left( \frac{1}{6}\right)^{n-1}\\
&= \sum_{k=0}^\infty\left(\frac{1}{2}\right)^{k} - \sum_{k=0}^\infty \left( \frac{1}{6}\right)^{k}\\
&= \frac{1}{1 - 1/2} - \frac{1}{1-1/6}\\
&= 2 - 6/5\\
&= 4/5.
\end{align*}
}
\else
\fi
\vfill
\item $\DS\sum_{n=1}^\infty \frac{4}{2^n}$
\ifdefined\SOLUTION
\SOLUTION{
Here again we let $k = n-1$ so that
\begin{align*}
\sum_{n=1}^\infty \frac{4}{2^n} 
&= 4\sum_{n=1}^\infty \left(\frac{1}{2}\right)^n
= 4\sum_{k = 0}^\infty \left( \frac{1}{2}\right)^{k+1}
= 4\sum_{k = 0}^\infty \frac{1}{2}\left( \frac{1}{2}\right)^{k}
= 4\frac{1/2}{1-1/2}
= \frac{2}{1/2}
= 4.
\end{align*}
}
\else
\fi
\vfill
\end{enumerate}
\end{example}
