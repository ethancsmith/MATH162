\parindent 0pt

% packages
\usepackage{amssymb}
\usepackage{siunitx}
\usepackage{cancel}
\usepackage{fullpage}
\usepackage{polynom}
\usepackage{pgfplots}
\pgfplotsset{compat=newest}
\usepackage{tikz}
\usetikzlibrary{positioning} 
\usetikzlibrary{angles, quotes}
\usepackage{float}
\usepackage{centernot}
\usepackage{enumitem}
\setlist[enumerate]{
label=(\alph*),
noitemsep
}
\usepackage{mdframed}
\usepackage{mdframed}
\mdfsetup{
	skipabove=.25\baselineskip,
	skipbelow=.5\baselineskip, 
	%innertopmargin=.5\baselineskip,
	innerbottommargin=.5\baselineskip
 }
\surroundwithmdframed{definition}
\surroundwithmdframed{theorem}
\surroundwithmdframed{axiom}
\surroundwithmdframed{corollary}
\surroundwithmdframed{lemma}
\surroundwithmdframed{remark}
\surroundwithmdframed{quote}

\renewcommand{\thesection}{\thechapter.\arabic{section}}
\makeatletter
\renewcommand\theequation{\thesection.\arabic{equation}} \@addtoreset{equation}{section}
\renewcommand\thefigure{\thesection.\arabic{figure}} \@addtoreset{figure}{chapter}
\renewcommand\thetable{\thesection.\arabic{table}} \@addtoreset{table}{chapter}
\makeatother

\theoremstyle{definition}
\newtheorem{theorem}{Theorem}[section]
\newtheorem{lemma}[theorem]{Lemma}
\newtheorem{proposition}[theorem]{Proposition}
\newtheorem{axiom}[theorem]{Axiom}
\newtheorem{corollary}[theorem]{Corollary}
\newtheorem{definition}[theorem]{Definition}
\newtheorem{example}[theorem]{Example}
\newtheorem{remark}[theorem]{Remark(s)}
\newtheorem{exercise}{Exercise}[section]
\newtheorem{question}[theorem]{Questions}

\newcommand{\R}{\mathbb R}
\newcommand{\Q}{\mathbb Q}
\newcommand{\C}{\mathbb C}
\newcommand{\Z}{\mathbb Z}
\newcommand{\N}{\mathbb N}
\newcommand{\dom}{\mathrm{dom}}
\newcommand{\rng}{\mathrm{rng}}
\newcommand{\Sym}{\mathrm{Sym}}

\newcommand{\dee}{\mkern 2mu \mathrm{d}}
\newcommand{\E}{\mathrm{e}}
\newcommand{\I}{\mathrm{i}}
\renewcommand{\Re}{\mathrm{Re}}
\renewcommand{\Im}{\mathrm{Im}}
\renewcommand{\bar}{\overline}

\DeclareMathOperator{\arccsc}{arccsc}
\DeclareMathOperator{\arcsec}{arcsec}
\DeclareMathOperator{\arccot}{arccot}
\DeclareMathOperator{\csch}{csch}
\DeclareMathOperator{\sech}{sech}
\DeclareMathOperator{\arcsinh}{arcsinh}
\DeclareMathOperator{\erf}{erf}
\DeclareMathOperator{\sinc}{sinc}
\DeclareMathOperator{\Si}{Si}
\DeclareMathOperator{\Li}{Li}

\newcommand{\floor}[1]{\left\lfloor#1\right\rfloor}
\newcommand{\ceil}[1]{\left\lceil#1\right\rceil}
\newcommand{\mchoose}[2]{\left(\!\!\left(\genfrac{}{}{0pt}{}{#1}{#2}\right)\!\!\right)}
\renewcommand{\brace}[2]{\genfrac{\{}{\}}{0pt}{}{#1}{#2}}
\newcommand{\ffactorial}[2]{(#1)_{#2}}
\newcommand{\rfactorial}[2]{#1^{(#2)}}

\newcommand{\DS}{\displaystyle}

% \lectureheader{course number}{course name}{lecture topic}{reading assignment}
\newcommand{\lectureheader}[4]{
\rule{\textwidth}{1pt}
%\vspace{\baselineskip}
{\textsc{MATH #1 (#2)\ \hfill #3}}\\
\textbf{Reading assignment:} #4\newline 
\rule{\textwidth}{1pt}
\vspace{\baselineskip}
}
