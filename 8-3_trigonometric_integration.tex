%!TEX root =  main.tex

\lectureheader{162}{Calculus II}{Trigonometric integration}{\textit{Thomas' Calculus} \textsection 8.3}

\begin{remark}
To evaluate $\int \sin^m x\cos^n x\dee x$ we have the following basic strategy.
\begin{enumerate}
\item If $m=2k+1$ is odd, then we save a factor of $\sin x\dee x = -\dee(\cos x)$ and use identity $\sin^2 x + \cos^2 x = 1$ to write
\begin{equation*}
\begin{split}
\int \sin^m x \cos^n x\dee x &= \int (\sin^2 x)^k\sin x\cos^n x\dee x \\
&= \int (1-\cos^2 x)^k\cos^n x\sin x\dee x\\
& = -\int (1-u^2)^ku^n\dee u,
\end{split}
\end{equation*}
where $u=\cos x$.
\item If $n=2k+1$ is odd, then we save a factor $\cos x\dee x = \dee(\sin x)$ and write
\begin{equation*}
\begin{split}
\int \sin^m x \cos^n x\dee x &= \int\sin^m x(\cos^2 x)^k\cos x\dee x \\
&= \int\sin^m x(1-\sin^2 x)^k\cos x\dee x \\
&= \int u^m(1-u^2)^k\dee u,
\end{split}
\end{equation*}
where $u=\sin x$.
\item If $m$ and $n$ are both even, then we reduce to lower powers using the identities
\begin{align*}
\cos^2 x = \frac{1+\cos 2x}{2},\\
\sin^2 x = \frac{1-\cos 2x}{2}.
\end{align*}
\end{enumerate}
\end{remark}

\begin{example}
Compute $\DS\int\sin^3 x\cos^2 x\dee x$.
\end{example}

\newpage

\begin{example}
Calculate $\DS\int\cos^5 x\dee x$.
\end{example}
\vfill

\begin{example}
Determine $\DS\int\sin^2 x\cos^4 x\dee x$.
\end{example}
\vfill

\newpage

\begin{remark}
To evaluate $\int\tan^m x\sec^n x\dee x$ we have the following basic strategy.
\begin{itemize}
\item If $n=2k$ is even, we save a factor of $\sec^2 x\dee x = \dee(\tan x)$ and use $1+\tan^2 x = \sec^2 x$ to write
\begin{equation*}
\begin{split}
\int\tan^m x\sec^n x\dee x &= \int\tan^m x(\sec^2 x)^{k-1}\sec^2 x\dee x\\
&= \int\tan^m x(1+\tan^2 x)^{k-1}\sec^2 x\dee x\\
&= \int u^m (1+u^2)^{k-1}\dee u,
\end{split}
\end{equation*}
where $u=\tan x$.
\item If $m=2k+1$ is odd, we save a factor of $\sec x\tan x\dee x = \dee(\sec x)$ and write
\begin{equation*}
\begin{split}
\int\tan^m x\sec^n x\dee x &= \int (\tan^{2} x)^k\sec^{n-1} x\sec x\tan x\dee x\\
&= \int (\sec^{2} x -1)^k\sec^{n-1} x\sec x\tan x\dee x\\
&= \int (u^2 -1)^ku^{n-1} \dee u,
\end{split}
\end{equation*}
where $u=\sec x$.
\item The situation is not so clear-cut in the remaining cases.
Integration by parts and a little ingenuity may be required.
\end{itemize}
\end{remark}

\begin{example}
Compute $\DS\int\tan^3 x\dee x$.
\end{example}
\vfill

\newpage

\begin{example}
Evaluate $\DS\int\sec^4 x\tan^4 x\dee x$.
\end{example}
\vfill

\begin{example}
Evaluate $\DS\int\tan^4 x\dee x$.
\end{example}
\vfill

\newpage

\begin{example}
Use integration by parts and the ``wrap-around trick" to compute $\DS\int\sec^3 x\dee x$.
\end{example}
\vfill

\newpage

\begin{remark}
The identities 
\begin{align*}
\cos^2 x = \frac{1+\cos 2x}{2},\\
\sin^2 x = \frac{1-\cos 2x}{2}
\end{align*}
can also be used to eliminate pesky square roots.
\end{remark}

\begin{example}
Evaluate $\DS\int_{\pi/4}^{\pi/2}\sqrt{1+\cos 4\theta}\dee\theta$.
\end{example}

\newpage

\begin{remark}
When integrating products of sines and cosines of differing frequencies, the identities
\begin{align*}
\sin mx\sin nx &= \frac{1}{2}\big(\cos(m-n)x - \cos(m+n)x\big),\\
\sin mx\cos nx &= \frac{1}{2}\big(\sin(m-n)x + \sin(m+n)x\big),\\
\cos mx\cos nx &= \frac{1}{2}\big(\cos(m-n)x + \cos(m+n)x\big)
\end{align*}
may be of use.
\end{remark}

\begin{example}
Evaluate $\DS\int \sin 3x\cos 5x\dee x$.
\end{example}
