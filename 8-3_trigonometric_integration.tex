%!TEX root =  main.tex
\setcounter{chapter}{8}
\setcounter{section}{3}
\setcounter{theorem}{0}
\setcounter{equation}{0}

\lectureheader{162}{Calculus II}{Trigonometric integration}{\textit{Thomas' Calculus}  \thesection}

\begin{remark}
To evaluate $\int \sin^m x\cos^n x\dee x$ we have the following basic strategy.
\begin{enumerate}
\item If $m=2k+1$ is odd, then we save a factor of $\sin x\dee x = -\dee(\cos x)$ and use identity $\sin^2 x + \cos^2 x = 1$ to write
\begin{equation*}
\begin{split}
\int \sin^m x \cos^n x\dee x &= \int (\sin^2 x)^k\sin x\cos^n x\dee x \\
&= \int (1-\cos^2 x)^k\cos^n x\sin x\dee x\\
& = -\int (1-u^2)^ku^n\dee u,
\end{split}
\end{equation*}
where $u=\cos x$.
\item If $n=2k+1$ is odd, then we save a factor $\cos x\dee x = \dee(\sin x)$ and write
\begin{equation*}
\begin{split}
\int \sin^m x \cos^n x\dee x &= \int\sin^m x(\cos^2 x)^k\cos x\dee x \\
&= \int\sin^m x(1-\sin^2 x)^k\cos x\dee x \\
&= \int u^m(1-u^2)^k\dee u,
\end{split}
\end{equation*}
where $u=\sin x$.
\item If $m$ and $n$ are both even, then we reduce to lower powers using the identities
\begin{align*}
\cos^2 x = \frac{1+\cos 2x}{2},\\
\sin^2 x = \frac{1-\cos 2x}{2}.
\end{align*}
\end{enumerate}
\end{remark}

\begin{example}
Calculate $\DS\int\cos^5 x\dee x$.
\end{example}
\ifdefined\SOLUTION
\SOLUTION[Solution]{
Since we have an odd power of cosine, we save one and trade the rest for sines:
\begin{equation*}
    \int\cos^5 x\dee x
    = \int\cos^4x \cos x \dee x
    = \int (1-\sin^2 x)^2 \cos x \dee x.
\end{equation*}
Now let $u = \sin x$.  So, $\dee u = \cos x$.  
Making the substitution gives 
\begin{equation*}
\begin{split}
    \int\cos^5 x\dee x
    &=\int (1-u^2)^2\dee u\\
    &= \int 1 - 2u^2 - u^4 \dee u\\
    &= u - \frac{2}{3}u^3 - \frac{1}{5}u^5 + C\\
    &= \sin x - \frac{2}{3}(\sin x)^3 - \frac{1}{5}(\sin x)^5 + C.
\end{split}
\end{equation*}
}
\fi
\vfill

\newpage

\begin{example}
Compute $\DS\int\sin^3 x\cos^2 x\dee x$.
\end{example}
\ifdefined\SOLUTION
\SOLUTION[Solution]{
Since the power on the sine is odd, we prep for the $u$-sub by saving a sine and trading the rest for cosines:
\begin{equation*}
    \int\sin^3 x\cos^2 x\dee x
    = \int\sin^2 x\cos^2 x \sin x \dee x
    = \int(1-\cos^2 x)\cos^2 x\sin x \dee x.
\end{equation*}
Now let $u = \cos x$.  So, $\dee u = -\sin x \dee x$.  
Making the substitution gives
\begin{equation*}
\begin{split}
\int\sin^3 x\cos^2 x\dee x
   &=-\int (1-u^2)u^2\dee u
    = -\int (u^2 - u^4) \dee u\\
    &= \int (u^4 - u^2) \dee u
    = \frac{1}{5}u^5 - \frac{1}{3}u^3 + C\\
    &= \frac{1}{5}(\cos x)^5 - \frac{1}{3}(\cos x)^3 + C.
\end{split}
\end{equation*}
}
\fi
\vfill

\begin{example}
Determine $\DS\int\sin^2 x\cos^4 x\dee x$.
\end{example}
\ifdefined\SOLUTION
\SOLUTION[Solution]{
Since both powers are even, we ``power-reduce" first:
\begin{align*}
    \int\sin^2 x\cos^4 x\dee x
    &= \int \left( \frac{1-\cos{(2x)}}{2}\right)
    \left(\frac{1+\cos{(2x)}}{2}
    \right)^2 \dee x \\
    &= \frac{1}{8}\int (1- \cos^2(2x))(1+\cos(2x))\dee x \\
    &= \frac{1}{8}\int 1 + \cos(2x) - \cos^2(2x) -\cos^3(2x) \dee x \\
    &= \frac{1}{8}\left(x + \frac{1}{2}\sin(2x) - \int\cos^2(2x)\dee x - \int \cos^3(2x) \dee x
    \right) \\
    &= \frac{1}{8}\left(x + \frac{1}{2}\sin(2x) - \int\frac{1+\cos(4x)}{2}\dee x - \int (1-\sin^2(2x))\cos(2x) \dee x\right). 
\end{align*}
Now let $u = \sin(2x)$.  So, $\dee u = 2\cos(2x)\dee x$.  
Making the substitution gives 
\begin{align*}
    \int\sin^2 x\cos^4 x\dee x
    &= \frac{1}{8}\left[ x + \frac{1}{2}\sin(2x) - \frac{1}{2}\left(x + \frac{1}{4}\sin(4x)\right)  - \frac{1}{2}\int (1-u^2) \dee u
    \right] \\
    &= \frac{1}{8}\left[ x + \frac{1}{2}\sin(2x) - \frac{1}{2}\left(x + \frac{1}{4}\sin(4x)\right)  - \frac{1}{2}(u -\frac{1}{3}u^3)
    \right]  + C\\
    &= \frac{1}{8}\left[ x + \frac{1}{2}\sin(2x) - \frac{1}{2}\left(x + \frac{1}{4}\sin(4x)\right)  - \frac{1}{2}\left(\sin(2x) -\frac{1}{3}(\sin(2x))^3\right) 
    \right]+ C \\  
    &= \frac{1}{8}\left[ x + \frac{1}{2}\sin(2x) - \frac{1}{2}x + \frac{1}{8}\sin(4x)  - \frac{1}{2}\sin(2x) +\frac{1}{6}\sin(2x)^3
    \right]+ C \\ 
    &= \frac{1}{8}\left[ x - \frac{1}{2}x + \frac{1}{8}\sin(4x) +\frac{1}{6}(\sin(2x))^3 
    \right]+ C \\ 
    &= \frac{1}{16}x - \frac{1}{64}\sin(4x) + \frac{1}{48}\sin^3(2x) + C.
\end{align*}
}
\else
\fi
\vfill
\vfill

\newpage

\begin{remark}
To evaluate $\int\tan^m x\sec^n x\dee x$ we have the following basic strategy.
\begin{enumerate}
\item If $n=2k$ is even, we save a factor of $\sec^2 x\dee x = \dee(\tan x)$ and use $1+\tan^2 x = \sec^2 x$ to write
\begin{equation*}
\begin{split}
\int\tan^m x\sec^n x\dee x &= \int\tan^m x(\sec^2 x)^{k-1}\sec^2 x\dee x\\
&= \int\tan^m x(1+\tan^2 x)^{k-1}\sec^2 x\dee x\\
&= \int u^m (1+u^2)^{k-1}\dee u,
\end{split}
\end{equation*}
where $u=\tan x$.
\item If $m=2k+1$ is odd, we save a factor of $\sec x\tan x\dee x = \dee(\sec x)$ and write
\begin{equation*}
\begin{split}
\int\tan^m x\sec^n x\dee x &= \int (\tan^{2} x)^k\sec^{n-1} x\sec x\tan x\dee x\\
&= \int (\sec^{2} x -1)^k\sec^{n-1} x\sec x\tan x\dee x\\
&= \int (u^2 -1)^ku^{n-1} \dee u,
\end{split}
\end{equation*}
where $u=\sec x$.
\item The situation is not so clear-cut in the remaining cases.
Integration by parts and a little ingenuity may be required.
\end{enumerate}
\end{remark}

\begin{example}
Evaluate $\DS\int\sec^4 x\tan^4 x\dee x$.
\end{example}
\ifdefined\SOLUTION
\SOLUTION[Solution]{
\begin{equation*}
    \int\sec^4 x\tan^4 x\dee x
    = \int\sec^2 x\tan^4 x\sec^2 x\dee x 
    = \int (1+\tan^2 x)\tan^4 x\sec^2 x\dee x. 
\end{equation*}
Let $u = \tan x.$  So, $\dee u = \sec^2x\dee x$.  
Making the substitution gives 
\begin{equation*}
\begin{split}
    \int\sec^4 x\tan^4 x\dee x
    &=\int(1+u^2)u^4\dee u \\
    &= \int u^4 + u^6 \dee u\\
    &= \frac{1}{5}u^5 + \frac{1}{7}u^7 + C\\
    &= \frac{1}{5}\tan^5x + \frac{1}{7}\tan^7x + C.
\end{split}
\end{equation*}
}
\else
\fi
\vfill

\newpage

\begin{example}
Compute $\DS\int\tan^3 x\dee x$.
\end{example}
\ifdefined\SOLUTION
\SOLUTION[Solution]{
Here we have a choice of strategies.
If we notice that the power on the tangent is odd, we would prep like this:
\begin{align*}
    \int\tan^3(x)\dee x
    &= \int\frac{\tan^2x\sec x\tan x}{\sec x}\dee x \\
    &= \int \frac{(\sec^2x-1)\sec x \tan x}{\sec x}\dee x.
\end{align*}
Then we let $u = \sec x$, and so $\dee u = \sec x\tan x \dee x$.  
Making the substitution gives 
\begin{align*}
    \int\tan^3(x)\dee x
    &=\int \frac{u^2-1}{u}\dee u\\
    &= \int u - \frac{1}{u}\dee u \\
    &= \frac{1}{2}u^2 - \ln|u| + C \\
    &= \frac{1}{2}\sec^2(x) - \ln|\sec x| + C.
\end{align*}
Alternatively, we might notice that the power of secant is even.
This leads us to adopt the following:
\begin{align*}
    \int\tan^3(x)\dee x
    &= \int (\sec^2(x)-1)\tan x \dee x \\
    &= \int \sec^2(x)\tan(x)\dee x - \int\tan(x) \dee x \\
    &= \frac{1}{2}\tan^2(x) - \ln|\sec x| + C.
\end{align*}
}
\else
\fi
\vfill


\begin{example}
Evaluate $\DS\int\tan^4 x\dee x$.
\end{example}
\ifdefined\SOLUTION
\SOLUTION[Solution]{
Note that if $u = \tan x, \dee u = \sec^2x\dee x$. 
Therefore,
\begin{align*}
    \int \tan^4x\dee x
    &= \int \tan^2x(\sec^2x-1)\dee x \\
    &= \int \tan^2x\sec^2x\dee x - \int \tan^2x\dee x \\
    &= \frac{1}{3}\tan^3x - \int (\sec^2x - 1)\dee x \\
    &= \frac{1}{3}\tan^3x - \tan x + x + C.
\end{align*}
}
\else
\fi
\vfill
\newpage

\begin{example}
Use integration by parts and the ``wrap-around trick" to compute $\DS\int\sec^3 x\dee x$.
\end{example}
\ifdefined\SOLUTION
\SOLUTION[Solution]{
Let $u = \sec x$ and $\dee v = \sec^2 x \dee x.$  So, $\dee u = \sec x\tan x\dee x$ and $v = \tan x$ 
Integration by parts yields 
\begin{align*}
    \int\sec^3x\dee x
    &= \sec x \tan x - \int\tan^2 x\sec x \dee x \\
    &= \sec x \tan x - \int(\sec^2x - 1)\sec x \dee x \\
    &= \sec x \tan x - \int\sec^3x\dee x + \int\sec x \dee x \\
    &= \sec x \tan x - \int\sec^3x\dee x + \ln{\left| \sec x + \tan x\right|}. 
\end{align*}
Note that we now have $-1$ times $\int\sec^3 x\dee x$ on the right, and that means that we win.
Solving for the integral we have 
\begin{equation*}
    2\int\sec^3x\dee x = \sec x\tan x + \ln{\left| \sec x + \tan x\right|} + C,
\end{equation*}
and hence
\begin{equation*}
    \int\sec^3x\dee x = \frac{1}{2}\left(\sec x\tan x + \ln{\left| \sec x + \tan x\right|}\right) + C.
\end{equation*}
}
\else
\fi
\vfill

\newpage

\begin{remark}
The identities 
\begin{align*}
\cos^2 x = \frac{1+\cos 2x}{2},\\
\sin^2 x = \frac{1-\cos 2x}{2}
\end{align*}
can also be used to eliminate pesky square roots.
\end{remark}

\begin{example}
Evaluate $\DS\int_{\pi/4}^{\pi/2}\sqrt{1+\cos 4\theta}\dee\theta$.
\end{example}
\ifdefined\SOLUTION
\SOLUTION[Solution]{
\begin{align*}
    \int_{\pi/4}^{\pi/2}\sqrt{1+\cos 4\theta}\dee\theta
    &= \int_{\pi/4}^{\pi/2}\sqrt{2\cos^2(2\theta)}\dee\theta \\
    &= \sqrt{2}\int_{\pi/4}^{\pi/2}\left| \cos(2\theta)\right| \dee \theta \\
    &= -\sqrt{2}\int_{\pi/4}^{\pi/2} \cos(2\theta)\dee \theta \\
    &= -\sqrt{2}\left[ \frac{\sin(2\theta)}{2}
    \right]_{\pi/4}^{\pi/2} \\
    &= \frac{-\sqrt{2}}{2}\left(\sin(\pi) - \sin(\pi/2)
    \right) \\
    &= -\frac{\sqrt{2}}{2}(0-1) \\
    &= \frac{\sqrt{2}}{2}
\end{align*}
}
\fi
\newpage

\begin{remark}
When integrating products of sines and cosines of differing frequencies, the identities
\begin{align*}
\sin mx\sin nx &= \frac{1}{2}\big(\cos(m-n)x - \cos(m+n)x\big),\\
\sin mx\cos nx &= \frac{1}{2}\big(\sin(m-n)x + \sin(m+n)x\big),\\
\cos mx\cos nx &= \frac{1}{2}\big(\cos(m-n)x + \cos(m+n)x\big)
\end{align*}
may be of use.
\end{remark}

\begin{example}
Evaluate $\DS\int \sin 3x\cos 5x\dee x$.
\end{example}
\ifdefined\SOLUTION
\SOLUTION{
Using the second identity above, we find that
\begin{equation*}
\begin{split}
\int \sin 3x\cos 5x\dee x 
&=\frac{1}{2}\int\left(\sin(-2x)+\sin(8x)\right)\dee x\\
&=\frac{1}{2}\left(\frac{-\cos(-2x)}{-2}+\frac{-\cos(8x)}{8}\right) + C\\
&=\frac{1}{4}\cos(2x)-\frac{1}{16}\cos(8x) + C.
\end{split}
\end{equation*}
Note that we used the fact that cosine is an even function in the last line.
}
\fi
