%!TEX root =  main.tex

\lectureheader{162}{Calculus II}{Improper integrals}{\textit{Thomas' Calculus} \textsection 8.8}

\begin{remark}\,
\begin{itemize}
\item When we defined the definite integral $\DS\int_a^bf(x)\dee x$ as a limit of Riemann sums, we assumed that $[a,b]$ was a finite length interval and that $f$ was continuous on $[a,b]$.
\item Occasionally, it is possible to somewhat relax these two conditions.
Such integrals are referred to as \textbf{improper integrals}.
\end{itemize}
\end{remark}

\begin{definition}[Improper integrals of type I]\,
\begin{itemize}
\item If $\DS\int_a^tf(x)\dee x$ exists for every $t\ge a$, then we define
\begin{equation*}
\int_a^\infty f(x)\dee x = \lim_{t\to\infty}\int_a^t f(x)\dee x
\end{equation*}
provided that the limit exists.
\item If $\DS\int_t^bf(x)\dee x$ exists for every $t\le b$, then we define
\begin{equation*}
\int_{-\infty}^b f(x)\dee x = \lim_{t\to-\infty}\int_t^b f(x)\dee x
\end{equation*}
provided that the limit exists.
\end{itemize}
We say that the integrals above are \textbf{convergent} when they exist; otherwise we say that they are \textbf{divergent}.
\begin{itemize}
\item If there is a real number $c$ so that both $\DS\int_c^\infty f(x)\dee x$ and $\DS\int_{-\infty}^cf(x)\dee x$ are both convergent, then we define
\begin{equation*}
\int_{-\infty}^\infty f(x)\dee x = \int_{-\infty}^cf(x)\dee x + \int_c^\infty f(x)\dee x.
\end{equation*}
\end{itemize}
\end{definition}

\newpage

\begin{example}
What is the area of the region under the curve $y=\DS\frac{\ln x}{x^2}$ over the interval $[1,\infty)$?
\end{example}

\newpage

\begin{theorem}
Let $p\in\R$.
The integral $\DS\int_1^\infty\frac{\dee x}{x^p}$ converges if and only if $p>1$.
\end{theorem}
\begin{proof}\,

\vspace{5in}
\end{proof}

\newpage

\begin{example}
Evaluate $\DS\int_{-\infty}^\infty\frac{\dee x}{1+x^2}$.
\end{example}

\newpage

\begin{theorem}[Test for divergence]
If $\int_a^\infty f(x)\dee x$ converges, then $f(x)\to 0$ as $x\to\infty$.
\end{theorem}
\begin{remark}
In practice, we use the test for divergence in the form of its contrapositive, viz.,
\begin{equation*}
\lim_{x\to\infty} f(x)\ne 0 \implies \int_a^\infty f(x)\dee x \text{ diverges}.
\end{equation*}
\end{remark}

\begin{example}
Show that $\DS\int_1^\infty \left(1-\frac{1}{x}\right)^{x}\dee x$ diverges.
\end{example}

\newpage

\begin{definition}[Improper integrals of type II]\,
\begin{itemize}
\item If $f$ is continuous on $[a,b)$ but discontinuous at $x=b$, then we define
\begin{equation*}
\int_a^b f(x)\dee x = \lim_{t\to b^-}\int_a^t f(x)\dee x
\end{equation*}
provided that the limit exists.
\item If $f$ is continuous on $(a,b]$ but discontinuous at $x=a$, then we define
\begin{equation*}
\int_{a}^b f(x)\dee x = \lim_{t\to a^+}\int_t^b f(x)\dee x
\end{equation*}
provided that the limit exists.
\end{itemize}
We say that the integrals above are \textbf{convergent} when they exist; otherwise we say that they are \textbf{divergent}.
\begin{itemize}
\item If $f$ is continuous on $[a,b]$ except for a discontinuity at $c\in (a,b)$, then we define
\begin{equation*}
\int_{a}^b f(x)\dee x = \int_{a}^cf(x)\dee x + \int_c^b f(x)\dee x.
\end{equation*}
\end{itemize}
\end{definition}

\begin{example}
Determine $\DS\int_2^5\frac{\dee x}{\sqrt{x-2}}$.
\end{example}

\newpage

\begin{example}
Calculate $\DS\int_0^{\pi/2}\sec x\dee x$.
\end{example}
\vfill

\begin{example}
Evaluate $\DS\int_0^3\frac{\dee x}{x-1}$.
\end{example}
\vfill

\newpage

\begin{theorem}[Direct comparison test for integrals]
Suppose that $f$ and $g$ are continuous functions on the interval $[a,\infty)$.
\begin{enumerate}
\item If $0\le f(x)\le g(x)$ for all $x\ge a$ and $\int_a^\infty g(x)\dee x$ converges, then $\int_a^\infty f(x)\dee x$ also converges.
\item If $0\le g(x)\le f(x)$ for all $x\ge a$ and $\int_a^\infty g(x)\dee x$ diverges, then $\int_a^\infty f(x)\dee x$ also diverges.
\end{enumerate}
\end{theorem}
\begin{proof}\,

\vspace{5in}
\end{proof}

\newpage


\begin{example}
Determine if $\DS\int_0^\infty \E^{-x^2}\dee x$ converges or diverges.
\end{example}

\newpage

\begin{remark}
The idea of the direct comparison test can also be applied to improper integrals of type II.
\end{remark}

\begin{example}
Determine if $\DS\int_0^{\pi/2}\frac{\cos x}{\sqrt x}\dee x$ converges or diverges.
\end{example}

\newpage

\begin{remark}\,
\begin{itemize}
\item The test for divergence tells us that if $\int_a^\infty f(x)\dee x$ is going to have a chance at convergence, then we must have $f(x)\to 0$ as $x\to\infty$.
\item The difference between convergence and divergence for the integral is ``how fast" $f(x)$ tends toward zero.
\item The comparison tests are about determining convergence by comparing the relative sizes of the integrands.
\item For the direct comparison test we have to be careful to get the inequalities correct for \textit{all sufficiently large} $x$.
\item The limit comparison tests allows for ``cruder" comparison using limits.
\end{itemize}
\end{remark}

\begin{theorem}[Limit comparison test for integrals]
Suppose that $f$ and $g$ are positive and continuous on $[a,\infty)$.
\begin{enumerate}
\item If $f(x) \asymp g(x)$ as $x\to\infty$ (in particular, if $\DS\lim_{x\to\infty}\frac{f(x)}{g(x)}=L>0$), 
then $\int_a^\infty f(x)\dee x$ and $\int_a^\infty g(x)\dee x$ both converge or both diverge.
\item If $f(x) \lll g(x)$ as $x\to\infty$ (i.e., if $\DS\lim_{x\to\infty}\frac{f(x)}{g(x)}=0$) and $\int_a^\infty g(x)\dee x$ converges, 
then $\int_a^\infty f(x)\dee x$ converges.
\item If $f(x) \ggg g(x)$ as $x\to\infty$ (i.e., if $\DS\lim_{x\to\infty}\frac{f(x)}{g(x)}=\infty$) and $\int_a^\infty g(x)\dee x$ diverges, 
then $\int_a^\infty f(x)\dee x$ diverges.
\end{enumerate}
\end{theorem}

\begin{example}
Determine if $\DS\int_1^\infty\frac{1-\E^{-x}}{x}\dee x$ converges or diverges.
\end{example}

\newpage

\begin{example}
Determine if $\DS\int_4^\infty\frac{x\dee x}{(3x^2+4)^{5/2}}$ converges or diverges.
\end{example}
