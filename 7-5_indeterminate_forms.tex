%!TEX root =  main.tex

\lectureheader{162}{Calculus II}{Indeterminate forms}{\textit{Thomas' Calculus} \textsection 7.5}

\begin{remark}\,
\begin{itemize}
\item If $f(x)\to 0$ and $g(x)\to L\ne 0$ as $x\to a$, then 
\begin{equation*}
\DS\lim_{x\to a}\frac{f(x)}{g(x)} = \frac{0}{L} = 0.
\end{equation*}
\item If $f(x)\to L\ne 0$ and $g(x)\to 0$ as $x\to a$, then 
\begin{equation*}
\DS\lim_{x\to a}\frac{f(x)}{g(x)} \quad \mathrm{DNE}.
\end{equation*}
\item It is possible (but not inevitable) that the second limit is $\pm\infty$, but that is only being more specific about the way in which the limit fails to exist.
\item We \underline{never} write (not even as an intermediate step) that the second limit is equal to $\DS\frac{L}{0}$.
\item It is \underline{wrong} to say that $L/0 = \infty$; don't believe it.
The expression $L/0$ is nonsense.
\end{itemize}
\end{remark}

\begin{example}[Calc I pop quiz]
Compute the limits:
\begin{enumerate}
\item $\DS\lim_{x\to 3}\frac{x^2-9}{x-3}$
\vfill
\item $\DS\lim_{x\to 0}\frac{\sin x}{x}$
\vfill
\end{enumerate}
\end{example}

\newpage

\begin{definition}\,
\begin{itemize}
\item If $\DS\lim_{x\to a}f(x) =  0 = \lim_{x\to a}g(x)$, then we say that the limit $$\DS\lim_{x\to a}\frac{f(x)}{g(x)}$$ exhibits  the \textbf{indeterminate form} ``$\frac{0}{0}$".
\item If $\DS\lim_{x\to a}f(x) = \pm\infty$ and $\DS\lim_{x\to a}g(x) = \pm\infty$, then we say that the limit $$\DS\lim_{x\to a}\frac{f(x)}{g(x)}$$ exhibits  the \textbf{indeterminate form} ``$\frac{\infty}{\infty}$".
\end{itemize}
\end{definition}

\begin{remark}\,
\begin{itemize}
\item There is no simple formula to determine the value of an indeterminate form.
\item The expressions ``$\frac{0}{0}$" and ``$\frac{\infty}{\infty}$" are \underline{not numbers}.
\item Never write that something equals an indeterminate form.
\item To say that the limit has the indeterminate form ``$\frac{0}{0}$" and ``$\frac{\infty}{\infty}$" is to realize that there is a race between the numerator and the denominator.
The existence/value of the limit depends on the relative rate/speed at which the two approach their respective limits.
\end{itemize}
\end{remark}

\begin{theorem}[L'H\^opital's rule]
Suppose that $f$ and $g$ are differentiable with $g'(x)\ne 0$ near $x=a$ (except possibly at $x=a$ itself).
Further suppose that 
\begin{equation*}
\lim_{x\to a}f(x) = 0 = \lim_{x\to a}g(x)
\end{equation*}
or that
\begin{equation*}
\lim_{x\to a}f(x) = \pm\infty = \lim_{x\to a}g(x).
\end{equation*}
Then
\begin{equation*}
\lim_{x\to a}\frac{f(x)}{g(x)} = \lim_{x\to a}\frac{f'(x)}{g'(x)}
\end{equation*}
assuming that the limit on the right side exists (or is $\infty$ or $-\infty$).
\end{theorem}

\begin{remark}\,
\begin{itemize}
\item Applying L'H\^opital's rule when the limit does not exhibit the indeterminate form ``$\frac{0}{0}$" or ``$\frac{\infty}{\infty}$" is not valid and will likely lead to incorrect results.
\item L'H\^opital's rule is also valid for limits at infinity.
\end{itemize}
\end{remark}

\newpage

\begin{example}
Compute the limits below.
\begin{enumerate}
\item $\DS\lim_{x\to 0}\frac{3x - \sin x}{x}$
\vfill
\item $\DS\lim_{x\to 0}\frac{\sqrt{1+x}-1}{x}$
\vfill
\item $\DS\lim_{x\to 0}\frac{\sqrt{1+x}-1-x/2}{x^2}$
\vfill
\end{enumerate}
\end{example}

\newpage 

\begin{example}
Compute the limits below.
\begin{enumerate}
\item $\DS\lim_{x\to \frac{\pi}{2}}\frac{\sec x}{1+\tan x}$
\vfill
\item $\DS\lim_{x\to \infty}\frac{\ln x}{2\sqrt x}$
\vfill
\item $\DS\lim_{x\to \infty}\frac{\E^x}{x^2}$
\vfill
\end{enumerate}
\end{example}

\newpage

\begin{definition}\,
\begin{itemize}
\item If $\DS\lim_{x\to a}f(x) = 0$ and $\DS\lim_{x\to a}g(x) = \infty$, then we say that the limit $$\DS\lim_{x\to a}f(x)g(x)$$ exhibits the \textbf{indeterminate form} ``$0\cdot\infty$".
\item If $\DS\lim_{x\to a}f(x) = \infty = \lim_{x\to a}g(x)$, then we say that the limit $$\DS\lim_{x\to a}\left(f(x)-g(x)\right)$$ exhibits the \textbf{indeterminate form} ``$\infty - \infty$".
\end{itemize}
\end{definition}

\begin{remark}
Indeterminate forms ``$0\cdot\infty$" and ``$\infty - \infty$" may be transformed (via algebra) into the forms ``$\frac{0}{0}$" or ``$\frac{\infty}{\infty}$" where L'H\^opital's rule applies.
\end{remark}

\begin{example}
Compute the limits below.
\begin{enumerate}
\item $\DS\lim_{x\to\infty}x\sin(1/x)$
\vfill
\item $\DS\lim_{x\to 0^+}\sqrt x\ln x$
\vfill
\item $\DS\lim_{x\to 0}\left(\frac{1}{\sin x} - \frac{1}{x}\right)$
\vfill
\end{enumerate}
\end{example}

\newpage

\begin{definition}\,
\begin{itemize}
\item If $\DS\lim_{x\to a}f(x) = 0$ and $\DS\lim_{x\to a}g(x) = 0$, then we say that the limit $$\DS\lim_{x\to a}f(x)^{g(x)}$$ exhibits the \textbf{indeterminate form} ``$0^0$".
\item If $\DS\lim_{x\to a}f(x) = \infty$ and $\DS\lim_{x\to a}g(x) = 0$, then we say that the limit $$\DS\lim_{x\to a}f(x)^{g(x)}$$ exhibits the \textbf{indeterminate form} ``$\infty^0$".
\item If $\DS\lim_{x\to a}f(x) = 1$ and $\DS\lim_{x\to a}g(x) = \infty$, then we say that the limit $$\DS\lim_{x\to a}f(x)^{g(x)}$$ exhibits the \textbf{indeterminate form} ``$1^{\infty}$".
\end{itemize}
\end{definition}

\begin{remark}
The forms above are called indeterminate powers.
Indeterminate powers may be transformed (via the identity $f^g = \exp(g\log f)$) into the form ``$0\cdot \infty$".
\end{remark}

\begin{example}
Compute the limits below.
\begin{enumerate}
\item $\DS\lim_{x\to 0^+}\left(1+\sin 4x\right)^{\cot x}$
\vfill
\item $\DS\lim_{x\to 0^+}x^x$
\vfill
\newpage
\item $\DS\lim_{x\to 0^+}(1+x)^{1/x}$
\vfill
\item $\DS\lim_{x\to \infty}x^{1/x}$
\vfill
\end{enumerate}
\end{example}

\newpage

\begin{remark}\,
\begin{itemize}
\item We should think of indeterminate forms as races between two functions that each want to drive the limit to a different value.
L'H\^opital's rule is one tool to help compare the relative velocities of the functions as they approach their respective limits.
\item Don't forget your algebraic techniques from calc I.
\begin{itemize}
\item Frequently an algebraic solution is much quicker.
\item It is also possible to get stuck in a L'H\^opital cycle that never leads to a final result.
\end{itemize}
\item Most indeterminate forms do not evaluate to numbers, but there is 1 exception.  
By convention, the number $0^0 = 1$, but the form ``$0^0$" is indeterminate.  
\item In any case, never write that a limit is equal to an indeterminate form.
If you do, you are at best writing nonsense.
In any case, you are writing falsehoods.
\end{itemize}
\end{remark}

\begin{example}
Compute $\DS\lim_{x\to\infty}\frac{2^x-3^x}{3^x+4^x}$.
\end{example}

%Determinate or Indeterminate?
%$\infty^\infty$, etc.

