%!TEX root =  main.tex
\setcounter{chapter}{7}
\setcounter{section}{5}
\setcounter{theorem}{0}
\setcounter{equation}{0}

\lectureheader{162}{Calculus II}{Indeterminate forms}{\textit{Thomas' Calculus}  \thesection}

\begin{remark}\,
\begin{itemize}
\item If $f(x)\to 0$ and $g(x)\to L\ne 0$ as $x\to a$, then 
\begin{equation*}
\DS\lim_{x\to a}\frac{f(x)}{g(x)} = \frac{0}{L} = 0.
\end{equation*}
\item If $f(x)\to L\ne 0$ and $g(x)\to 0$ as $x\to a$, then 
\begin{equation*}
\DS\lim_{x\to a}\frac{f(x)}{g(x)} \quad \mathrm{DNE}.
\end{equation*}
\item It is possible (but not inevitable) that the second limit is $\pm\infty$, but that is only being more specific about the way in which the limit fails to exist.
\item We \underline{never} write (not even as an intermediate step) that the second limit is equal to $\DS\frac{L}{0}$.
\item It is \underline{wrong} to say that $L/0 = \infty$; don't believe it.
The expression $L/0$ is nonsense.
\end{itemize}
\end{remark}

\begin{example}[Calc I pop quiz]
Compute the limits:
\begin{enumerate}
\item $\DS\lim_{x\to 3}\frac{x^2-9}{x-3}$
\ifdefined\SOLUTION
\SOLUTION[Solution] {
\begin{equation*}
    \DS\lim_{x\to 3}\frac{x^2-9}{x-3}
    = \DS\lim_{x\to 3}\frac{(x+3)(x-3)}{x-3}
    = \DS\lim_{x\to 3}(x+3)
    = 3+3
    = 6.
\end{equation*}
}
\else
\fi
\vfill

\item $\DS\lim_{x\to 0}\frac{\sin x}{x}$
\ifdefined\SOLUTION
\SOLUTION[Solution] {
Here we just recall the calc I fact:
\begin{equation*}
   \DS\lim_{x\to 0}\frac{\sin x}{x} = 1.
\end{equation*}
}
\else
\fi
\vfill
\end{enumerate}
\end{example}

\newpage

\begin{definition}\,
\begin{itemize}
\item If $\DS\lim_{x\to a}f(x) =  0 = \lim_{x\to a}g(x)$, then we say that the limit $$\DS\lim_{x\to a}\frac{f(x)}{g(x)}$$ exhibits  the \textbf{indeterminate form} ``$\frac{0}{0}$".
\item If $\DS\lim_{x\to a}f(x) = \pm\infty$ and $\DS\lim_{x\to a}g(x) = \pm\infty$, then we say that the limit $$\DS\lim_{x\to a}\frac{f(x)}{g(x)}$$ exhibits  the \textbf{indeterminate form} ``$\frac{\infty}{\infty}$".
\end{itemize}
\end{definition}

\begin{remark}\,
\begin{itemize}
\item There is no simple formula to resolve an indeterminate form.
\item The expressions ``$\frac{0}{0}$" and ``$\frac{\infty}{\infty}$" are \underline{not numbers}.
\item Never write that something equals an indeterminate form.
\item To say that the limit has the indeterminate form ``$\frac{0}{0}$" and ``$\frac{\infty}{\infty}$" is to realize that there is a race between the numerator and the denominator.
The existence/value of the limit depends on the relative rate at which the two approach their respective limits.
\end{itemize}
\end{remark}

\begin{theorem}[L'H\^opital's rule]
Suppose that $f$ and $g$ are differentiable with $g'(x)\ne 0$ near $x=a$ (except possibly at $x=a$ itself).
Further suppose that 
\begin{equation*}
\lim_{x\to a}f(x) = 0 = \lim_{x\to a}g(x)
\end{equation*}
or that
\begin{equation*}
\lim_{x\to a}f(x) = \pm\infty = \lim_{x\to a}g(x).
\end{equation*}
Then
\begin{equation*}
\lim_{x\to a}\frac{f(x)}{g(x)} = \lim_{x\to a}\frac{f'(x)}{g'(x)}
\end{equation*}
assuming that the limit on the right side exists (or is $\infty$ or $-\infty$).
\end{theorem}

\begin{remark}\,
\begin{itemize}
\item Applying L'H\^opital's rule when the limit does not exhibit the indeterminate form ``$\frac{0}{0}$" or ``$\frac{\infty}{\infty}$" is not valid and will likely lead to incorrect results.
\item L'H\^opital's rule is also valid for limits at infinity.
\end{itemize}
\end{remark}

\ifdefined\SOLUTION
\SOLUTION[FAQ] {
\textbf{Question:} Is ``$\frac{0}{\infty}$" an indeterminate form? 
\textbf{Answer:} No, there's no tension.  The numerator is driving the quotient to zero and so is the denominator.  So, $\frac{f(x)}{g(x)}\to 0$. 
\textbf{Question:} Is ``$\frac{\infty}{0}$" an indeterminate form? 
\textbf{Answer:}  No.  Again, there's no tension.  
But the numerator is driving the quotient to infinity and the denominator is driving towards DNE.  
So, the limit $\lim_{x\to a}\frac{f(x)}{g(x)}$ DNE.  
It might be positive or negative infinity, but we need further analysis to say.
}
\else
\fi
\newpage

\begin{example}
Compute the limits below.
\begin{enumerate}
\item $\DS\lim_{x\to 0}\frac{3x - \sin x}{x}$
\ifdefined\SOLUTION
\SOLUTION[Solution]{
\begin{equation*}
    \lim_{x\to 0}\frac{3x - \sin x}{x}
    = \lim_{x\to 0}\frac{3-\cos x}{1}
    = 3 - \cos{(0)}
    = 3-1
    = 2.
\end{equation*}
}
\else
\fi
\vfill

\item $\DS\lim_{x\to 0}\frac{\sqrt{1+x}-1}{x}$
\ifdefined\SOLUTION
\SOLUTION[Solution]{
\begin{equation*}
    \lim_{x\to 0}\frac{\sqrt{1+x}-1}{x}
    = \lim_{x\to 0}\frac{\frac{1}{2}(1+x)^{-\frac{1}{2}}}{1}
    = \frac{1}{2}(1+0)^{-\frac{1}{2}}
    = \frac{1}{2}.
\end{equation*}
}
\else
\fi
\vfill

\item $\DS\lim_{x\to 0}\frac{\sqrt{1+x}-1-x/2}{x^2}$
\ifdefined\SOLUTION
\SOLUTION[Solution]{
\begin{equation*}
\begin{split}
    \lim_{x\to 0}\frac{\sqrt{1+x}-1-x/2}{x^2}
    &= \lim_{x\to 0}\frac{\frac{1}{2}(1+x)^{-\frac{1}{2}}-\frac{1}{2}}{2x}
     = \frac{1}{4}\lim_{x\to 0}\frac{(1+x)^{-\frac{1}{2}}-1}{x}\\
    &= \frac{1}{4}\lim_{x\to 0}\frac{\frac{1 - (1+x)^{\frac{1}{2}}}{(1+x)^\frac{1}{2}}}{x} 
     = \frac{1}{4}\lim_{x\to 0}\frac{1 - (1+x)^{\frac{1}{2}}}{x(1+x)^{\frac{1}{2}}}\\
    &= \frac{1}{4}\lim_{x\to 0} \frac{-\frac{1}{2}(1+x)^{-\frac{3}{2}}\cdot 1}{1}
     = \frac{1}{4}\cdot \left(-\frac{1}{2}\right)\\
    &= -\frac{1}{8}.
\end{split}
\end{equation*}
}
\else
\fi
\vfill
\end{enumerate}
\end{example}

\newpage 

\begin{example}
Compute the limits below.
\begin{enumerate}
\item $\DS\lim_{x\to \frac{\pi}{2}}\frac{\sec x}{1+\tan x}$
\ifdefined\SOLUTION
\SOLUTION[Solution] {
This example teaches us not to become too dependent on L'H\^o pital's rule to the point where we forget our calc I skills:
\begin{equation*}
    \lim_{x\to \frac{\pi}{2}}\frac{\sec x}{1+\tan x}
    = \lim_{x\to \frac{\pi}{2}}\frac{\sec{x}\tan{x}}{\sec^2{x}}
    = \lim_{x\to \frac{\pi}{2}}\frac{\tan{x}}{\sec{x}}
    = \lim_{x\to \frac{\pi}{2}}\frac{\sec^2{x}}{\sec{x}\tan{x}}
    = \lim_{x\to \frac{\pi}{2}}\frac{\sec{x}}{\tan{x}} 
    = \dots
\end{equation*}
Continuing down this path will clearly just take us in circles.
It's algebra and trigonometry to the rescue:
\begin{equation*}
     \lim_{x\to \frac{\pi}{2}}\frac{\sec x}{1+\tan x}
     = \lim_{x\to \frac{\pi}{2}}\frac{\frac{1}{\cos{x}}}{1+\frac{\sin{x}}{\cos{x}}}
     = \lim_{x\to \frac{\pi}{2}}\frac{1}{\cos{x} + \sin{x}}
     = \frac{1}{0+1}
     = 1.
\end{equation*}
}
\else
\fi
\vfill

\item $\DS\lim_{x\to \infty}\frac{\ln x}{2\sqrt x}$
\ifdefined\SOLUTION
\SOLUTION[Solution]{
\begin{equation*}
    \lim_{x\to \infty}\frac{\ln x}{2\sqrt x}
    = \lim_{x\to \infty}\frac{\frac{1}{x}}{2\cdot \frac{1}{2}x^{-\frac{1}{2}}}
    = \lim_{x\to \infty}\frac{\frac{1}{x}}{\frac{1}{x^{\frac{1}{2}}}}
    = \lim_{x\to \infty} \frac{x^{\frac{1}{2}}}{x}
    = \lim_{x\to \infty}\frac{1}{\sqrt{x}}
    = 0.
\end{equation*}
}
\else
\fi
\vfill

\item $\DS\lim_{x\to \infty}\frac{\E^x}{x^2}$
\ifdefined\SOLUTION
\SOLUTION[Solution]{
\begin{equation*}
    \lim_{x\to \infty}\frac{\E^x}{x^2}
    = \lim_{x\to \infty}\frac{\E^x}{2x}
    = \lim_{x\to \infty}\frac{\E^x}{2}
    = \infty.
\end{equation*}
}
\else
\fi
\vfill
\end{enumerate}
\end{example}

\newpage

\begin{definition}\,
\begin{itemize}
\item If $\DS\lim_{x\to a}f(x) = 0$ and $\DS\lim_{x\to a}g(x) = \infty$, then we say that the limit $$\DS\lim_{x\to a}f(x)g(x)$$ exhibits the \textbf{indeterminate form} ``$0\cdot\infty$".
\item If $\DS\lim_{x\to a}f(x) = \infty = \lim_{x\to a}g(x)$, then we say that the limit $$\DS\lim_{x\to a}\left(f(x)-g(x)\right)$$ exhibits the \textbf{indeterminate form} ``$\infty - \infty$".
\end{itemize}
\end{definition}

\begin{remark}
Indeterminate forms ``$0\cdot\infty$" and ``$\infty - \infty$" may be transformed (via algebra) into the forms ``$\frac{0}{0}$" or ``$\frac{\infty}{\infty}$" where L'H\^opital's rule applies.
\end{remark}

\begin{example}
Compute the limits below.
\begin{enumerate}
\item $\DS\lim_{x\to\infty}x\sin(1/x)$
\ifdefined\SOLUTION
\SOLUTION[Solution]{
\begin{equation*}
    \lim_{x\to\infty}x\sin(1/x)
    = \lim_{x\to\infty}\frac{\sin(1/x)}{1/x}
    = \lim_{x\to\infty}\frac{\cos(1/x)\cdot(-x^{-2})}{-x^{-2}}
    = \lim_{x\to\infty}\cos(1/x)
    = 1.
\end{equation*}
However, do not try to transform the equation into a fraction like the following:
\begin{equation*}
    \lim_{x\to\infty}x\sin(1/x)
    = \lim_{x\to\infty}\frac{x}{\csc(1/x)}
    \lim_{x\to\infty}\frac{1}{-\csc(1/x)\cot(1/x)(-x^{-2})} 
    = \dots
\end{equation*}
}
\else
\fi
\vfill

\item $\DS\lim_{x\to 0^+}\sqrt x\ln x$
\ifdefined\SOLUTION
\SOLUTION[Solution]{
\begin{equation*}
    \lim_{x\to 0^+}\sqrt x\ln x
    = \lim_{x\to 0^+}\frac{\ln{x}}{\frac{1}{\sqrt{x}}}
    = \lim_{x\to 0^+}\frac{1/x}{-\frac{1}{2}x^{-\frac{3}{2}}}
    = \lim_{x\to 0^+} -2\sqrt{x}
    = 0.
\end{equation*}
The following is a bad decision, but a good exercise.
\begin{equation*}
    \lim_{x\to 0^+}\sqrt x\ln x
    = \lim_{x\to 0^+}\frac{\sqrt{x}}{\frac{1}{\ln{x}}}
    = \lim_{x\to 0^+}\frac{\frac{1}{2}x^{-\frac{1}{2}}}
    {-(\ln{x})^{-2}\cdot\frac{1}{x}}
    =-\frac{1}{2}\lim_{x\to 0^+}\sqrt x (\ln{x})^2
	=\dots
\end{equation*}
}
\else
\fi
\vfill

\item $\DS\lim_{x\to 0}\left(\frac{1}{\sin x} - \frac{1}{x}\right)$
\ifdefined\SOLUTION
\SOLUTION[Solution]{
\begin{equation*}
\begin{split}
    \lim_{x\to 0}\left(\frac{1}{\sin x} - \frac{1}{x}\right)
    &= \lim_{x\to 0} \frac{x - \sin{x}}{x\sin{x}}
    = \lim_{x\to 0} \frac{1 - \cos{x}}{\sin{x} + x\cos{x}}\\
    &= \lim_{x\to 0} \frac{\sin{x}}{\cos{x} + \cos{x} - x\sin{x}}
    = \frac{0}{1 + 1 - 0}
    = 0.  
\end{split}
\end{equation*}
}
\else
\fi
\vfill
\end{enumerate}
\end{example}

\newpage

\begin{definition}\,
\begin{itemize}
\item If $\DS\lim_{x\to a}f(x) = 0$ and $\DS\lim_{x\to a}g(x) = 0$, then we say that the limit $$\DS\lim_{x\to a}f(x)^{g(x)}$$ exhibits the \textbf{indeterminate form} ``$0^0$".
\item If $\DS\lim_{x\to a}f(x) = \infty$ and $\DS\lim_{x\to a}g(x) = 0$, then we say that the limit $$\DS\lim_{x\to a}f(x)^{g(x)}$$ exhibits the \textbf{indeterminate form} ``$\infty^0$".
\item If $\DS\lim_{x\to a}f(x) = 1$ and $\DS\lim_{x\to a}g(x) = \infty$, then we say that the limit $$\DS\lim_{x\to a}f(x)^{g(x)}$$ exhibits the \textbf{indeterminate form} ``$1^{\infty}$".
\end{itemize}
\end{definition}

\begin{remark}
The forms above are called indeterminate powers.
Indeterminate powers may be transformed (via the identity $f^g = \exp(g\log f)$) into the form ``$0\cdot \infty$".
\end{remark}

\begin{example}
Compute the limits below.
\begin{enumerate}
\item $\DS\lim_{x\to 0^+}\left(1+\sin 4x\right)^{\cot x}$
\ifdefined\SOLUTION
\SOLUTION[Solution]{
\begin{equation*}
    \lim_{x\to 0^+}\left(1+\sin 4x\right)^{\cot x}
    = \lim_{x\to 0^+} \exp(\cot{x}\cdot\ln{(1+\sin{4x})})
    = \exp \left(\lim_{x\to 0^+}\frac{\ln{(1+\sin{4x})}}{\tan{x}}\right)
\end{equation*}
\begin{equation*}
    = \exp \left(\lim_{x\to 0^+} \frac{\frac{1}{1+\sin{4x}}\cdot (\cos{(4x)}\cdot4)}{\sec^2{x}}\right)
    = \exp \left(\frac{4}{1}\right)
    = \E^4.
\end{equation*}
}
\else
\fi
\vfill

\item $\DS\lim_{x\to 0^+}x^x$
\ifdefined\SOLUTION
\SOLUTION[Solution]{
\begin{equation*}
    \lim_{x\to 0^+}x^x
    = \exp \left( \lim_{x\to 0^+} x\ln{x} \right)
    = \exp \left( \lim_{x\to 0^+} \frac{\ln{x}}{x^{-1}} \right)
    = \exp \left( \lim_{x\to 0^+} \frac{1/x}{-x^{-2}} \right) 
\end{equation*}
\begin{equation*}
    = \exp \left( \lim_{x\to 0^+} -x \right)
    = \E^0 
    = 1.
\end{equation*}
}
\else
\fi
\vfill

\newpage
\item $\DS\lim_{x\to 0^+}(1+x)^{1/x}$
\ifdefined\SOLUTION
\SOLUTION[Solution]{
\begin{equation*}
    \lim_{x\to 0^+}(1+x)^{1/x}
    = \exp\left( \lim_{x\to 0^+} \frac{1}{x}\ln{(1+x)} \right)
    = \exp\left( \lim_{x\to 0^+} \frac{\ln{(1+x)}}{x} \right)
\end{equation*}
\begin{equation*}
    = \exp\left( \lim_{x\to 0^+} \frac{\frac{1}{1+x}}{1} 
    \right)
    = \exp (1)
    = \E.
\end{equation*}
}
\else
\fi
\vfill

\item $\DS\lim_{x\to \infty}x^{1/x}$
\ifdefined\SOLUTION
\SOLUTION[Solution]{
\begin{equation*}
    \lim_{x\to \infty}x^{1/x}
    = \exp\left( \lim_{x\to \infty} \frac{1}{x}\ln{x}
    \right)
    = \exp\left( \lim_{x\to \infty} \frac{\frac{1}{x}}{1}
    \right)
    = \exp(0)
    = 1.
\end{equation*}
}
\else
\fi
\vfill
\end{enumerate}
\end{example}

\newpage

\begin{remark}\,
\begin{itemize}
\item We should think of indeterminate forms as races between two functions that each want to drive the limit to a different value.
L'H\^opital's rule is one tool to help compare the relative velocities of the functions as they approach their respective limits.
\item Don't forget your algebraic techniques from calc I.
\begin{itemize}
\item Frequently an algebraic solution is much quicker.
\item It is also possible to get stuck in a L'H\^opital cycle that never leads to a final result.
\end{itemize}
\item Most indeterminate forms do not evaluate to numbers, but there is 1 exception.  
By convention, the number $0^0 = 1$, but the form ``$0^0$" is indeterminate.  
\item In any case, never write that a limit is equal to an indeterminate form.
If you do, you are at best writing nonsense.
In any case, you are writing falsehoods.
\end{itemize}
\end{remark}

\begin{example}
Compute $\DS\lim_{x\to\infty}\left(2^x-3^x\right)$.
\end{example}
\ifdefined\SOLUTION
\SOLUTION{
Guess which term is dominant and factor it out.
\begin{equation*}
\lim_{x\to \infty} \left(2^x - 3^x\right) 
	= \lim_{x\to \infty} 3^x\left((2/3)^x - 1\right) 
	=-\infty.
\end{equation*}
}
\fi
\vfill


\begin{example}
Compute $\DS\lim_{x\to\infty}\frac{2^x-3^x}{3^x+4^x}$.
\end{example}
\ifdefined\SOLUTION
\SOLUTION{
Don't fall for the L'H\^opital's rule trap.
Guess which term is dominant in the numerator and in the denominator and factor each out.
\begin{equation*}
    \lim_{x\to\infty}\frac{2^x-3^x}{3^x+4^x}
    =\lim_{x\to\infty}\frac{3^x\left((2/3)^x-1\right)}{4^x\left((3/4)^x+1\right)}
	=0\cdot\frac{0-1}{0+1} = 0.
\end{equation*}
}
\fi
\vfill


